\documentclass[../main.tex]{subfiles}
\begin{document}
\clearpage
\section*{Список обозначений}
\addcontentsline{toc}{section}{Список обозначений}
\begin{tabularx}{\textwidth}{lY}
	$\mathbb{R}^n$ & пространство действительный векторов размерности $n$; \\ 
	$A^{\top}$ & транспонированная вещественная матрица $A$; \\ 
	$I$ & единичная матрица соответствующей размерности; \\ 
	$0$ & нулевой вектор или нулевая матрица соответствующей размерности; \\
	$(x,y)=x^{\top}y$ & скалярное произведение двух векторов $x,\ y \in \mathbb{R}^n $, $x = (x_1,\dots,x_n)$ ; \\
	$\|x\| $ & евклидова норма вектора $x \in \mathbb{R}^n$, т.е. $\|x\| = (x,x)^{\frac{1}{2}}$; \\
	$\|A\|$ & спектральная норма вещественной  матрицы $A$ размера $n \times n$, индуцированная евклидовой нормой вектора;\\ 
	$\mathbb{L}_1$, $\mathbb{L}_2$ & пространства интегрируемых и интегрируемых с квардратом функций соответственно; \\
	$\|\cdot\|_{\mathbb{L}_1}$, $\|\cdot\|_{\mathbb{L}_2}$ & нормы в пространствах интегрируемых и интегрируемых с квардратом функций соответственно; \\
	$B_X(a,r)$ & замкнутый шар радиуса $r>0$ с центром в точке $a$, $B_X(a, r) = \{x\in X: \|x-a\|_X \leqslant r \}$. Здесь $X$ --- это некоторое линейное пространство с нормой $\|\cdot\|_X$; 
\end{tabularx}
\end{document}