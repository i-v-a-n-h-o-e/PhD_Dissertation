\documentclass[../main.tex]{subfiles}
\begin{document}
\clearpage
\section*{Введение}
\addcontentsline{toc}{section}{Введение}
\textbf{Актуальность темы исследования и степень ее разработанности.} 
Важными задачами теории управления являются задачи с малым параметром.  
Исследования А.\,Н.~Тихонова (1948-1952) \cite{Tihonov1948,Tihonov1952} положили начало теории сингулярных возмущений в динамических системах. 
В его работах изучались неавтономные системы обыкновенных дифференциальных уравнений вида
\begin{gather*}
	\dot{z}(t) = F(z,y, t), \\
	\varepsilon \dot{y}(t)  = G(z,y,t),
\end{gather*}
где $\varepsilon$ --- малый параметр. 
А.\,Н.~Тихонов сформулировал условия, при которых решение <<сингулярно-возмущенной>> задачи приближается к решению «редуцированной» (при $\varepsilon = 0$ ) вместе с заданным начальным условием (функция <<корня>> $z=\phi(y,t)$). 
В частности он ввел понятие условия устойчивости корня алгебраического уравнения $F(z,y,t)=0$. 
В последующих работах 1950–1952 годов эти результаты были обобщены на векторный случай и несколько малых параметров, получены общие формулировки критерия устойчивости корня и асимптотической согласованности решений. 
Фундаментальное значение работ Тихонова заключается в том, что он разработал единый методологический подход к исследованию зависимости решений дифференциальных уравнений от малого параметра. 
Это позволило впоследствии создать стройную теорию асимптотических методов и применить их к широкому спектру задач прикладного характера

Под сингулярно возмущенными системами управления обычно понимают системы, в которых малый параметр при старших производных (или другие параметры) приводит к вырождению порядка уравнений при $\varepsilon\to0$. 
В таких задачах редуцированная система обладает алгебраическим ограничением, требующим дополнительных условий («быстрые» и «медленные» переменные). 
В задачах оптимального управления сингулярность проявляется, например, когда весовой коэффициент при энергозатратах содержит $\varepsilon$ или при разнице масштабов управляющих сигналов. 
Обзору основных результатов, полученных при исследовании сингулярно-возмущенных задач управления, посвящена работа \cite{Dmitriev}.
\todo[inline]{Автор(ы)	Год	Вклад
М.И. Вишик, Л.А. Лястерник	1950	Асимптотические методы для уравнений с малым параметром.
Н.Х. Багирова и др.	1967	Начало исследований асимптотики в сингулярно возмущённых задачах.
А.Б. Васильева	1982	Метод пограничной функции для задач с открытым доменом.
Ф.Т. Черноуско	1968	Первая аппроксимация для задач с ограничениями на управление.
П.В. Кокотович, А. Хаддад	1975	Управляемость систем с медленными и быстрыми режимами.
А.М. Ильин, А.Р. Данилин	1980	Сложные асимптотические разложения при геометрических ограничениях.

 При анализе таких задач используются методы разделения масштабов, в частности:
Метод усреднения по времени (Bogolubov–Krylov–Mitropol’skiĭ) для «медленной» динамики;
Метод граничного слоя (на основе идей Прандтля) для описания резких переходов в «быстрой» подсистеме;
Асимптотические разложения (matched expansions) для согласования внешнего и внутреннего решения;
Принцип максимума Понтрягина и теория оптимального управления для поиска оптимального закона управления при наличии разности масштабов;
Методы устойчивости (Ньютоновское приближение, Lyapunov-функции) для проверки «корневой» и условной устойчивости.
Эти и другие методы составляют современный аппарат исследований сингулярно возмущенных задач управления. (См. также обзор
books.google.com
, где отмечается широкое применение методик разных временных масштабов.)

Особое направление – периодическая оптимизация (Periodic Optimal Control) для систем с малым параметром. Изучаются вопросы, касающиеся периодических по условию или по управлению процессов: можно ли улучшить работу стационарного режима циклическими модификациями, оптимальные периодические процессы высокой частоты, квазипериодические возмущения и т.д. Например, Гуардабасси и Локателли в конференц-томе («Variable Structure Systems – Economics and Biology») кратко рассмотрели классические задачи периодической оптимизации (улучшение стационарной работы за счет циклического управления, оптимальная циклическая работа в режиме релаксации) именно в контексте сингулярно возмущенных систем
link.springer.com
. Они ссылаются на пионерские работы 1960–1970-х (Бэйли и Хорн, Локателли–Ринальди и др.). Например, Локателли и Ринальди (1970) и позднее Ринальди (1970) показали, что в некоторых системах периодическая оптимизация может значительно превосходить статический режим
link.springer.com.}
Под регулярными задачами (или задачами регулярной структуры) понимаются случаи, когда при $\varepsilon\to0$ система не вырождается и редуцированная модель не накладывает алгебраических ограничений. 
То есть малый параметр появляется только в качестве плавной поправки. 
В этом случае возможно применение методов обычного анализа (разложения, критерии оптимальности без специальных условий). 
М.С. Никольский (2005) ввел понятие «очень регулярной задачи оптимального управления», характеризуя оптимизацию со «скользящим» малым параметром, при которой решения зависят гладко от $\varepsilon$. 
В таких задачах обычно выполняется принцип максимума Понтрягина без «сингулярных» участков (то есть оптимальный импульс не возникает), и анализ может вестись стандартными методами вариационного исчисления. 
К регулярным можно отнести, в частности, задачи линейно-квадратичного регулирования (LQR) при плавном введении малых параметров, де-параметризованные задачи сингулярных регуляторов, оптимальные задачи с «плохой» и «хорошей» матрицами (без переключений) и др. 
Такие задачи проще для численного анализа: для них сходимость стандартных схем (метод стрельбы, РК-методы, итеративные решения уравнений Пуансо) сохраняется при малых $\varepsilon$.
\todo[inline]{Cвязать с малым параметром в множествах достижимости}
В теории управления хорошо развиты методы, разработанные для линейных систем. 
В ряде задач оптимального управления для линейных систем удается получать аналитические решения. 
В случае же нелинейных систем, аналитическое решение является скорее исключением, чем правилом. 
Поэтому в задачах управления нелинейными системами часто применяются решения, найденные в линеаризованной постановке. 
Иногда такой подход может быть строго обоснован. 
Например, согласно теореме об устойчивости по первому приближению (см., например, \cite{Barbashin_book}), из асимптотической устойчивости линеаризованной в окрестности положения равновесия системы следует устойчивость (локальная) исходной нелинейной системы. 
При решении задачи стабилизации это позволяет приближенно заменять нелинейную систему ее линеаризацией в окрестности положения равновесия. 
И если линеаризованная система окажется вполне управляемой (стабилизируемой), то линейная обратная связь, стабилизирующая эту систему, будет локально (в некоторой окрестности положения равновесия) стабилизировать и нелинейную систему\cite{Kras_add, Stab_lectures, Khalil, Polyak_book}.  
Однако зачастую, метод линеаризации применяется без должного обоснования, так как соответствующие условия либо сложны для проверки, либо вообще отсутствуют.

Исследованию условий близости линейных систем к нелинейным системам с малым параметром посвящена настоящая работа. 
Основным объектом изучения являются  множества достижимости для аффинно-управляемых систем с малым параметром, описываемых обыкновенными дифференциальными уравнениями, с интегральными ограничениями на управление. 

Системы с малой нелинейностью в правой части обычно называют квазилинейными. 
Изучение таких систем в теории управления началось еще в 1960-х годах \cite{Subbotin, Kiselev, Kras_book}.
Э.\,Г.~Альбрехт исследовал несколько задач для квазилинейных систем \cite{Albrecht3}, в том числе задачу оптимального управления движением \cite{Albrecht1} и игровую задачу о встрече движений \cite{Albrecht2}.
Задачи управления квазилинейными системами также рассматриваются в следующих работах \cite{Dauer, Kremlev, KalininLavrinovich2018, Gabasov}.
Исследование А.\,Г.~Кремлева \cite{Kremlev} посвящено управлению системами с неопределенными начальными условиями, где делается акцент на построении оптимальных стратегий управления.
В работе \cite{KalininLavrinovich2018} предложен новый метод минимизации интегральных квадратичных функционалов на траекториях квазилинейных систем, а в статье Р.\,Ф.~Габасова и соавторов \cite{Gabasov} подробно рассматриваются вопросы оптимизации и устойчивости таких систем в прикладных задачах.
В современных приложениях теории управления, квазилинейные системы возникают при использовании линеаризации обратной связью и стохастистической линеаризации \cite{Ching, Gui}.

\textbf{Цель и задачи исследования.} Изучение поведения множеств достижимости нелинейных систем, содержащих малый параметр, с интегральными ограничениями на управление и исследование их взаимосвязи с множествами достижимости линеаризованных систем.

\textbf{Методология и методы исследования.} Предлагаемые исследования основаны на результатах теории дифференциальных уравнений и математической теории управления, нелинейном и выпуклом анализе.
В работе изучается условие выпуклости нелинейного отображения малого шара в гильбертовом пространстве.
Исследования на эту тему были инициированы Б.Т. Поляком\cite{Polyak2001} и продолжены другими авторами, например, Ю.С. Ледяевым\cite{Ledyaev}.
Полученные результаты иллюстрируются численными примерами.

\textbf{Основные положения, выносимые на защиту:} 
\begin{enumerate}
	\item Для аффинных по управлению систем с интегральными ограничениями на малом интервале времени исследована выпуклость и асимптотика множеств достижимости по выходу.
	Доказано, что при выполнении определенных условий, накладываемых на асимптотику собственных чисел соответствующего грамиана управляемости линеаризованной системы, множества достижимости нелинейной системы будут асимптотически эквиваленты множествам достижимости линеаризованной системы. 
	Разработан метод проверки этих условий, основанный на рекуррентной процедуре вычисления коэффициентов разложения грамиана управляемости в ряд по степеням малого параметра.
	Доказано, что нелинейные системы второго порядка, линеаризация которых приводит к линейным стационарным управляемым системам, обладают выпуклыми множествами достижимости на малых интервалах времени.
	
	\item Исследована задача синтеза обратной связи для нелинейной аффинной по управлению системы с интегрально-квадратичным функционалом. 
	Доказано, что линейная обратная связь, рассчитанная для линеаризованной системы, также обеспечивает локальное решение задачи синтеза для нелинейной системы на достаточно малом промежутке времени.  
	Для этого требуются достаточно строгие ограничения на асимптотику грамиана управляемости, которые совпадают с достаточными условиями, обеспечивающими асимптотическую эквивалентность множеств достижимости (множеств нуль-управляемости). 
	То есть, в этом случае, линейная обратная связь, решающая задачу приведения линеаризованной системы в начало координат за заданное время, будет приводить в начало координат и нелинейную систему из малой окрестности.
	Также, при этих условиях получена оценка для относительных значений погрешности интегрального функционала. 
	Для демонстрации реализации описанного метода синтеза приведен пример нелинейной упругой системы под действием внешней силы. 
	
	\item Для квазилинейных систем с интегральными ограничениями на управление исследована выпуклость множеств достижимости. 
	Доказано, что при выполнении условий, накладываемых на правую часть квазилинейной системы, множества достижимости остаются выпуклыми при малых значениях малого параметра. 
\end{enumerate}


\textbf{Научная новизна.} Все полученные в работе результаты являются новыми.

\textbf{Теоретическая и практическая ценность работы.} Диссертация носит в основном теоретический характер.
Полученные результаты могут использоваться в дальнейших исследованиях задач управления с малым параметром и интегральными ограничениями, а также при разработке численных методов их решения

\textbf{Степень достоверности и апробация результатов.}  Степень достоверности результатов проведенных исследований подтверждается строгостью математических доказательств, а также проведенными вычислительными экспериментами.
Основные результаты, полученные в процессе исследования, докладывались и обсуждались на следующих конференциях:
\begin{enumerate}
	\item Воронежская весеняя математическая школа <<Современные методы теории краевых задач. Понтрягинские чтения–XXX>>, Воронеж, 3–9 мая 2019;
	\item Application of Mathematics in Technical and Natural Sciences (AMiTaNS'11): 11th Intern. Conf., June 20-25, 2019, Albena, Bulgaria;
	\item 51-я, 52-я, 53-я, 54-я, и 55-я Международные молодежные школы-конференции "Современные проблемы математики и ее приложений"(2020, 2021, 2022, 2023, 2024, 2025), Екатеринбург;
	\item III международный семинар, посвященный 75-летию академика А.И.Субботина, Екатеринбург, 26–30 октября 2020.
	\item Динамические системы: устойчивость, управление, оптимизация: материалы Междунар. науч. конф. памяти Р.Ф. Габасова, Минск, 5-10 окт. 2021;
	\item XVI Международная конференция «Устойчивость и колебания нелинейных систем управления» (конференция Пятницкого) 
	\item Теория оптимального управления и приложения (OCTA 2022),
	Екатеринбург, 27 июня – 1 июля 2022 г.
	\item 7-я Международная школа-семинар  «Нелинейный анализ и экстремальные задачи» (NLA-2022), 15-22 июля, Иркутск.
	\item 22nd INTERNATIONAL CONFERENCE MATHEMATICAL OPTIMIZATION THEORY AND OPERATIONS RESEARCH
	(MOTOR 2023) DEDICATED TO 90th ANNIVERSARY OF ACADEMICIAN I.I.EREMIN JULY 2–8, 2023, EKATERINBURG
	\item ВСПУ  XIV Всероссийское совещание по проблемам управления Россия, Москва, ИПУ РАН, 17-20 июня 2024
	\item Международная конференция «Динамические системы: устойчивость, управление, дифференциальные игры» (SCDG2024), посвященная 100-летию со дня рождения академика Н.Н. Красовского, 9 сентября 2024 г. - 13 сентября 2024 г.  г. Екатеринбург
	\item Современные проблемы математики и ее приложений.Международная (56-я Всероссийская) молодежная школа-конференция памяти ученого и учителя Александра Георгиевича Гейна (29.01.1950-23.01.2025) 2 февраля 2025 г. - 18 февраля 2025 г.
	г. Екатеринбург
\end{enumerate}

\textbf{Публикации.}

\textbf{Личный вклад.} Все основные результаты диссертации получены автором самостоятельно.  В совместных работах научному руководителю М.И. Гусеву принадлежат постановки задач и общая схема их исследования, формулировки и доказательства результатов принадлежат автору диссертации;

\textbf{Структура и объем работы.}

\textbf{Краткое содержание работы.}  \textbf{Глава 1. } 
Первая глава посвящена множествам достижимости нелинейных систем с интегральными ограничениями. 
Здесь даются основные понятия, описываются предположения и свойства решений исследуемых систем, которые используются в работе. 
Вводится отображение, действующее из пространства управлений в пространство концов траекторий нелинейной системы. 
Приводится условия выпуклости образа малого шара в гильбертовом пространстве при его нелинейном отображении, полученное в \cite{Polyak2001}. 
Использование этого условия позволило Б.Т. Поляку получить условие выпуклости множеств достижимости при малых интегральных ограничениях на управление \cite{Polyak2004}.
Множество достижимости нелинейной системы рассматривается как образ шара, соответствующего допустимым управлениям, при его нелинейном отображении, заданном решениями системы.
При помощи замены времени удается сформулировать условия выпуклости множеств достижимости рассматриваемых нелинейных систем на малых интервалах времени. 
Проверка этих условий требует изучения асимптотики наименьшего собственного числа грамиана управляемости линеаризованной системы.
Один из возможных способов проверки асимптотики собственных чисел грамиана предлагается в последнем разделе этой главы. 
Там же приводится доказательство выпуклости множеств достижимости на малых интервалах времени для некоторых классов нелинейных систем второго порядка. 

\textbf{Глава 2. }
Вторая глава состоит из двух разделов, связанных общим понятием --- асимптотической эквивалентностью множеств достижимости. 

В первом разделе этой главы исследуются множества достижимости по выходу нелинейных систем на малых интервалах времени.
Используя понятие асимптотической эквивалентности множеств\cite{Ovs}, основанное на расстоянии Банаха-Мазура\cite{Thompson}, доказывается близость множеств достижимости нелинейных систем по выходу соответствующим множествам достижимости линеаризованных систем на малых интервалах времени. 

В следующем разделе предлагается метод решения задачи локального синтеза для аффинных по управлению систем на малом интервале времени.  
Этот метод основан на линеаризации исходной нелинейной системы в окрестности точки равновесия. 
Линеаризация часто применяется для решения различных задач управления, таких как задача стабилизации \cite{Kras_add,Khalil}, стохастические и численные методы управления \cite{Roxin,EKF,denBerg,Pang}, в подходах, основанных на предсказывающей модели (MPC) \cite{Murillo,LTV_MPC} и т.д.

В этом разделе изучается задача синтеза управления с интегральным квадратичным функционалом. 
Отдельно отметим, что задача рассматривается на конечном и, более того, малом интервале времени. 
Цель управления -- перевести систему в начало координат за заданное время, обеспечив при этом минимальное значение функционала. 
Линейное управление с обратной связью, найденное для линеаризованной системы, используется в качестве входа исходной нелинейной системы. 
Для линейной системы управления оптимальная обратная связь является линейной по состоянию, и ее коэффициент усиления неограниченно возрастает при приближении к конечному моменту времени. 
Последнее затрудняет обоснование применимости метода линеаризации. 
В этом случае необходимо выполнение условий, имеющих вид ограничений на асимптотику грамиана управляемости линеаризованной системы, в отличие, например, от задачи стабилизации, в которой управляемость (стабилизируемость) линеаризованной системы приводит к стабилизируемости нелинейной системы. 
Эти условия совпадают с условиями асимптотической эквивалентности для множеств достижимости (множеств нуль-управляемости) нелинейных и линеаризованных систем, которые приведены в первой половине главы. 
В работе \cite{GusevOsipov} было показано, что при этих условиях управление в виде линейной обратной связи по состоянию приводит все траектории, выходящие из некоторой окрестности нуля, к нулю, если интервал времени управления достаточно мал. 

В  диссертации приведено обобщение результата \cite{GusevOsipov}, которое было опубликовано в \cite{GusevOsipovMotor}. 
Предлагаемое достаточное условие имеет форму неравенства с некоторым несобственным интегралом.
Подынтегральная функция выбирается так, чтобы ограничивать сверху отношение наибольшего к корню из наименьшего собственного числа грамиана управляемости линеаризованной системы, которая содержит малый параметр. 
Выбор этой функции позволяет охватить более широкий класс систем управления, а условия из \cite{GusevOsipov} получены здесь как частный случай для определенного значения параметра.

Для линеаризованной системы рассматриваемый линейный регулятор обеспечивает минимальное значение интегрального функционала для любого начального состояния. 
Для нелинейной системы это не так, поэтому важно получить оценку значения функционала. 
Это и было сделано в заключительном разделе второй главы, где была исследована связь между значениями интегрального функционала для траекторий нелинейной и линеаризованной систем и дана оценка относительной погрешности. 

\textbf{Глава 3. }
Предметом исследования в этой главе являются множества достижимости квазилинейных систем с интегрально-квадратичными ограничениями.

Как и другие главы, третья глава опирается на работы Б.\,Т.~Поляка\cite{Polyak2001}, в которых были получены достаточные условия выпуклости нелинейного отображения малого гильбертова шара.
%These conditions were further applied to problems in control theory, demonstrating that the reachable set of a nonlinear system exhibits convexity given a sufficient small control resource, provided that the linearized system is controllable. A series of papers \cite{GusOsSteklov, GusevUMJ, Osipov, GusevOsipov} used the convexity conditions of the small ball mapping to investigate the reachable sets of nonlinear systems under integral constraints over small time intervals. 
% In this case, it is important to note that the controllability of the linearized system alone does not guarantee convexity of the reachable sets for the nonlinear system. 
% Additional conditions related to the asymptotic behavior of the eigenvalues of the controllability Gramian of the linearized system need to be imposed.  
% Once these conditions are fulfilled, the reachable sets of the nonlinear system not only exhibit convexity but are also asymptotically equivalent to the reachable sets of the linearized system.

В предыдущим главах исследовались множества достижимости нелинейных систем  с малым ресурсом управления или на малом интервале времени.
В этой главе обсуждается еще один, третий вариант задачи о выпуклости множеств достижимости нелинейных систем с малым параметром, а именно, с малой нелинейностью в правой части. 

В этой главе изучается выпуклость множеств достижимости квазилинейных систем с интегральными ограничениями в пространстве $\mathbb{L}_2$. 
Как и в предыдущих главах,  исследование сводится к анализу нелинейного отображения из пространства управлений в пространство концов траекторий, порожденных этими управлениями.
В таком случае, множество достижимости --- это образ гильбертова шара при применении к его точкам этого отображения. 
Основной особенностью отображения, определенного решением квазилинейной системы является тот факт, что при нулевом значении параметра, это отображение становится линейным.
Доказано, что для того, чтобы образ шара сохранял свою выпуклость при малых значениях малого параметра, достаточно, чтобы производная нелинейной части отображения была липшицевой. 
Схема доказательства этого утверждения похожа на схему доказательства основной теоремы в \cite{Polyak2001}.
\pagebreak
\end{document}