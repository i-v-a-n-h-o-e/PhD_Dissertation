\documentclass[../main.tex]{subfiles}
\begin{document}
\clearpage
\begin{thebibliography}{99}
    \subsection*{Публикации автора по теме диссертации}
    \addcontentsline{toc}{subsection}{Статьи автора по теме диссертации}

    % 2019

    \bibitem{AIP}
    Gusev, M.I. 
    On Convexity of Small-time Reachable Sets of Nonlinear Control Systems / 
    M.I. Gusev, I.O. Osipov // 
    AIP Conference Proceedings. 
    2019. 
    Vol.2164 : 
    Application of Mathematics in Technical and Natural Sciences (AMiTaNS’11): 
    11th Intern. Conf., 
    June 20-25, 2019, Albena, Bulgaria. 
    Art. no.060007. 9 p.

    \bibitem{GusevOsipovTrudy}
    Гусев, М.И. 
    Асимптотическое поведение множеств достижимости на малых временных промежутках / 
    М.И.Гусев, И.О.Осипов // 
    Труды Ин-та математики и механики. 
    2019. Т. 25, № 3. С. 86-99. 
    \hrefdoi{10.21538/0134-4889-2019-25-3-86-99}
    
    % 2020
    \bibitem{SubbotinConf}
    Гусев, М.И. 
    Об асимптотике множеств достижимости на малых временных промежутках / 
    М.И.Гусев, И.О.Осипов // 
    Теория управления и теория обобщенных решений уравнений Гамильтона-Якоби: материалы III международного семинара, посвященного 75-летию академика А.И.Субботина 
    Екатеринбург, 26–30 октября 2020. 
    Екатеринбург: ИММ УрО РАН, 2020. С.142-145.

    \bibitem{GusOsSteklov}
    Gusev~M.\,I., Osipov I.\,O.: Asymptotic behavior of reachable sets on small time intervals. \emph{Proc. Steklov Inst. Math}, 2020, Vol. 309, Suppl. 1, P. 52--64.  \hrefdoi{10.1134/S0081543820040070}

    %2021
    \bibitem{Osipov}
    Осипов И.О. 
    О выпуклости множеств достижимости по части координат нелинейных управляемых систем на малых промежутках времени / 
    И.О. Осипов // 
    Вестник Удмуртского университета. Серия Математика. Механика. Компьютерные науки. 
    2021. Т. 31. Вып. 2. C.210-225.
    Osipov~I.\,O. On the convexity of the reachable set with respect to a part of coordinates at small time intervals. 
    \emph{Vestn. Udmurtsk. Univ. Mat. Mekh. Komp. Nauki}, 2021,  Vol. 31, No. 2, P. 210--225.
    \hrefdoi{10.35634/vm210204}

    \bibitem{Voronezh}
    Осипов, И. О. 
    Об асимптотике собственных чисел грамиана управляемости линейной системы с малым параметром / 
    И. О. Осипов // 
    Итоги науки и техники. Сер. Соврем. математика и ее прил. Тем. обзоры. 
    2021. Т.191: Материалы Воронежской весенней мат. шк. «Современные методы теории краевых задач. Понтрягинские чтения–XXX», 
    Воронеж, 3–9 мая 2019. Ч. 2. С.115–122.

    \bibitem{Minskconf}
    Гусев М.И. Асимптотика множеств достижимости нелинейных управляемых систем на малых промежутках времени / М.И.Гусев, И.О.Осипов // Динамические системы: устойчивость, управление, оптимизация: материалы Междунар. науч. конф. памяти Р.Ф. Габасова, Минск, 5-10 окт. 2021. Минск: Изд. центр БГУ, 2021. С.85-87. 

    %2022
    \bibitem{GusevOsipov} 
    М. И. Гусев, И. О. Осипов.
    О задаче локального синтеза для нелинейных систем с интегральными ограничениями // 
    Вестник Удмуртского университета. Математика. Механика. Компьютерные науки. 2022
    \textbf{32}(2), 171–186 (2022).
    \hrefdoi{10.35634/vm220202}

    \bibitem{GusevOsipovPyat}
    Gusev, M. 
    On Local Control Synthesis for Nonlinear Systems under Integral Constraints / 
    M.Gusev, I.Osipov // 
    Stability and Oscillations of Nonlinear Control Systems (Pyatnitskiy's Conference): 
    16th International Conference, 
    June 1 - 3, 2022, Moscow, Russian Federation. Moscow, 
    2022. P.1-4, 

    \bibitem{GusevOsipovOCTA}
    Gusev, M.I. 
    Asymptotics of Small-Time Reachable Sets and a Problem Of Local Control Synthesis / 
    M.I. Gusev, I.O. Osipov // 
    Теория оптимального управления и приложения - (OCTA 2022): 
    Междунар. конф., Екатеринбург, 27 июня–1 июля 2022: 
    материалы. С. 300-303

    %2023
    \bibitem{GusevOsipovMotor}
    Gusev, M. 
    Approximate Solution of Small-Time Control Synthesis Problem Based on Linearization / 
    M. Gusev, I. Osipov // 
    Lecture Notes in Computer Science. Cham: Springer, 
    2023. Vol 13930: 
    Mathematical Optimization Theory and Operations Research. 
    MOTOR 2023 / eds. M. Khachay, et al.
     P. 362-377.

    \bibitem{OsipovUMJ}
    Osipov I.O. 
    Convexity of Reachable Sets of Quasilinear Systems /
    I.O.Osipov // 
    Ural Math. J. 
    2023. Vol. 9, No 2. P. 141–156. 
    DOI: 10.15826/umj.2023.2.012


\subsection*{Цитированные статьи}
\addcontentsline{toc}{subsection}{Библиография}

\bibitem{Albrecht1}
Albrecht~E.\,G. On the optimal motion control of quasilinear systems, \emph{Differential Equations}, 1969, Vol.5, No. 3 , P. 430-442 (in Russian)

\bibitem{Albrecht2}
Albrecht~E.\,G. The coming together of quasilinear objects in the regular case, \emph{Differential Equations}, 1971, vol. 7, no. 7, 1171-1178 (in Russian)

\bibitem{Albrecht3}
Albrecht~E.\,G. \emph{Metod Lyapunova-Puankare v zadachah optimalnogo upravleniya}. Diss.
dokt [Lyapunov-Poincare method in optimal control problems. Dr. phys. and math. sci. diss.].
Sverdlovsk, 1986. 280~p. (in Russian)

\bibitem{Calvet}
Calvet~J.-P., Arkun~Y. Design of P and PI stabilizing controllers for quasi-linear systems. \emph{Computers \& Chemical Engineering}, 1990, Vol. 14(4-5), P. 415–426. \hrefdoi{doi:10.1016/0098-1354(90)87017-j}

\bibitem{Ching}
Ching~Sh., Eun~Yo., Gokcek~C., Kabamba~P., Meerkov~S. \emph{Quasilinear Control: Performance Analysis and Design of Feedback Systems with Nonlinear Sensors and Actuators}, 2010 \hrefdoi{10.1017/CBO9780511976476}. 

\bibitem{Dauer}
Dauer~J.\,P. Nonlinear perturbations of quasi-linear control systems,
\emph{Journal of Mathematical Analysis and Applications},
Volume 54, Issue 3,
1976,
Pages 717-725,
\hrefdoi{10.1016/0022-247X(76)90191-8}.

\bibitem{Fillipov}
Filippov~A.\,F.: \emph{Differential Equations with Discontinuous Righthand Sides}. Kluwer Academic Press, Boston, 1988

\bibitem{Fillipov2}
Filippov~A.\,F.: \emph{Vvedenie v teoriju differencial'nyh uravnenij} [Introduction to the theory of differential equations]. Moscow: Comkniga, 2007. 240~p. (in~Russian)

\bibitem{Gabasov}
Gabasov~R.\,F., Kalinin~A.\,I.,  Kirillova~F.\,M., Lavrinovich~L.\,I. On asymptotic optimization methods for quasilinear control systems., \emph{Trudy Instituta Matematiki i Mekhaniki URO
    RAN}, 2019, vol. 25, no. 3, pp. 62–72.

\bibitem{Gui}
Guo~Y., Kabamba~P.\,T., Meerkov~S.\,M.,  Ossareh~H.\,R., Tang~C.\,Y. Quasilinear Control of Wind Farm Power Output, \emph{IEEE Transactions on Control Systems Technology}, 2015 vol. 23, no. 4, P. 1555-1562,  \hrefdoi{10.1109/TCST.2014.2363431}.


\bibitem{GusZyk}
Gusev~M.\,I., Zykov~I.\,V. On Extremal Properties of the Boundary Points of Reachable Sets for Control Systems with Integral Constraints, \emph{Proc. Steklov Inst. Math}, 2018, Vol. 300, P. 114--125. \hrefdoi{10.1134/S0081543818020116}

\bibitem{GusZyk2017}
Gusev M.I., Zykov I.V. On extremal properties of boundary points of reachable sets for a system
with integrally constrained control // IFAC PapersOnline. 2017. Vol. 50, iss. 1. P. 4082–4087. doi:
10.1016/j.ifacol.2017.08.792 .

\bibitem{GusevUMJ}
Gusev~M.\,I., The limits of applicability of the linearization method in calculating small-time reachable sets // \emph{Ural Mathematical Journal}. 2020. Vol. 6, No. 1. P. 71-83
\hrefdoi{https://doi.org/10.15826/umj.2020.1.006}


\bibitem{KalininLavrinovich2018}
Kalinin~A.\,I., Lavrinovich L.I. Asymptotic minimization method of the integral quadratic functional on
the trajectories of a quasilinear dynamical system. \emph{Dokl. NAN Belarusi}, 2018, vol. 62, no. 5, pp. 519–524.
\hrefdoi{10.29235/1561-8323-2018-62-5-519-524} 

\bibitem{Khalil}
Khalil K.H.: Nonlinear Systems, 3rd edn. Pearson, New Jersey (2001)
Халил Х.К. Нелинейные системы. 3 изд. М.-Ижевск. НИЦ <<Регулярная и хаотическая динамика>>, Институт компьютерных исследований, 2009, 832 стр.

\bibitem{Kiselev}
Kiselev~Yu.\,N. An asymptotic solution of the problem of time-optimal control systmes which are close to
linear ones. \emph{Soviet Math. Dokl.}, 1968, vol. 9, no. 5, pp. 1093–1097.

\bibitem{Kras_add}
Красовский Н.Н. Проблемы стабилизации управляемых движений. Дополнение редактора к книге И.Г.Малкина <<Теория устойчивости  движения>>. М.: Наука, 1966, с. 475-514.

\bibitem{Kras_book}
{Krasovskii~N.\,N.} \emph{Teoriya upravleniya dvizheniem} [Theory of Control of Motion]. Moscow: Nauka, 1968. 476~p. (in~Russian)

\bibitem{Kremlev}
Kremlev~A.\,G. On the control of quasilinear system under uncertain initial conditions, \emph{Differential Equations}, 1980, Vol. 16, No. 11, P. 1967-1979. (in Russian)

\bibitem{Polyak1964}
Polyak~B.\,T. Gradient methods for solving equations and inequalities, \emph{USSR Comput. Math. Math. Phys.}, 1964, Vol. 4, No. 6, P. 17–32.

\bibitem{Polyak2001}
Polyak~B.\,T. Convexity of Nonlinear Image of a Small Ball with Applications to Optimization. \emph{Set-Valued Analysis}, 2001, Vol. 9, P. 159–168.\hrefdoi{10.1023/A:1011287523150}

\bibitem{Polyak2004}
Polyak~B.\,T. Convexity of the reachable set of nonlinear systems under L2 bounded controls. \emph{Dynam. Contin. Discrete Impuls. Systems Ser. A Math. Anal.}, 2004, Vol. 11, Suppl. 2-3,  P. 255–267.

\bibitem{Roxin}
Roxin, E. O.: Linearization and approximation of control systems. In: Proceedings of the First World Congress of Nonlinear Analysts, Tampa, Florida, August 19-26, 1992, edited by V. Lakshmikantham, Berlin, Boston: De Gruyter, 2531-2540 (1996). \hrefdoi{10.1515/9783110883237.2531}

\bibitem{Subbotin}
Subbotin~A.\,I. Control of motion of a quasilinear system. \emph{Differ. Uravn.,} 1967, vol. 3, no. 7, pp. 1113–1118
(in Russian).

\bibitem{Patent} 
Zykov~I.\,V., Osipov I.O. A program for constructing the reachable sets of nonlinear systems with integral control constraints by the Monte Carlo method, certificate of state registration of a computer program No. 2020661557, 2020.

\bibitem{Zykov}
Zykov~I.\,V. An Algorithm for Constructing Reachable Sets for Systems with Multiple Integral Constraints  // {\textit Mathematical Analysis With Applications : Intern. Conf. CONCORD-90}, Ekaterinburg, July 2018. Cham : Springer, 2020. P. 51-60. (Springer Proceedings in Mathematics \& Statistics; vol. 318. \hrefdoi{10.1007/978-3-030-42176-2\_6}

\bibitem{Zykov2019}
Zykov~I.\,V., External estimates of reachable sets for control systems with integral constraints, Proceedings of the Voronezh spring mathematical school “Modern methods of the theory of boundary-value problems. Pontryagin readings – XXX”. Voronezh, May 3-9, 2019. Part 1, Itogi Nauki i Tekhniki. Ser. Sovrem. Mat. Pril. Temat. Obz., 190, VINITI, Moscow, 2021, 107–114 \hrefdoi{10.36535/0233-6723-2021-190-107-114}


\bibitem{EKF}
Li, L., Wang, T., Xia, Y., et al:  Trajectory tracking control for wheeled mobile robots based on nonlinear disturbance observer with extended Kalman filter. Journal of the Franklin Institute \textbf{357}(13), 8491--8507 (2020). \hrefdoi{10.1016/j.jfranklin.2020.04.043}.

\bibitem{denBerg}
van den Berg, J.: Iterated LQR smoothing for locally-optimal feedback control of systems with non-linear dynamics and non-quadratic cost. In: 2014 American Control Conference, Portland, OR, USA, 1912--1918 (2014). \hrefdoi{10.1109/ACC.2014.6859404}

\bibitem{Pang}
Pang, Z. -H., Ma, B., Liu, G. -P.  et al: Data-Driven Adaptive Control: An Incremental Triangular Dynamic Linearization Approach. In: IEEE Transactions on Circuits and Systems II: Express Briefs, \textbf{69}(12), 4949--4953 (2022). \hrefdoi{10.1109/TCSII.2022.3181232}.

% \bibitem{Chi}
% Chi, R., Hui, Y., Huang, B., Hou, Z.: Adjacent-Agent Dynamic Linearization-Based Iterative Learning Formation Control. IEEE Transactions on Cybernetics, \textbf{50}(10), 4358-4369 (2019). \doi{10.1109/tcyb.2019.2899654} 

\bibitem{Murillo}
Murillo, M.H., Limache, A.C., Rojas Fredini, P.S. et al.: Generalized Nonlinear Optimal Predictive Control using Iterative State-Space
Trajectories: Applications to Autonomous Flight of UAVs. Int. J. Control Autom. Syst. \textbf{13}(2), 361--370 (2015).
\hrefdoi{10.1007/s12555-013-0416-y}

\bibitem{LTV_MPC}
Papadimitriou, D., Rosolia, U., Borrelli, F.: Control of Unknown Nonlinear Systems with Linear Time-Varying MPC. In: 59th IEEE Conference on Decision and Control (CDC) 2020, Jeju, Korea (South), 2258--2263 (2020). \hrefdoi{10.1109/CDC42340.2020.9304441}

% \bibitem{Zhong}
% Zhong, Z., del Rio-Chanona, E.A., Petsagkourakis, P.: Tube-based distributionally robust model predictive control for nonlinear process systems via linearization. Computers \& Chemical Engineering, \textbf{170}, (2023) \doi{10.1016/j.compchemeng.2022.108112}.

\bibitem{Abgar}
Abgaryan K.A. Matrix Calculus with Applications in the Theory of Dynamical Systems Fizmatlit, Moscow, (1994). (in Russian)

\bibitem{Kur1}
Kurzhanski, A.B., Varaiya, P.: Dynamics and Control of Trajectory Tubes. Theory and Computation. Birkhauser. (2014)\\ https://people.eecs.berkeley.edu/~varaiya/Download/KurzhanskiVaraiya.pdf

\bibitem{Guseinov}
Guseinov, K.G.: Approximation of the attainable sets of the nonlinear control systems with integral constraint on controls. Nonlinear Analysis \textbf{71}(1), 622--645 (2009) \hrefdoi{10.1016/j.na.2008.10.097}

\bibitem{Rousse}
Rousse, P., Alexandre dit Sandretto, J., Chapoutot, A., et al. Guaranteed Simulation of Dynamical Systems with Integral Constraints and Application on Delayed Dynamical Systems. In: Chamberlain, R., Edin Grimheden, M., Taha, W. (eds) Cyber Physical Systems. Model-Based Design. CyPhy WESE 2019. Lecture Notes in Computer Science, vol 11971. Springer, Cham (2020). \hrefdoi{10.1007/978-3-030-41131-2\_5}

\bibitem{GusZykIFAC}
Gusev, M.I., Zykov, I.V.: On extremal properties of boundary points of reachable sets for a system with integrally constrained control. In: Proceedings of 20th World Congress International Federation of Automatic Control, vol. 50, pp. 4082--4087. Elsevier (2017) \hrefdoi{10.1016/j.ifacol.2017.08.792}

\bibitem{Lassak}
Lassak, M.: Banach–Mazur Distance from the Parallelogram to the Affine-Regular Hexagon and Other Affine-Regular Even-Gons. Results Math \textbf{76}(62), (2021). \hrefdoi{10.1007/s00025-021-01368-8}

\bibitem{Ovs}
Goncharova, E., Ovseevich, A.: Small-time reachable sets of linear systems with integral control constraints: birth of the shape of a reachable set. J Optim Theory Appl \textbf{168}, 615--624 (2016).
\hrefdoi{10.1007/s10957-015-0754-4}

\bibitem{GusevMotor}
Gusev, M.I.: Estimates of the minimal eigenvalue of the controllability Gramian for a system containing a small parameter.  In: 
Khachay, M., Kochetov, Y., Pardalos, P. (eds) Mathematical Optimization Theory and Operations Research. MOTOR 
2019. Lecture Notes in Computer Science(), vol 11548. Springer, Cham. \hrefdoi{10.1007/978-3-030-22629-9\_32}

\bibitem{walter}
Walter, W.: Differential and integral inequalities. Springer, Berlin (1970)

\bibitem{Wilkinson}
Уилкинсон, Дж. Х.
Алгебраическая проблема собственных значений.
М.: Наука, 1970, 564 с.

\bibitem{GusevZykov2018}
Gusev, M.I., Zykov, I.V.
An algorithm for computing reachable sets of control systems under isoperimetric constraints // AIP Conference Proceedings. 
2018. 
Vol.2025: Application of Mathematics in Technical and Natural Sciences 
(AMiTaNS'2018): 10th Jubilee Intern. Conf.,
 June 20-25, 2018, Albena, Bulgaria.

 \bibitem{Cockayne}
 Cockayne E.J., Hall G.W.C. Plane motion of a particle subject to curvature constraints // SIAM J. Control. 1975. Vol. 13, no. 1. P. 197--220.  \hrefdoi{10.1137/0313012}

 \bibitem{Patsko}
 Пацко В.С., Пятко С.Г., Федотов А.А. Трехмерное множество достижимости нелинейной управляемой системы // Известия РАН. Теория и системы управления. 2003. № 3. С. 8--16

 \bibitem{Vdovin}
 Вдовин С.А., Тарасьев А.М., Ушаков В.Н., Построение множества достижимости интегратора Брокетта, Прикл. математика и механика, 68:5 (2004), 707-724

 \bibitem{Ledyaev}
 Ю. С. Ледяев, “Критерии выпуклости замкнутых множеств в банаховых пространствах”, Оптимальное управление и дифференциальные уравнения, Сборник статей. К 110-летию со дня рождения академика Льва Семеновича Понтрягина, Труды МИАН, 304, МИАН, М., 2019, 205–220; Proc. Steklov Inst. Math., 304 (2019), 190–204
 \hrefdoi{10.4213/tm3967}
 

 \bibitem{Polyak_book}
 Поляк Б.Т., Хлебников М.В., Рапопорт Л.Б., Математическая теория автоматического управления: учебное пособие. М. Ленанд, 2019, 500 с.

 \bibitem{Stab_lectures}
 Альбрехт Э.Г., Шелементьев Г.С. Лекции по теории стабилизации, Уральский государственный университет им. А.М. Горького, 1972, 273 с.

\bibitem{Barbashin_book}
Барбашин Е.А. Функции Ляпунова. М.:Наука,1970, 240 c.

\bibitem{Rudin}
Rudin, Walter (1973). Functional Analysis. International Series in Pure and Applied Mathematics. Vol. 25 (First ed.). New York, NY: McGraw-Hill Science/Engineering/Math. ISBN 9780070542259.

\bibitem{Polyak2001ru}
Б. Т. Поляк, Локальное программирование, Ж. вычисл. матем. и матем. физ., 2001, том 41, номер 9, 1324–1331

\bibitem{Dmitruk1980}
А. В. Дмитрук, А. А. Милютин, Н. П. Осмоловский, Теорема Люстерника и теория экстремума, УМН, 1980, том 35, выпуск 6, 11–46

\bibitem{Ioffe}
loffe A.D. On the local surjection property // Nonlinear Analysis: Theory, Methods and AppL 1987. V. 11. No 5. P. 565-592.

\bibitem{Guseinov2010}
Kh. G. Guseinov, A. S. Nazlipinar, Attainable sets of the control system with limited resources, Тр. ИММ УрО РАН, 2010, том 16, номер 5, 261–268

\bibitem{Guseinov2022}
N. Huseyin, A. Huseyin, Kh. G. Guseinov, “On the properties of the set of trajectories of nonlinear control systems with integral constraints on the control functions”, Тр. ИММ УрО РАН, 28, no. 3, 2022, 274–284

\bibitem{Bellman}
 Bellman, Richard (1943), "The stability of solutions of linear differential equations", Duke Math. J., 10 (4): 643–647, doi:10.1215/s0012-7094-43-01059-2, MR 0009408, Zbl 0061.18502
 
 \bibitem{Chen}
Chen, Chi-Tsong (1999). Linear System Theory and Design Third Edition. New York, New York: Oxford University Press. ISBN 0-19-511777-8.

\bibitem{vial}
Vial, J.-P. (1982). Strong convexity of sets and functions. Journal of Mathematical Economics, 9(1-2), 187–205. doi:10.1016/0304-4068(82)90026-x

\bibitem{horn2012}
 R. A. Horn and C. R. Johnson, Matrix Analysis, 2nd ed. Cambridge: Cambridge University Press, 2012. p. 662. https://doi.org/10.1017/CBO9781139020411
 
 \bibitem{Thompson}
 A. C. Thompson, Minkowski Geometry. Cambridge: Cambridge University Press, 1996. 
 https://doi.org/10.1017/CBO9781107325845
 
 \bibitem{Patsko2023}
 Валерий С. Пацко, Георгий И. Трубников, Андрей А. Федотов, “Множество достижимости машины Дубинса с интегральным ограничением на управление”, МТИП, 15:2 (2023), 89–104

\bibitem{Polyak1966}
Е. С. Левитин, Б. Т. Поляк, “Методы минимизации при наличии ограничений”, Ж. вычисл. матем. и матем. физ., 6:5 (1966), 787–823; U.S.S.R. Comput. Math. Math. Phys., 6:5 (1966), 1–50 https://doi.org/10.1016/0041-5553(66)90114-5

\bibitem{Polovinkin}
Е. С. Половинкин, “Сильно выпуклый анализ”, Матем. сб., 187:2 (1996), 103–130;  https://doi.org/10.4213/sm111 
E. S. Polovinkin, “Strongly convex analysis”, Sb. Math., 187:2 (1996), 259–286 https://doi.org/10.1070/SM1996v187n02ABEH000111

\bibitem{Gornov2015}
А.\,Ю.~Горнов, Е.\,А.~Финкельштейн, Алгоритм кусочно-линейной аппроксимации границы множества достижимости, {\it Автомат. и телемех.}, 2015, выпуск 3, 22–31

\bibitem{Gornov2017}
Е.\,А.~Финкельштейн, А.\,Ю.~Горнов, Алгоритм квазиравномерного заполнения множества достижимости нелинейной управляемой системы, {\it Известия Иркутского государственного университета. Серия Математика}, 2017, том 19, 217–223.

\bibitem{Lew2020}
T.~Lew, M.~Pavone, Sampling-based Reachability Analysis: A Random Set Theory Approach with Adversarial Sampling, {\it 4th Conference on Robot Learning (CoRL 2020), Cambridge MA, USA.} https://arxiv.org/abs/2008.10180 

\bibitem{Lew2022}
T.~Lew, L.~Janson, R.~Bonalli, M.~Pavone,  A Simple and Efficient Sampling-based Algorithm for General Reachability Analysis. {\it Proceedings of The 4th Annual Learning for Dynamics and Control Conference in Proceedings of Machine Learning Research
	168:1086-1099}. https://proceedings.mlr.press/v168/lew22a.html.
	
\bibitem{Tihonov1948}
Тихонов А. Н. О зависимости решений дифференциальных уравнений от малого параметра / А. Н. Тихонов // Матем. сб. - 1948. - Т. 22, №. 2. - С. 193–204.

\bibitem{Tihonov1952}
Тихонов А. Н. Системы дифференциальных уравнений, содержащих малые параметры при производных / А. Н. Тихонов // Матем. сб. - 1952. - Т. 31(73), №.
3. - C. 575–586

\bibitem{Dmitriev}
Дмитриев М. Г. Сингулярные возмущения в задачах управления / М. Г. Дмитриев, Г. А. Курина // Автоматика и телемеханика.  2006.  №. 1.  С. 3–51.

\bibitem{Vasilieva}
А. Б. Васильева, М. Г. Дмитриев, Сингулярные возмущения в задачах оптимального управления, Итоги науки и техн. Сер. Мат. анал., 1982, том 20, 3–77

\bibitem{Guardabassi}
Guardabassi, G., Locatelli, A. (1975). Periodic Control of Singularly Perturbed Systems. In: Ruberti, A., Mohler, R.R. (eds) Variable Structure Systems with Application to Economics and Biology. Lecture Notes in Economics and Mathematical Systems, vol 111. Springer, Berlin, Heidelberg. https://doi.org/10.1007/978-3-642-47457-6\_6

\bibitem{HornBailey1968}
F.J.M.Horn,J.E.Bailey “An Application of the Theorem of Relaxed Control to the Problem of Increasing Catalyst Selectivity” J. of Optimization Theory and Applications,vol.2,n.6,pp.441–449, 1968.

\bibitem{Rinaldi}
S.Rinaldi “High-Frequency Optimal Periodic Processes” IEEE Trans. on Automatic Control, vol. AC-15, n. 6, pp.671–672, 1970.

\bibitem{Locatelli}
A. Locatelli, S.Rinaldi “Optimal Quasi Stationary Periodic Proces ses”, Automatica, vol. 6, n. 6, pp. 779–785, 1970.

\bibitem{HornBailey1971}
J.E.Bailey, F.J.M.Horn “Comparison Between Two sufficient Conditions for Improvement of an Optimal Steady State Process by Perio dic Operation” J. of Optimization Theory and Applications, vol. 7, n. 5, pp. 378–385, 1971.

\bibitem{Haratishvili}
Г. Л. Харатишвили, Т. А. Тадумадзе, Регулярные возмущения в задачах оптимального управления с переменными запаздываниями и со свободным правым концом, Докл. АН СССР, 1990, том 314, номер 1, 151–155

\bibitem{Nikolski}
М. С. Никольский, Очень регулярная задача оптимального управления, Дифференц. уравнения, 2005, том 41, номер 11, 1526–1532

\bibitem{Vishik}
М. И. Вишик, Л. А. Люстерник, “Регулярное вырождение и пограничный слой для линейных дифференциальных уравнений с малым параметром”, УМН, 12:5(77) (1957), 3–122  mathnet  mathscinet  zmath [M. I. Vishik, L. A. Lyusternik, “Regular degeneration and boundary layer for linear differential equations with small parameter”, Uspekhi Mat. Nauk, 12:5(77) (1957), 3–122]

\bibitem{Chernousko1968}
Черноусько Ф. Л. Некоторые задачи оптимального управления с малым параметром / Ф. Л. Черноусько // Прикладная математика и механика.  1968.  Т. 32, №. 1.  С. 15–26.

\bibitem{Chernousko1977}
Черноусько Ф. Л. Вычислительные и приближенные методы оптимального управ-ления / Ф. Л. Черноусько, В. Б. Колмановский // Итоги науки и техники. Серия Математический анализ. 1977.  Т. 20.  С. 101–166.

\bibitem{Kokotovic}
Kokotovic P. V. Controllability and time–optimal control of systems with slow and fast modes / P. V. Kokotovic, A. H. Haddad // IEEE Trans. Automatic Control. 
1975.  Vol. 20, №. 1.  P. 111–113.

\bibitem{Ilyin1989}
Ильин A. M. Согласование асимптотических разложений решений краевых задач А. М. Ильин.  Москва.: Наука, 1989.  336 c.

\bibitem{Ilyin1998}
Ильин А. М. О структуре решения одной возмущенной задачи быстродействия /А. М. Ильин, А. Р. Данилин // Фундамент. и прикл. матем. 1998. Т. 4, №. 3. C. 905–926.

\bibitem{FilippovaKurzhansky}
А. Б. Куржанский, Т. Ф. Филиппова, “О методе сингулярных возмущений для дифференциальных включений”, Докл. АН СССР, 321:3 (1991), 454–459; Dokl. Math., 44:3 (1992), 705–710

\bibitem{Veliov}
Veliov, V. A generalization of the Tikhonov theorem for singularly perturbed differential inclusions. Journal of Dynamical and Control Systems 3, 291–319 (1997). https://doi.org/10.1007/BF02463254

\bibitem{GONCHAROVAOVSEEVICH}
GONCHAROVA, E. V.,  OVSEEVICH, A. I. (2010). REFINED ASYMPTOTICS FOR SINGULARLY PERTURBED REACHABLE SETS. World Scientific Series on Nonlinear Science Series B, 259–264. https://doi.org/10.1142/9789814313155\_0039

%\bibitem{ref_proc1}
%Author, A.-B.: Contribution title. In: 9th International Proceedings
%on Proceedings, pp. 1--2. Publisher, Location (2010)

%\bibitem{ref_article1}
%Author, F.: Article title. Journal \textbf{2}(5), 99--110 (2016)

%\bibitem{ref_lncs1}
%Author, F., Author, S.: Title of a proceedings paper. In: Editor,
%F., Editor, S. (eds.) CONFERENCE 2016, LNCS, vol. 9999, pp. 1--13.
%Springer, Heidelberg (2016). \doi{10.10007/1234567890}


%\bibitem{Zyk}
%Zykov I.V. On external estimates of reachable sets of control systems with integral constraints, {\it Izvestiya Instituta Matematiki i Informatiki Udmurtskogo Gosudarstvennogo Universiteta}, 2019. vol. 53. pp. 61--72. (in Russian). 
%doi: 10.20537/2226-3594-2019-53-06

%\bibitem{GusZyk}
%Gusev, M.I., Zykov, I.V.: On Extremal Properties of the Boundary Points of Reachable Sets for Control Systems with Integral Constraints. In: Sergeev, A.G. (eds.), Proc. Steklov Inst. Math. 300, pp. 114--125. Springer (2018). \doi{10.1134/S0081543818020116}

%\bibitem{Atans}
%Atans, M., Falb, P.: Optimal control. 
%Dover Publications, New York, McGraw-Hill (1966)

%\bibitem{Perv}
%Pervozvanskii A.A., {\it Kurs teorii avtomaticheskogo upravleniya} (Course in automatic control theory). Moscow: Nauka, 1986, 615 p.

%\bibitem{Gus1}
%Gusev, M.I.: On Convexity of Reachable Sets of a Nonlinear System under Integral Constraints. IFAC-PapersOnLine \textbf{51}(32), 207--212 (2018). \doi{10.1016/j.ifacol.2018.11.382}


%\bibitem{Krener}
%Krener, A., Sch\"{a}ttler, H.: The structure of small-time reachable sets in low dimensions. SIAM Journal on Control and Optimization %\textbf{27}(1) 120--147 (1989). \doi{10.1137/0327008}

%\bibitem{Schattler}
%Sch\"{a}ttler, H.: Small-time reachable sets and time-optimal feedback control. In: Mordukhovich, B.S.,
%Sussmann, H.J. (eds.) Nonsmooth Analysis and Geometric Methods in Deterministic Optimal Control.
%The IMA Volumes in Mathematics and Its Applications. Springer, New York. 1996. Vol. 78. p.~203–-225. \doi{10.1007/978-1-4613-8489-2\_9}

% \bibitem{Watson}
% Watson, J., Abdulsamad, H., Peters, J.: Stochastic Optimal Control as Approximate Input Inference. In: Kaelbling, L. P., Kragic, D., Sugiura, K. (eds.) Proceedings of the Conference on Robot Learning. Proceedings of Machine Learning Research. \textbf{100}, 697--716 PMLR (2020) 
% \doi{10.48550/arXiv.1910.03003}

% \bibitem{Cao}
% Cao, R., Jiang, N., Lu, M.: Sensorless Control of Linear Flux-Switching Permanent Magnet Motor Based on Extended Kalman Filter. In: IEEE Transactions on Industrial Electronics, \textbf{67}(7), 5971--5979, (2020) \doi{10.1109/TIE.2019.2950865}.



%\bibitem{ref_book1}
%Author, F., Author, S., Author, T.: Book title. 2nd edn. Publisher,
%Location (1999)



%\bibitem{ref_url1}
%LNCS Homepage, \url{http://www.springer.com/lncs}. Last accessed 4
%Oct 2017
\end{thebibliography}
\end{document}