\documentclass[../main.tex]{subfiles}
\begin{document}
\clearpage
\section{Приближенное решение задачи синтеза управления на малом интервале времени}
\subsection{Асимптотика множеств достижимости нелинейных систем с интегральными ограничениями, в том числе по части координат} 
Данная раздел содержит материалы, опубликованные в статье \cite{Osipov}.
Здесь исследуется выпуклость множеств достижимости по части координат нелинейных систем с интегральными ограничениями на управление на малых промежутках времени.
Доказаны достаточные условия выпуклости, имеющие вид ограничений на асимптотику собственных чисел грамиана управляемости линеаризованной системы по части координат.
В качестве примеров, в статье описаны две нелинейные системы третьего порядка, в одной из которых линеаризованная вдоль траектории, порожденной нулевым управлением, система неуправляема, а в другом управляема.
Исследованы достаточные условия выпуклости проекций множеств достижимости. 
Проведено численное моделирование, продемонстрировавшее невыпуклость некоторых проекций даже для малых длин временного промежутка.
\subsubsection{Расстояние Банаха-Мазура и асимптотическая эквивалентность множеств}\label{sec21:AsymptoticEquality}
Напомним, что расстоянием Банаха-Мазура между выпуклыми компактными множествами $ X,Y \subset \mathbb R^n $ называют величину $ \rho (X, Y)  $, определенную равенством 
\begin{gather*}
    \rho (X, Y): = \log (r(X,Y) \cdot r(Y, X)),
\end{gather*}
где $r(X, Y) = \inf \{t \geq 1: tX \supset Y \}$.

Пусть $ X = X(\varepsilon) $,  $ Y = Y(\varepsilon) $ -- выпуклые компактные множества, такие, что $ 0 \in \operatorname{int}\,X(\varepsilon) $, $ 0 \in \operatorname{int}\,Y(\varepsilon) $ при $0 \leqslant \varepsilon \leqslant \overline{\varepsilon} $.
Тогда, следуя \cite{Ovs}, назовем множества $  X(\varepsilon)  $ и $  Y(\varepsilon) $ {\textit асимптотически эквивалентными}, если $  \rho (X(\varepsilon), Y(\varepsilon)) \rightarrow 0 $ при $\varepsilon \rightarrow 0 $.

Приведем далее достаточное условие асимптотической эквивалентности, выраженное через хаусдорфово расстояние $ h $ между ними.
\begin{theorem}\label{suff}\cite{GusevUMJ}
     Выполнения следующих условий достаточно для того, чтобы множества $ X(\varepsilon) $ и $ Y(\varepsilon) $ были асимптотически эквивалентны:
    \begin{gather*}
        \lim\limits_{\varepsilon \rightarrow 0}h(X(\varepsilon),Y(\varepsilon)) = 0, \qquad    \lim\limits_{\varepsilon \rightarrow 0}\frac{h(X(\varepsilon),Y(\varepsilon))}{\delta_{min}(Y(\varepsilon))} = 0,
    \end{gather*}
    где $ \delta_{min}(Y(\varepsilon)) = \inf\limits_{\left\|y \right\| =1 } \delta(y|Y(\varepsilon))$, а $ \delta(y|Y(\varepsilon)) $ -- опорная функция множества $ Y(\varepsilon) $.
\end{theorem}
\subsubsection{Достаточное условие асимптотической эквивалентности множеств достижимости нелинейной и линеаризованной систем}
    Рассмотрим нелинейную систему, аффинную по управлению
\begin{gather}\label{nonlinearY}
    \begin{gathered}
        \dot{x}(t)=f_1(t,x(t))+f_2(t,x(t))u(t), \qquad t_0 \leq t \leq t_0 + \bar{\varepsilon}, \qquad x(t_0) = x_0, \\
        y(t) = C x(t).
    \end{gathered}
\end{gather}
Здесь $ x \in \mathbb{R}^n $ -- вектор состояния, $ u \in \mathbb{R}^r $ -- управление,  $ y\in\mathbb{R}^m (m \leqslant n) $ -- выход системы,
$ C\in \mathbb{R}^{m \times n} $  -- матрица полного ранга, $m\leq n$, $ \bar{\varepsilon} $ --- некоторое фиксированное положительное число.
    Как и в предыдущих разделах функции $ f_1: \mathbb{R}^{n+1} \rightarrow \mathbb{R}^{n} $, $ f_2: \mathbb{R}^{n+1} \rightarrow \mathbb{R}^{n \times r} $ предполагаются непрерывными и непрерывно-дифференцируемыми по $ x $.
По-прежнему предполагается, что функции $ f_1 $, $ f_2 $ удовлетворяют условиям
\begin{gather*}
    \begin{gathered}
        \left\| f_1(t,x) \right\| \leqslant    l_1(t)(1 + \left\| x \right\| ), \\
        \left\| f_2(t,x) \right\| \leqslant    l_2(t),  \qquad t_0 \leqslant t \leqslant t_0 + \bar{\varepsilon}, \qquad   x \in \mathbb{R}^n,
    \end{gathered}
\end{gather*}
где $ l_1(\cdot) \in \mathbb{L}_1[t_0,t_0+\bar{\varepsilon}] $, $ l_2(\cdot) \in \mathbb{L}_2[t_0,t_0+\bar{\varepsilon}] $.
Предположение \ref{Pred} также считается выполненным.

 Управление $u(t)$ будем выбирать из
пространства $\mathbb{L}_2[t_0,t_0+\bar{\varepsilon}]$ вектор-функций и ограничим его шаром радиуса $ \mu > 0 $
\begin{gather}\label{constrY}
    \lVert u(\cdot)\rVert^2_{\mathbb{L}_2} = \left(u(\cdot),u(\cdot) \right) \leqslant \mu^2.
\end{gather}

В условиях описанных предположений, каждому $ u(\cdot) \in \mathbb{L}_2 $ соответствует единственное абсолютно непрерывное решение $ x(t)=x(t,u(\cdot)) $ системы \eqref{nonlinearY}, определённое на интервале $ [t_0,t_0+\bar{\varepsilon}] $.

Все траектории $ x(t) $ системы \eqref{nonlinearY}, отвечающие удовлетворяющим \eqref{constrY} управлениям,  лежат внутри некоторого компактного множества $ D \subset \mathbb{R}^n $.
Пусть $ 0 <  \varepsilon \leqslant \bar{\varepsilon} $. 

Множеством достижимости системы \eqref{nonlinearY} по состоянию, как и прежде, будем называть множество \begin{gather*}
    G(\varepsilon)=\{x\in \mathbb{R}^n:\exists u(\cdot)\in B_{\mathbb{L}_2}(0,\mu),\; x=x(t_0+\varepsilon,x^0,u(\cdot))\}.
\end{gather*}


\begin{definition}
    {\textit Множеством достижимости $G_y(\varepsilon)$ системы \eqref{nonlinearY} по выходу} $ y = C x $ будем называть множество всех выходов $ y(t_0+\varepsilon) $,
    соответствующих концам траекторий $ x(t_0+\varepsilon) $, порождённых управлениями $ u(t) \in B_{\mathbb{L}_2}(0,\mu)$
    \begin{gather*}
        G_y(\varepsilon)=\{y\in \mathbb{R}^m:\exists u(\cdot)\in B_{\mathbb{L}_2}(0,\mu),\; y=Cx(t_0+\varepsilon,x^0,u(\cdot))\}.
    \end{gather*}
\end{definition}

В определениях множеств, приведённых выше, можно считать, что $ \mathbb{L}_2 =\mathbb{L}_2[t_0,t_0+\varepsilon] $, либо  $ \mathbb{L}_2=\mathbb{L}_2[t_0,t_0+\bar{\varepsilon}] $.
Нетрудно понять, что для любого из этих пространств мы получаем одно и то же множество достижимости. Будем далее считать, что $ \mathbb{L}_2 =\mathbb{L}_2[t_0,t_0+\varepsilon] $.
    
Заметим, что  $ G_y(\varepsilon) = C G(\varepsilon) $.    
    
Если матрица $ C \in \mathbb{R}^{m \times n} $ такова, что в каждой её строке только один элемент равен 1, а остальные равны 0, а в каждом столбце содержится не более одного ненулевого элемента, то $ y=Cx $ состоит из $ m$ координат вектора $ x $, а  множество достижимости $G_y(\varepsilon)$ представляет собой проекцию множества $ G(\varepsilon) $ на $m$--мерную координатную плоскость.


\begin{definition}
    Симметричная матрица, определённая равенством
    \begin{gather*}
        W_y(\varepsilon) = C\int_{t_0}^{t_0+\varepsilon}X(t_0+\varepsilon,t)B(t)B^{\top}(t)X^{\top}(t_0+\varepsilon,t) \, dtC^\top=CW(\varepsilon)C^\top,
    \end{gather*}
    называется грамианом управляемости системы \eqref{linear} на интервале времени $  t_0 \leq t \leq t_0 + \varepsilon $ по выходу $y$.
\end{definition}
    
    Система \eqref{linear} вполне управляема на  $ [t_0, t_0 + \varepsilon] $ по выходу $y=Cx$ тогда и только тогда,
    когда  грамиан управляемости по выходу $ \widetilde{W}_y(\varepsilon) = C \widetilde{W}(\varepsilon) C^{\top}  $ -- положительно определенная матрица, то есть его минимальное собственное число $ \nu^y(\varepsilon) $ положительно.
    
    \todo[inline]{В С Т А В И Т Ь Ф Р А Г М Е Н Т И Р А С П И С А Т Ь\\ данном случае, $S'(0)B_{\mathbb{L}_2}(0,\mu\sqrt{\varepsilon}) = \widetilde{W}^{1/2}(\varepsilon)B_{\mathbb{R}^n}(0,\mu\sqrt{\varepsilon}) $ ~--- множество достижимости в момент $t = 1$ системы \eqref{sec1:eps_linearized}, линеаризованной вдоль траектории, отвечающей нулевому управлению. Здесь $W^{1/2}(\varepsilon)$ ~--- арифметический квадратных корень из матрицы $W(\varepsilon)$, $ B_{\mathbb{R}^n}(0,\mu) $ ~--- евклидов шар радиуса $ \mu $ в $ \mathbb{R}^n $.}
    
    Основной результат этого раздела отражен в следующей теореме.
    \begin{theorem}\label{th2}
        При достаточно малых $ \varepsilon $ множество достижимости $ G_y(\varepsilon) $ системы \eqref{nonlinearY} по выходу $ y = C x $ выпукло и асимптотически эквивалентно множеству $W_y^{1/2}(\varepsilon)B_{\mathbb{R}^n}(0,\mu) + Cx(t_0+\varepsilon,0)$, если найдутся такие $ K>0 $, $ \alpha > 0 $, $ 0< \varepsilon_0<\overline{\varepsilon}  $, что для всех $ \varepsilon \leqslant \varepsilon_0 $
        \begin{gather}\label{cond1}
            \nu^y(\varepsilon) \geqslant \left\{ {\begin{array}{*{20}{l}}
                    {K\varepsilon ^{3 - \alpha}, \mbox{\ если \ } f_2(t,x) \mbox{\ не зависит от \ } x}, \\
                    {K\varepsilon ^{1 - \alpha}}, \mbox{\ в противном случае}.
            \end{array}} \right.
        \end{gather}
    \end{theorem}
    
    Здесь $W_y^{1/2}(\varepsilon)$ --- арифметический квадратных корень из матрицы $W_y(\varepsilon)$, $ B_{\mathbb{R}^n}(0,\mu) $ --- евклидов шар радиуса $ \mu $ в $ \mathbb{R}^n $
    
    \doc. 
    При фиксированном $ \varepsilon $ введем отображение $ F_{\varepsilon}: \mathbb{L}_2[0,1]  \rightarrow \mathbb{R}^n$, зависящее от параметра $ \varepsilon $ равенством $F_{\varepsilon}(\upsilon(\cdot)) = z(1,\upsilon(\cdot)) $, где  $ z(\tau,\upsilon(\cdot)) $ -- траектория системы \eqref{epsnonlinear}, соответствующая управлению $ \upsilon(\cdot) $. Тогда композиция отображений $ F_{\varepsilon}$ и $ C $, $ H_{\varepsilon}: \mathbb{L}_2[0,1]  \rightarrow \mathbb{R}^k $, есть $ H_{\varepsilon}(\upsilon(\cdot)) = CF_{\varepsilon}(\upsilon(\cdot)) = Cz(1,\upsilon(\cdot)) $. В силу \eqref{epscond}, мы имеем $ H_{\varepsilon}(B_{\mathbb{L}_2[0,1]}(0,\varrho(\varepsilon))) = \widetilde{G}_y(\varepsilon) = G_y(\varepsilon)  $, где $ \varrho(\varepsilon) = \mu \sqrt{\varepsilon} $.
    
    Отображение $ H_{\varepsilon} $ непрерывно дифференцируемо по Фреше для $ \forall \upsilon(\cdot) \in  B_{\mathbb{L}_2[0,1]}(0,\varrho(\varepsilon)) $, как композиция непрерывно дифференцируемых отображений и его производная Фреше $ H_{\varepsilon}': \mathbb{L}_2[0,1]  \rightarrow  \mathbb{R}^k $ определена равенством:
    \begin{gather*}
        H_{\varepsilon}'( \upsilon(\cdot))\delta \upsilon = C \delta z(1) = C F_{\varepsilon}'(\upsilon(\cdot))\delta  \upsilon,
    \end{gather*}
    где $ \delta z(1) $ --- решение линеаризованной системы \eqref{epslinear} c нулевыми начальными условиями и управлением $  \delta  \upsilon(\tau) $, $ F_{\varepsilon}'( \upsilon(\cdot)) $ --- производная Фреше отображения $ F _{\varepsilon}$.
    Для $ F_{\varepsilon}' $ имеет место равенство (см., например \cite{GusevMotor}).
    \begin{gather}\label{derivF}
        F_{\varepsilon}'( \upsilon(\cdot)) = X_{\varepsilon}(1,\tau, \upsilon(\cdot))B_{\varepsilon} (\tau, \upsilon(\cdot)), \quad \tau\in[0,1],
    \end{gather}
    где $ X_{\varepsilon}(1,\tau,\upsilon(\cdot)) $ --- фундаментальная матрицы системы \eqref{epslinear}, матрицы которой $ A_{\varepsilon}(\tau) $, $ B_{\varepsilon}(\tau)  $ зависят от $ \upsilon(\cdot) $.
    Из \eqref{derivF} следует, что $ H_{\varepsilon}'(u(\cdot)) $ есть ни что иное, как
    \begin{gather*}
        H_{\varepsilon}'(\upsilon(\cdot)) = C X_{\varepsilon}(1,\tau, \upsilon(\cdot))B_{\varepsilon} (\tau, \upsilon(\cdot)), \quad \tau\in[0,1],
    \end{gather*}
    Можно доказать, что $ F_{\varepsilon}'(\upsilon(\cdot)) $ --- липшицева
    \begin{gather}\label{lipdF}
        \left\| F_{\varepsilon}'(\upsilon_1(\cdot)) - F_{\varepsilon}'(\upsilon_2(\cdot)) \right\| \leqslant L(\varepsilon) \left\| \upsilon_1(\cdot) - \upsilon_2(\cdot)\right\|
    \end{gather}
    при $ \upsilon \in B_{\mathbb{L}_2[0,1]}(0,\rho(\varepsilon))  $, но тогда и $ H_{\varepsilon}'(\upsilon(\cdot)) $ --- липшицева. Доказательство проводится по схеме работы\cite{GusevUMJ}.  При этом $ L(\varepsilon) = L_0 + L_1 \varepsilon $, где $ L_0 = 0$, если $ f_2(t,x)  $ не зависит от $ x $.
    Тогда можно оценить максимальный радиус $ \rho(\varepsilon) $ гильбертова шара $ B_{\mathbb{L}_2[0,1]}(0,\rho(\varepsilon)) $, образ которого $ F_{\varepsilon}(B_{\mathbb{L}_2[0,1]}(0,\rho(\varepsilon))) $ будет выпуклым (см. \cite{Polyak2004}):
    \begin{gather}\label{Polest}
        \rho(\varepsilon) \leqslant \frac{\sqrt{\nu(\varepsilon)}}{2L(\varepsilon)}.
    \end{gather}
    Теперь проведем аналогичные рассуждения для $ H_{\varepsilon} $. Аналог неравенства \eqref{lipdF}  имеет вид
    \begin{gather*}
        \left\| H_{\varepsilon}'(\upsilon_1(\cdot)) - H'_{\varepsilon}(\upsilon_2(\cdot)) \right\| \leqslant L(\varepsilon) \left\| \upsilon_1(\cdot) - \upsilon_2(\cdot)\right\|,
    \end{gather*}
    где для константы Липшица $ L(\varepsilon) = L_0 + L_1 \varepsilon$ отображения $ H_{\varepsilon} $ сохраним то же обозначение, что и в \eqref{lipdF}.
    А неравенство \eqref{Polest} перепишем так, чтобы найти условие, при котором множество достижимости системы \eqref{epsnonlinear} по выходу $ \widetilde{G}_y(1) = H_{\varepsilon} (B_{\mathbb{L}_2[0,1]}(0,\varrho(\varepsilon)))$ будет выпуклым:
    \begin{gather}\label{est2}
        4\varrho^2(\varepsilon)L^2(\varepsilon) = 4\mu^2\varepsilon L^2(\varepsilon) \leqslant \nu^y(\varepsilon),
    \end{gather}
    где $ \varrho(\varepsilon)  = \mu\sqrt{\varepsilon} $.
    
    В случае, если функция $ f_2(t,x) $ не зависит от $ x $, то $ L(\varepsilon) = L_1 \varepsilon  $, а \eqref{est2} принимает вид $ \nu^y(\varepsilon) \geqslant 4\mu^2L_1^2 \varepsilon^3 $.  Это неравенство будет выполнено, если $ K\varepsilon^{3 - \alpha} \geqslant 4\mu^2L_1^2 \varepsilon^3 $, что равносильно соотношению $ \varepsilon \geqslant \varepsilon_1 := \left(\frac{K}{4\mu^2L_1^2}\right)^{1/\alpha} $.  Таким образом, множество достижимости $ G_y(\varepsilon)$ выпукло при $ 0 \leqslant \varepsilon \leqslant \min\{\varepsilon_0,\varepsilon_1\}  $.  
    
    В другом случае, если $ f_2(t,x) $ зависит от $ x $, $ L(\varepsilon) =L_0+L_1\varepsilon $, а \eqref{est2} принимает вид $ \nu^y(\varepsilon) \geqslant 4\mu^2 \varepsilon (L_0 + L_1 \varepsilon)^2 $.  Учитывая, что $ \nu^y(\varepsilon)  \geqslant K \varepsilon^{1-\alpha} $, достаточно доказать, что $ K \varepsilon^{1-\alpha}  \geqslant 4\mu^2 \varepsilon (L_0 + L_1 \varepsilon_0)^2 $. Последнее неравенство выполняется при $ \varepsilon \leqslant \varepsilon_2 := \left(\frac{K}{4\mu^2(L_0 + L_1\varepsilon_0)^2} \right)^{1/\alpha} $. Следовательно, $ G_y(\varepsilon) $ выпукло, если $ 0 \leqslant \varepsilon \leqslant \min\{\varepsilon_0, \varepsilon_2\} $.
    
    Множество $ W_y^{1/2}(\varepsilon)B_{\mathbb{R}^n}(0,\mu) + Cx(t_0+\varepsilon,0)  = H'_{\varepsilon}(0)B_{\mathbb{L}_2[0,1]}(0,\varrho(\varepsilon))+ H_{\varepsilon}(0)$ есть не что иное, как множество достижимости по выходу линеаризованной системы \eqref{epslinear}.
    
    Доказательство асимптотической эквивалентности проекций множеств достижимости нелинейной и линеаризованной систем проводится по той же схеме, что и доказательство следствий теоремы 2 в \cite{GusevUMJ}. Оценим сверху хаусдорфово расстояние между образами гильбертова шара $ H_{\varepsilon}\left( B_{\mathbb{L}_2[0,1]}(0,\varrho(\varepsilon))\right)  $ и $ H'_{\varepsilon}(0)B_{\mathbb{L}_2[0,1]}(0,\varrho(\varepsilon))+ H_{\varepsilon}(0) $:
    \begin{gather*}
        h\left( G_y(\varepsilon), H'_{\varepsilon}(0)B_{\mathbb{L}_2[0,1]}(0,\varrho(\varepsilon))+ H_{\varepsilon}(0)\right)  \leqslant L(\varepsilon) \varrho^2(\varepsilon).
    \end{gather*}
    А так как $ \lim\limits_{\varepsilon \rightarrow 0}  L(\varepsilon) \varrho^2(\varepsilon) = 0 $, то и $ \lim\limits_{\varepsilon \rightarrow 0} h\left( G_y(\varepsilon), H'_{\varepsilon}(0)B_{\mathbb{L}_2[0,1]}(0,\varrho(\varepsilon))+ H_{\varepsilon}(0)\right)  = 0 $.
    
    
    Множество достижимости линеаризованной системы -- конечномерный эллипсоид и его наименьшая полуось $ \delta_{min}\left(H'_{\varepsilon}(0)B_{\mathbb{L}_2[0,1]}(0,\varrho(\varepsilon)) \right)=\varrho(\varepsilon)\sqrt{\nu^y(\varepsilon)} $. 
    Следовательно,
    \begin{gather*}
        \frac{h\left(G_y(\varepsilon), H'_{\varepsilon}(0)B_{\mathbb{L}_2[0,1]}(0,\varrho(\varepsilon))+ H_{\varepsilon}(0)\right) }{\delta_{min}\left( H'_{\varepsilon}(0)B_{\mathbb{L}_2[0,1]}(0,\varrho(\varepsilon))\right) } \leqslant \frac{L(\varepsilon) \varrho(\varepsilon)}{\sqrt{\nu^y(\varepsilon)}}.
    \end{gather*}
    
    При выполнении условий \eqref{cond1} в первом случае ($ f_2(t,x) $ не зависисит от $ x $) имеем
    \begin{gather*}
        \frac{L(\varepsilon) \varrho(\varepsilon)}{\sqrt{\nu^y(\varepsilon)}} 
        \leqslant
        \frac{L_1\varepsilon\mu\sqrt{\varepsilon}}{K^{\frac{1}{2}}\varepsilon^{\frac{3}{2}-\frac{\alpha}{2}}} 
        =
        L_1\mu K^{-\frac{1}{2}}\varepsilon^{\frac{\alpha}{2}} \rightarrow 0 \mbox{\ при\ } \varepsilon \rightarrow 0.
    \end{gather*}
    
    Во втором случае,
    \begin{gather*}
        \frac{L(\varepsilon) \varrho(\varepsilon)}{\sqrt{\nu^y(\varepsilon)}} 
        \leqslant
        \frac{(L_0+L_1\varepsilon)\mu\sqrt{\varepsilon}}{{K^{\frac{1}{2}}}\varepsilon^{\frac{1}{2}-\frac{\alpha}{2}}}
        =
        (L_0+L_1\varepsilon)\mu K^{-\frac{1}{2}}\varepsilon^{\alpha/2} \rightarrow 0 \mbox{\ при\ } \varepsilon \rightarrow 0.
    \end{gather*}
    
    Таким образом условия теоремы \ref{suff} выполнены, а значит множества $ G_y(\varepsilon) $ и \\ $ H'_{\varepsilon}(0)B_{\mathbb{L}_2[0,1]}(0,\varrho(\varepsilon))+ H_{\varepsilon}(0) $ асимптотически эквиваленты.
    \hfill $\square$
    \begin{utv}
        Пусть $ W_1 $, $ W $ -- симметричные матрицы, причем $ W_1 = C W C^{\top} $, где $ C $ -- матрица  полного ранга размерности $ m \times n $, $ m \leqslant n $, а $ \nu(W) $, $ \nu(W_1) $ и $ \nu(CC^{\top}) $ -- наименьшие собственные числа соответствующих матриц.
        
        Тогда
        \begin{gather*}
            \nu(W_1) \geqslant \nu(CC^{\top})  \nu(W).
        \end{gather*}
    \end{utv}
    \doc. 
    Действительно,
    \begin{gather*}
        \forall x \in \mathbb{R}^m, \: x^{\top} W_1 x = x^{\top} C W C^{\top} x \geqslant \nu(W)\left\| C^{\top}x \right\| ^2,\\
        \nu(W)\left\| C^{\top}x \right\| ^2 = \nu(W) x^{\top} C C^{\top} x
    \end{gather*}
    Следовательно,
    \begin{gather*}
        \nu(W_1) = \min \limits_{\left\| x\right\| =1}x^{\top}W_1x \geqslant \nu(W)\min \limits_{\left\| x\right\| =1}x^{\top}CC^{\top}x = \nu(CC^{\top})  \nu(W) 
    \end{gather*} \begin{flushright}
        \hfill $ \square $
    \end{flushright}
    
    Применяя утверждение к грамиану управляемости линеаризованной системы $ \widetilde{W}(\varepsilon) $, грамиану управляемости по выходу $ \widetilde{W}_y(\varepsilon) $, и их наименьшим собственным числам $ \nu^y(\varepsilon) $ и $ \nu(\varepsilon) $, получим
    \begin{gather*}
        \nu^y(\varepsilon) \geqslant \nu(CC^{\top}) \nu(\varepsilon),
    \end{gather*}
    где $ \nu(CC^{\top}) $ не зависит от $ \varepsilon $. Значит асимптотика $ \nu^y(\varepsilon) $ при $\varepsilon \rightarrow 0$ не может быть хуже асимптотики $ \nu(\varepsilon) $, если $ C $ -- матрица полного ранга. Это соответствует очевидному факту, если множество $ G(\varepsilon) $ -- выпуклое, то и для всех возможных матриц полного ранга $ C $, соответствующие множества $ {G}_y(\varepsilon) $ -- выпуклы. Однако невыпуклое множество $ G(\varepsilon) $  может иметь выпуклые проекции, что продемонстрировано в одном из примеров.
\subsubsection{О выпуклости двумерных проекций множеств достижимости уницикла на малых промежутках времени}    
    Исследуем проекции множеств достижимости на примере системы третьего порядка
    \begin{gather}\label{unicycle0}
        \dot{x_1} = v(t)\cos(x_3), \qquad
        \dot{x_2} = v(t)\sin(x_3), \qquad
        \dot{x_3} = u(t), \qquad 0 \leq t \leq \varepsilon
    \end{gather}
    при ограничениях на управление 
    \begin{gather*}
        v(t) = 1, \qquad \int_0^1 u^2(t) \, dt \leqslant 1
    \end{gather*}
    и нулевых начальных условиях $ x_1(0) = x_2(0) = x_3(0) = 0 $.
    
    Система \eqref{unicycle0} известна как уницикл (при $ v(t) = 1$  --- машина Дубинса). При геометрическом
    ограничении на
    управление ($|u(t)|\leq 1$) проекции множества достижимости  машины Дубинса  на двумерное пространство координат $(x_1,\,x_2)$ были исследованы в
    \cite{Cockayne}. Общая трехмерная картина множества достижимости  получена в
    работе  \cite{Patsko}. Множество достижимости интегратора Брокетта, к которому нелинейными преобразованиями могут быть приведены уравнения уницикла \eqref{unicycle0}, исследовано в\cite{Vdovin}.
    
    Запишем решение $ x(t,u(t)) $, порожденное нулевым управлением $ u(t) \equiv 0 $:
    \begin{gather*}
        \begin{gathered}
            \dot{x_3} = 0 \longrightarrow x_3(t) = x_3(0) + 0 = 0, \\
            \dot{x_2} = \sin(x_3(t)) = \sin(0) \longrightarrow x_2(t) = x_2(0) + \int_0^t 0 \, d\tau = 0,\\
            \dot{x_1} = \cos(x_3(t)) = \cos(0) \longrightarrow x_3(t) = x_3(0) + \int_0^t    1 \, d\tau = t.
        \end{gathered}
    \end{gather*}
    Матрицы линеаризованной системы не зависят от $ t $ и имеют вид:
    \begin{gather}\label{linear0}
        A = \begin{pmatrix}
            0 & 0 & 0 \\ 
            0 & 0 & 1 \\ 
            0 & 0 & 0
        \end{pmatrix}, \qquad  B = \begin{pmatrix}
            0 \\ 
            0 \\ 
            1
        \end{pmatrix} 
    \end{gather}
    Пара $ (A,B) $, очевидно, не является вполне управляемой.
    Выпишем фундаментальную матрицу системы \eqref{linear0}, а затем получим грамиан управляемости линеаризованной системы
    \begin{gather*}
        \begin{gathered}
            \dot{X}(t,t_1) = A(t) X(t,t_1), \qquad X(t_1,t_1) = I \\
            X(t,\tau) = \begin{pmatrix}
                1 & 0 & 0 \\ 
                0 & 1 & \varepsilon \\ 
                0 & 0 & 1
            \end{pmatrix}  \\
            W(\varepsilon) = \int_0^{\varepsilon}X(\varepsilon,t) B B^{\top} X^{\top}(\varepsilon,t)dt 
            =\begin{pmatrix}
                0 & 0 & 0 \\
                0 & \frac{\varepsilon^3}{3} & \frac{\varepsilon^2}{2} \\
                0 &  \frac{\varepsilon^2}{2} & \varepsilon
            \end{pmatrix} 
        \end{gathered}
    \end{gather*}
    Сделаем замену времени и перепишем грамиан управляемости в виде
    \begin{gather*}
        \widetilde{W}(\varepsilon) = \frac{1}{\varepsilon}W(\varepsilon)     =\begin{pmatrix}
            0 & 0 & 0 \\
            0 & \frac{\varepsilon^2}{3} & \frac{\varepsilon}{2} \\
            0 &  \frac{\varepsilon}{2} & 1
        \end{pmatrix} 
    \end{gather*} 
    Теперь последовательно рассмотрим проекции системы \eqref{unicycle0} на координатные плоскости $ (x_1, x_2) $, $ (x_1, x_3) $, $ (x_2, x_3) $. \\
    
    \begin{enumerate}
        \item Рассмотрим плоскость $ (x_1, x_2) $. Матрица $ C $ для этой проекции имеет вид
        \begin{gather*}
            C = \begin{pmatrix}
                1 & 0 & 0 \\
                0 & 1 & 0
            \end{pmatrix}.
        \end{gather*}
        Грамиан управляемости в нормированном времени
        \begin{gather*}
            \widetilde{W}_{x_1,x_2}(\varepsilon) =  C \widetilde{W} (\varepsilon) C^{\top}  =\begin{pmatrix}
                0 & 0 \\
                0 & \frac{\varepsilon^2}{3}. \\
            \end{pmatrix}.
        \end{gather*}
        Нетрудно заметить, что $ \widetilde{W}_{x_1,x_2}(\varepsilon) $ -- вырожденная. Следовательно, система \eqref{unicycle0} не управляема по выходу $ (x_1, x_2) $, а значит, достаточное условие выпуклости множества достижимости по выходу $ G_{x_1,x_2}(\varepsilon) $ не выполняется. Множество $ G_{x_1,x_2}(\varepsilon) $ полученное в численном эксперименте показано на рисунке~\ref{fig:RS}-\subref{fig:u=0_x1-x2}. Отметим, что в работе \cite{GusevOsipovTrudy} при помощи принципа максимума доказано, что $ G(\varepsilon) $ -- невыпукло. Из данного доказательства также следует, что $ G_{x_1,x_2}(\varepsilon) $ тоже не является выпуклой.
        
        В этой работе для построения множеств достижимости мы используем алгоритм, основанный на методе Монте-Карло. %\todo{тут  будет ссылка либо на статью Игоря, либо на раздел, когда он появится}\footnote{Zykof2}.
        Удовлетворяющее интегральным ограничениям управление $ u(t) $ представляется в виде линейной комбинации ортогональных полиномов. Коэффициенты этого разложения -- равномерно распределенные случайные нормированные векторы. Перебирая такие векторы, будем получать программные управления, удовлетворяющие ограничениям \eqref{constr}. Концы траекторий, порожденные такими управлениями, покрывают множество достижимости.
        
        
        \item Теперь рассмотрим проекцию на плоскость $ (x_1, x_3) $. Выпишем матрицы $ C $ и $ \widetilde{W}_{x_1,x_3}(\varepsilon) $
        \begin{gather*}
            C = \begin{pmatrix}
                1 & 0 & 0 \\
                0 & 0 & 1
            \end{pmatrix}, \qquad
            \widetilde{W}_{x_1,x_3}(\varepsilon) =  C \widetilde{W} (\varepsilon) C^{\top}  =\begin{pmatrix}
                0 & 0 \\
                0 & 1 \\
            \end{pmatrix} .
        \end{gather*}
        Ситуация аналогична предыдущему случаю: матрица $ \widetilde{W}_{x_1,x_3}(\varepsilon)  $ вырождена, достаточное условие выпуклости не выполняется. Результат численного моделирования приведен на рисунке~\ref{fig:RS}-\subref{fig:u=0_x1-x3}.
        \item Наконец, перейдем к плоскости $ (x_2, x_3) $. Здесь
        \begin{gather*}
            C = \begin{pmatrix}
                0 & 1 & 0 \\
                0 & 0 & 1
            \end{pmatrix}, \qquad \widetilde{W}_{x_2,x_3}(\varepsilon) =  C \widetilde{W} (\varepsilon) C^{\top}  =\begin{pmatrix}
                \frac{\varepsilon^2}{3} & \frac{\varepsilon}{2} \\
                \frac{\varepsilon}{2} & 1
            \end{pmatrix}.
        \end{gather*}
        В этом случае матрица $ \widetilde{W}_{x_2,x_3}(\varepsilon) $ не вырождена и ее минимальное собственное число $ \nu^{x_2,x_3} = \frac{\varepsilon^2}{12} + O(\varepsilon^4)  $ удовлетворяет критерию \eqref{cond1} при достаточно малых $ \varepsilon $ и множество достижимости по выходу $ G_{x_2,x_3}(\varepsilon) $, как следует из теоремы \ref{th2}, выпукло и асимптотически эквивалентно соответствующему множеству достижимости линеаризованной системы, что и проиллюстрировано на рисунке~\ref{fig:RS}-\subref{fig:u=0_x2-x3}. Пунктирной линией на рисунке показана точная граница множества достижимости линеаризованной системы, сдвинутая на $ Cx(\varepsilon, 0) $. Из рисунка видно, что эта граница (эллипс) практически совпадает с границей множества достижимости нелинейной системы.
    \end{enumerate} 
    
    \begin{figure}[ht!] 
        \hspace{-2.5ex}
        \begin{minipage}[b]{.49\linewidth} 
            \small
            \centering 
            \includegraphics[width=\linewidth]{images/OsipovI_u=0_x1-x2.eps}
            %\input{OsipovI_u=0_x1-x2_0.tex}
            \subcaption{$ G_{x_1, x_2}(\varepsilon) $ системы \eqref{unicycle0};}
            \label{fig:u=0_x1-x2} 
        \end{minipage}
        \hfill
        \begin{minipage}[b]{.49\linewidth} 
            \small
            \centering
            \includegraphics[width=\linewidth]{images/OsipovI_u=1_x1-x2.eps}
            %\input{OsipovI_u=1_x1-x2_0.tex}
            \subcaption{$ G_{x_1, x_2}(\varepsilon) $ системы \eqref{unicycle1};}
            \label{fig:u=1_x1-x2}  
        \end{minipage} 
        \vfill
        \hspace{-2.5ex}
        \begin{minipage}[b]{.49\linewidth} 
            \small
            \centering 
            \includegraphics[width=\linewidth]{images/OsipovI_u=0_x1-x3.eps}
            %\input{OsipovI_u=0_x1-x3_0.tex}
            \subcaption{$ G_{x_1, x_3}(\varepsilon) $ системы \eqref{unicycle0};}
            \label{fig:u=0_x1-x3} 
        \end{minipage}
        \hfill
        \begin{minipage}[b]{.49\linewidth} 
            \small
            \centering
            \includegraphics[width=\linewidth]{images/OsipovI_u=1_x1-x3.eps}
            %\input{OsipovI_u=1_x1-x3_0.tex}
            \subcaption{$ G_{x_1, x_3}(\varepsilon) $ системы \eqref{unicycle1};}
            \label{fig:u=1_x1-x3}  
        \end{minipage} 
        \vfill
        \hspace{-2.5ex}
        \begin{minipage}[b]{.49\linewidth} 
            \small
            \centering 
            \includegraphics[width=\linewidth]{images/OsipovI_u=0_x2-x3.eps}
            %\input{OsipovI_u=0_x2-x3_0.tex}
            \subcaption{$ G_{x_2, x_3}(\varepsilon) $ системы \eqref{unicycle0};}
            \label{fig:u=0_x2-x3} 
        \end{minipage}
        \hfill
        \begin{minipage}[b]{.49\linewidth} 
            \small
            \centering
            \includegraphics[width=\linewidth]{images/OsipovI_u=1_x2-x3.eps}
            %\input{OsipovI_u=1_x2-x3_0.tex}
            \subcaption{$ G_{x_2, x_3}(\varepsilon) $ системы \eqref{unicycle1};}
            \label{fig:u=1_x2-x3}  
        \end{minipage}
        \caption{Результаты численного эксперимента для $ \varepsilon = 0.01 $.}\label{fig:RS}
    \end{figure}
    
    Немного изменим рассмотренный пример для того, чтобы линеаризованная система оставалась управляемой. Итак, рассматривается нелинейная система
    \begin{gather}\label{unicycle1}
        \dot{x_1} = \cos(x_3), \qquad
        \dot{x_2} = \sin(x_3), \qquad
        \dot{x_3} = 1 + u(t), \qquad 0 \leq t \leq \varepsilon
    \end{gather}
    при интегральных ограничениях на управление 
    \begin{gather*}
        \int_0^1 u^2(t) dt \leqslant 1
    \end{gather*}
    и нулевых начальных условиях $ x_1(0) = x_2(0) = x_3(0) = 0 $. Фактически, это то же самое, что рассматривать исходную систему при ограничении $ \displaystyle{\int_0^1} \left( u(t) - 1\right)^2 \ dt \leqslant 1$.
    
    Порождённое нулевым управлением $ u(t) \equiv 0 $ решение обозначим $ x(t,0) = x(t) $ и будем использовать его как опорное. 
    \begin{gather}\label{trj}
        \begin{gathered}
            \dot{x_3} = 1 \longrightarrow x_3(t) = x_3(0) + t = t, \\
            \dot{x_2} = \sin(x_3(t)) = \sin(t) \longrightarrow x_2(t) = x_2(0) + \int_0^t \sin(\tau) d\tau = 1 - \cos(t),\\
            \dot{x_1} = \cos(x_3(t)) = \cos(t) \longrightarrow x_1(t) = x_1(0) + \int_0^t \cos(\tau) d\tau = \sin(t).
        \end{gathered}
    \end{gather}
    Выпишем матрицы линеаризованной вдоль траектории \eqref{trj} системы 
    \begin{gather*}
        A(t) = \begin{pmatrix}
            0 & 0 & -\sin(t) \\ 
            0 & 0 & \cos(t) \\ 
            0 & 0 & 0
        \end{pmatrix}, \qquad  B = \begin{pmatrix}
            0 \\ 
            0 \\ 
            1
        \end{pmatrix}.
    \end{gather*}
    Для изучения грамиана управляемости выпишем фундаментальную матрицу линеаризованной системы
    \begin{gather*}
        \begin{gathered}
            X(t,\tau) = \begin{pmatrix}
                1 & 0 & \cos(t)-\cos(t_1) \\ 
                0 & 1 & \sin(t)-\sin(t_1) \\ 
                0 & 0 & 1
            \end{pmatrix}.
        \end{gathered}    
    \end{gather*}
    Грамиан управляемости имеет вид
    \begin{gather*}
        W(\varepsilon) = \int_0^{\varepsilon}X(\varepsilon,t) B B^{\top} X^{\top}(\varepsilon,t)dt =
    \end{gather*}
    \begin{gather*}
        =\begin{pmatrix}
            \varepsilon-\dfrac{3}{4} \sin(2\varepsilon) +\dfrac{1}{2}\varepsilon \cos(2\varepsilon)& \dfrac{3}{2}\cos^2(\varepsilon) - \cos(\varepsilon) + \dfrac{1}{2}\varepsilon \sin(2 \varepsilon) - \dfrac{1}{2} &  \varepsilon\cos( \varepsilon)-\sin( \varepsilon) \\[8pt] 
            * & \dfrac{3}{4}\sin(2\varepsilon) + \dfrac{1}{2}\varepsilon + \varepsilon \sin^2(\varepsilon) - 2 \sin(\varepsilon) & \cos(\varepsilon) + \varepsilon\sin(\varepsilon)-1 \\ 
            * & * & \varepsilon
        \end{pmatrix}.    
    \end{gather*}
    Здесь и далее, будем заменять элементы симметричных матриц под главной диагональю на $ * $ для лаконичной записи.
    Проделаем замену времени $ t = \varepsilon \tau $ и выпишем грамиан управляемости $ \widetilde{W}(\varepsilon) $ линеаризованной системы в новом времени $ \tau $
    \begin{gather*}
        \widetilde{W}(\varepsilon) = \dfrac{1}{\varepsilon} W(\varepsilon) = 
    \end{gather*} \footnotesize
    \begin{gather*}
        =\begin{pmatrix} 
            \cos^2(\varepsilon)-\dfrac{3}{4\varepsilon}\sin(2\varepsilon)+\dfrac{1}{2} & 
            \cos\left(\varepsilon \right)\,\sin\left(\varepsilon \right)+\dfrac{1}{2\,\varepsilon}\left( 3\cos^2\left(\varepsilon \right)-2\cos\left(\varepsilon\right)-1\right) &
            \cos\left(\varepsilon \right)-\dfrac{1}{\varepsilon} \sin\left(\varepsilon \right) \\[8pt] 
            * &
            \dfrac{3}{2}-\dfrac{2\,\sin\left(\varepsilon \right)-\dfrac{3\,\cos\left(\varepsilon \right)\,\sin\left(\varepsilon \right)}{2}}{\varepsilon }-{\cos\left(\varepsilon \right)}^2 & \sin\left(\varepsilon \right)+\dfrac{1}{\varepsilon } \left(\cos\left(\varepsilon \right)-1 \right)\\
            * &
            * & 
            1 \end{pmatrix}.    
    \end{gather*}
    \normalsize
    Далее последовательно рассмотрим проекции системы \eqref{unicycle1} на координатные плоскости $ (x_1, x_2) $, $ (x_1, x_3) $, $ (x_2, x_3) $. \\
    
    \begin{enumerate}
        \item Будем рассматривать проекцию $ G_{x_1, x_2}(\varepsilon) $ области достижимости системы \eqref{unicycle1} на плоскость первых двух фазовых координат. Матрица проектирования будет иметь вид
        \begin{gather*}
            C = \begin{pmatrix}
                1 & 0 & 0 \\ 
                0 & 1 & 0 
            \end{pmatrix}. 
        \end{gather*} 
        Тогда: \small
        \begin{gather}\label{Ve}
            \widetilde{W}_{x_1,x_2}(\varepsilon)= \begin{pmatrix}
                \cos^2(\varepsilon)-\dfrac{3}{4\varepsilon}\sin(2\varepsilon)+\dfrac{1}{2} & 
                \cos\left(\varepsilon \right)\,\sin\left(\varepsilon \right)+\dfrac{1}{2\,\varepsilon}\left( 3\cos^2\left(\varepsilon \right)-2\cos\left(\varepsilon\right)-1\right) \\[6pt]
                * &
                \dfrac{3}{2}-\dfrac{1}{\varepsilon }\left(2\,\sin\left(\varepsilon \right)-\dfrac{3\,\cos\left(\varepsilon \right)\,\sin\left(\varepsilon \right)}{2} \right) -{\cos\left(\varepsilon \right)}^2 
            \end{pmatrix}
        \end{gather}
        \normalsize
        Для исследования асимптотики $ \nu^{x_1,x_2}(\varepsilon) $ -- минимального собственного числа $ \widetilde{W}_{x_1,x_2}(\varepsilon) $, перепишем \eqref{Ve}, разложив тригонометрические функции в ряд вблизи точки $ \varepsilon = 0 $.
        \begin{gather*}
            \widetilde{W}_{x_1,x_2}(\varepsilon) = \begin{pmatrix}
                \dfrac{2\,\varepsilon ^4}{15} + O(\varepsilon^6)&
                -\dfrac{5\,\varepsilon ^3}{24} + O(\varepsilon^5)\\[8pt]
                -\dfrac{5\,\varepsilon ^3}{24} + O(\varepsilon^5) & 
                \dfrac{\varepsilon ^2}{3}-\dfrac{3\,\varepsilon ^4}{20} + O(\varepsilon^6).
            \end{pmatrix}.
        \end{gather*}
        
        Минимальное собственное число $ \nu^{x_1,x_2}(\varepsilon) = \frac{1}{120}\varepsilon^4 + O(\varepsilon^6)$, а $ \varepsilon^4 < \varepsilon^{3-\alpha} $ для всех $ \alpha > 0 $ при достаточно малых $ \varepsilon $, то есть достаточное условие выпуклости $ G_{x_1, x_1}(\varepsilon) $ не выполняется. Результаты численного моделирования, приведённые на рисунке~\ref{fig:RS}-\subref{fig:u=1_x1-x2}, показывают невыпуклость проекции. 
        \item Перейдем к плоскости $ (x_1,x_3) $. 
        \begin{gather*}
            C = \begin{pmatrix}
                1 & 0 & 0 \\
                0 & 0 & 1
            \end{pmatrix}, \qquad
            \widetilde{W}_{x_2,x_3}(\varepsilon) =  C \widetilde{W} (\varepsilon) C^{\top}  = \\ =\begin{pmatrix}
                \cos^2(\varepsilon)-\dfrac{3}{4\varepsilon}\sin(2\varepsilon)+\dfrac{1}{2} & 
                \cos\left(\varepsilon \right)-\dfrac{1}{\varepsilon} \sin\left(\varepsilon \right) \\ 
                * & 1
            \end{pmatrix}.
        \end{gather*}
        Так же, как и в случае плоскости $ (x_1,x_2) $ разложим компоненты $     \widetilde{W}_{x_1,x_3}(\varepsilon)  $ в ряд вблизи точки $ \varepsilon = 0 $:
        \begin{gather*}
            \widetilde{W}_{x_1,x_3}(\varepsilon) = \begin{pmatrix} 
                \dfrac{2\,\varepsilon ^4}{15} + O(\varepsilon^6) &
                -\dfrac{\varepsilon^2}{3}+ O(\varepsilon ^4)\\[8pt]
                -\dfrac{\varepsilon^2}{3} + O(\varepsilon^4) & 1 \end{pmatrix}.
        \end{gather*}
        Минимальное собственное число матрицы $  \widetilde{W}_{x_1,x_3}(\varepsilon)  $, $ \nu^{x_1,x_3} =   \frac{1}{45}\varepsilon^4 + O(\varepsilon^6) $, также не удовлетворяет условию \eqref{cond1}. Соответствующий результат численного построения проекции множества достижимости показан на рисунке~\ref{fig:RS}-\subref{fig:u=1_x1-x3}.
        \item Последний случай -- плоскость $ (x_2,x_3) $.
        \begin{gather*}
            C = \begin{pmatrix}
                0 & 1 & 0 \\
                0 & 0 & 1
            \end{pmatrix}, \qquad
            \widetilde{W}_{x_2,x_3}(\varepsilon) =  C \widetilde{W} (\varepsilon) C^{\top}  =\\=\begin{pmatrix}
                \dfrac{3}{2}-\dfrac{2\,\sin\left(\varepsilon \right)-\dfrac{3\,\cos\left(\varepsilon \right)\,\sin\left(\varepsilon \right)}{2}}{\varepsilon }-{\cos\left(\varepsilon \right)}^2 & \sin\left(\varepsilon \right)+\dfrac{1}{\varepsilon } \left(\cos\left(\varepsilon \right)-1 \right)\\[8pt]
                * & 
                1 
            \end{pmatrix}.
        \end{gather*}
        Точно также разложим $ \widetilde{W}_{x_2,x_3}(\varepsilon) $ в ряд
        \begin{gather*}
            \widetilde{W}_{x_2,x_3}(\varepsilon)  = \begin{pmatrix}
                \dfrac{\varepsilon^2}{3} + O(\varepsilon^4) &
                \dfrac{\varepsilon }{2} + O(\varepsilon^3) \\[8pt]
                \dfrac{\varepsilon }{2} + O(\varepsilon^3) & 1
            \end{pmatrix}
        \end{gather*}
        Минимальное собственное число в этом случае равно $ \nu^{x_2,x_3}(\varepsilon) = \frac{\varepsilon^2}{12} + O(\varepsilon^4) $, то есть удовлетворяет условию \eqref{cond1}. Выпуклость этой проекции проиллюстрирована на рисунке~\ref{fig:RS}-\subref{fig:u=1_x2-x3}.  Как и на рисунке~\ref{fig:RS}-\subref{fig:u=0_x2-x3}, здесь пунктирной линией показана точная граница множества достижимости линеаризованной системы.
    \end{enumerate}
\end{document}