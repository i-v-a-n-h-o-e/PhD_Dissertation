\documentclass[../main.tex]{subfiles}
\begin{document}
\clearpage
\section{Выпуклость множеств достижимости квазилинейных систем с интегральными ограничениями на управление}
В этой главе исследуется выпуклость множеств достижимости для квазилинейных систем с интегральными ограничениями. 
Опираясь на работы Б.\,Т.~Поляка \cite{Polyak2001,Polyak2004} по образу малого гильбертова шара при нелинейных отображениях, исследование распространяет анализ на сценарии, когда в правой части системы существует малая нелинейность. 
Если малый параметр принимает нулевое значение, то квазилинейная система становится линейной, а ее множество достижимости выпуклым. 
Выясняется, что для сохранения выпуклости множества достижимости этих систем производная нелинейного отображения должна быть непрерывной по Липшицу. 
Доказательство следует схеме Поляка. 

Глава организована следующим образом. 
Постановка задачи и некоторые предварительные замечания составляют первый раздел. 
Второй раздел содержит исследование нелинейного отображения, зависящего от параметра. 
В следующем разделе мы применяем результаты предыдущего раздела к отображению, порождаемому квазилинейной системой. 
Наконец, в третьем разделе мы приводим два иллюстративных примера.

\subsection{Постановка задачи и предварительные замечания}

Рассмотрим квазилинейную систему
\begin{gather}\label{sec3:nonlinear}
 \dot{x}(t) = A(t)x(t)+B(t)u(t)+\varepsilon f\big(x(t),t\big), \qquad t_0 \leqslant t \leqslant T, \qquad x(t_0) = x_0,
\end{gather}
где $ x \in \mathbb{R}^n $ --- вектор состояния, $ u \in \mathbb{R}^r $ --- вектор управления, $t_0$ --- неотрицательное число, $T$ --- положительное число, а $\varepsilon$ --- малый параметр, такой что $\varepsilon \in [0,\overline{\varepsilon}]$, $ \overline{\varepsilon} > 0$. 
Матричные отображения $A:[t_0,T] \to \mathbb{R}^{n\times n} $, $B: [t_0,T] \to \mathbb{R}^{n\times r} $ предполагаются непрерывными. 
Вектор-функция $f: \mathbb{R}^n \times [t_0,T] \to \mathbb{R}^n$ непрерывна по паре $(x,t)$ и непрерывно-дифференцируема по $x$.

Система \eqref{sec3:nonlinear} является частным случаем системы \eqref{s1:common_nonlinear}, которая исследовалась в Главе \ref{s1}, при $f_1(t,x) = A(t) x(t) + \varepsilon f(x(t),t)$ и $f_2(t, x(t)) = B(t)$.

Управление $ u(\cdot) $ будем выбирать из шара $ B_{\mathbb{L}_2}(0,\mu) $, где $ \mu > 0$.

Каждому управлению $u(\cdot) \in \mathbb{L}_2$ и любому $\varepsilon \in [0,\overline{\varepsilon}]$ соответствует единственное абсолютно непрерывное решение $ x(t,\varepsilon, u(\cdot)) $ системы \eqref{sec3:nonlinear}, удовлетворяющее начальному условию $ x(t_0,\varepsilon, u(\cdot)) = x_0$, и это решение продолжимо на интервал $[t_0, t_0 + \Delta]$, где $t_0 + \Delta < T$. 

Далее, мы будем считать, что Предположения \ref{s1:as:right_hand_side_conditions_global} и \ref{s1:as:right_hand_side_diff_lip} выполнены для правой части системы \eqref{sec3:nonlinear} при управлениях $ u(\cdot) \in B_{\mathbb{L}_2}(0,\mu) $ на интервале $ [t_0, T]$.

В частности, Предположение \ref{s1:as:right_hand_side_conditions_global} выполняется, если $f$ удовлетворяет одному из следующих условий \cite{Filippov2}:
\begin{gather}\label{sec3:sublinear_growth}
 \left\|f\big(x,t\big) \right\| \leqslant l_1(t) (1 + \|x\|), \\ 
 x \cdot f(x,t) \leqslant a(t) \|x\|^2 + b(t),
\end{gather}
где $l_1(\cdot) \in \mathbb{L}_1[t_0,T]$ и $a(\cdot), b(\cdot)$ --- некоторые непрерывные функции.

\begin{definition} 
 Множеством достижимости $G(T,\mu,\varepsilon) $ системы \eqref{sec3:nonlinear} в момент $T$ называется множество всех возможных состояний, в которые система может быть переведена к моменту $T$ при помощи управлений $ u(\cdot) \in B_{\mathbb{L}_2}(0,\mu) $.
 \begin{gather*}
 G(T,\mu,\varepsilon) =\{\widetilde{x}\in \mathbb{R}^n:\exists u(\cdot)\in B_{\mathbb{L}_2}(0,\mu),\; x(T,\varepsilon,u(\cdot)) = \widetilde{x}\}.
 \end{gather*}
\end{definition} 
Возникает вопрос: при каких условиях множество достижимости системы \eqref{sec3:nonlinear} будет сохранять выпуклость при малых $\varepsilon$?

\subsection{Нелинейное отображение с малым параметром}

В этом параграфе, $x$ (а также $x_0$, $x_1$ и другие) никак не связан с вектором состояния системы \eqref{sec3:nonlinear}.
Здесь $x$ --- это элемент пространства $X$, а вот $\varepsilon$ по-прежнему малый неотрицательный параметр.

Будем рассматривать отображение $F(x, \varepsilon) = a_0 + A_0x + \varepsilon A_1(x,\varepsilon): B_X(0, r) \times \mathbb{R}_+ \rightarrow Y$, где $X$ и $Y$ --- гильбертовы пространства.
Здесь $a_0 \in Y$ --- константа, которая не зависит ни от $x$, ни от $\varepsilon$; $A_0$ --- линейный оператор, который предполагается сюръективным отображением $X$ в $Y$. 
%It is assumed that for $\varepsilon = 0$ the mapping $F(x, 0)=A_0x$ is a continuous operator which is linear with respect to $x$ and is a surjective mapping from $X$ to $Y$. 
Из последнего следует, что существует такое $\nu > 0$, что

\begin{gather}\label{regular}
 \| A_0^*y\| \geqslant \nu \|y\|, \quad \forall y \in Y.
\end{gather}
Здесь $A_0^* $ --- сопряженный к $A_0$ линейный оператор.
Наконец, $A_1: B_X(0, r) \times \mathbb{R}_+ \to Y $ --- это нелинейный оператор, непрерывный по $x$ и $\varepsilon$.
\begin{assumption}\label{as:derivative_of_A1}
 Существует такое $\overline{\varepsilon} > 0$, что для всех $x \in B_X(0,r)$, $\varepsilon \in [0, \overline{\varepsilon}]$ отображение $A_1(x, \varepsilon)$ имеет производную Фреше $\frac{\partial A_1(x, \varepsilon)}{\partial x} = A_1'(x, \varepsilon)$ которая непрерывна по $\varepsilon$ и липшицева по $x$: существует $L>0$, такая что
 \begin{gather*}
 \|A_1'(x_1,\varepsilon) - A_1'(x_2,\varepsilon) \| \leqslant L\|x_1-x_2\|, \qquad x_1, x_2 \in B_X(0,r), \qquad \varepsilon \in [0, \overline{\varepsilon}].
 \end{gather*}
\end{assumption}

Для дальнейших рассуждений нам потребуется процитировать результат из \cite{Polyak2001, Polyak1964}. 
В формулировке следующей леммы предполагается, что $f:X \to Y$ --- это нелинейное отображение, имеющее производную Фреше.
\begin{lemma}[\cite{Polyak2001, Polyak1964}]\label{lem:Polyak_lemma}
 Пусть существуют такие $L$, $\rho$, $\mu > 0$ и $y_0 \in Y$, что 
 \begin{gather*}
 \| f'(x) - f'(z) \| \leqslant L \| x - z\|, \quad \forall x,z \in B(x_0,\rho), \\
 \| f'(x)^*y \| \geqslant \mu \|y \|, \quad \forall y \in Y, \quad \forall x \in B(x_0, \rho), \\
 \| f(x_0) - y_0 \| \leqslant \rho \mu,
 \end{gather*}
 тогда уравнение $f(x) = y_0$ имеет решение $x^* \in B(x_0,\rho)$ и 
 \begin{gather*}
 \|x^* - x_0\| \leqslant \frac{1}{\mu} \left\| f(x_0) - y_0 \right\|.
 \end{gather*}
\end{lemma}
Теперь мы можем сформулировать и доказать следующую теорему.
\begin{theorem}\label{th:ImageConvexity}
 Через $F(x,\varepsilon)$ обозначим образ шара $B_X(0, r)$ при его отображении $F\big(B_X(0,r),\varepsilon\big)$, т.~е. $F\big(B_X(0,r),\varepsilon\big) = \big\{F(x,\varepsilon): x\in B_X(0, r)\big\}$.
 Пусть выполнено Предположение \ref{as:derivative_of_A1} и $F\big(B_X(0,r),\varepsilon\big)$ --- замкнутое множество для каждого $\varepsilon \in [0, \overline{\varepsilon}]$. 
Тогда найдется такое $ \varepsilon_0 \in (0, \overline{\varepsilon}]$, что при всех положительных $\varepsilon < \varepsilon_0$ образ $F\big(B_X(0,r),\varepsilon\big)$ --- выпуклое множество в $Y$. 
\end{theorem}
\doc
Заметим, что константа $a_0$ не влияет на выпуклость образа $F\big(B_X(0,r),\varepsilon\big)$. 
Поэтому в доказательстве мы будем считать ее равной нулю.

Рассмотрим две произвольные точки, $x_1$ и $x_2$, из шара $B_X(0,r)$. 
Обозначим $x_0 = \frac{1}{2}(x_1 + x_2)$, $y(\varepsilon) = \frac{1}{2}\big(y_1(\varepsilon) + y_2(\varepsilon)\big)$, где $y_1(\varepsilon) = F(x_1, \varepsilon)$ и $y_2(\varepsilon) = F(x_2, \varepsilon)$. 

Для того, чтобы доказать выпуклость множества $F\big(B_X(0,r),\varepsilon\big)$, достаточно показать, что $y(\varepsilon) \in F\big(B_X(0,r),\varepsilon\big)$ или, что то же самое, найдется такое $x^* \in B_X(0,r) $, что $F(x^*, \varepsilon) = y(\varepsilon) $.
Выпишем выражение для $y(\varepsilon)$
\begin{gather}\label{y}
 \begin{gathered}
 y(\varepsilon)=
 \frac{1}{2} \big(
 F(x_1,\varepsilon)+ 
 F(x_2,\varepsilon)
 \big) = 
 \frac{1}{2} \big(
 A_0 x_1 +
 \varepsilon A_1(x_1,\varepsilon) +
 A_0 x_2 +
 \varepsilon A_1(x_2,\varepsilon) 
 \big) = \\ = A_0 x_0 + 
 \frac{1}{2} \varepsilon \big( 
 A_1(x_1,\varepsilon)+ 
 A_1(x_2,\varepsilon)
 \big).
 \end{gathered}
\end{gather}

Пусть $x \in X$ и $h \in X$ выбраны так, чтобы выполнялись включения $x\in B_X(0, r)$ и $x+h \in B_X(0, r)$.
В условиях Предположения \ref{as:derivative_of_A1}, разложим $A_1$ в ряд около точки $x$:

\begin{gather}\label{A1_series}
 A_1(x + h,\varepsilon) = A_1(x,\varepsilon) + A_1'(x,\varepsilon) h + R(\varepsilon, x, h).
\end{gather}

Умножив обе части равенства на $y^* \in Y^*$, $\|y^*\| \leqslant 1$ имеем
\begin{gather*}
 \langle y^*, R(\varepsilon, x, h) \rangle = 
 \langle y^*, A_1(x + h,\varepsilon) \rangle -
 \langle y^*, A_1(x,\varepsilon) \rangle -
 \langle y^*, A_1'(x,\varepsilon) h \rangle
\end{gather*}

Применив теорему о среднем к функции $\langle y^*, A_1(x,\varepsilon) \rangle$, получим
\begin{gather*}
 \langle y^*, A_1(x + h,\varepsilon) \rangle -
 \langle y^*, A_1(x,\varepsilon) \rangle = 
 \langle y^*, A_1'(x + \theta h,\varepsilon) h \rangle,
 \qquad
 0 \leqslant \theta \leqslant 1.
\end{gather*}

Два последних соотношения приводят нас к оценке
\begin{gather}
 \begin{gathered}
 \|\langle y^*, R(\varepsilon, x, h) \rangle \| \leqslant
 \| y^* \| 
 \| A_1'(x + \theta h,\varepsilon) -
 A_1'(x,\varepsilon) \| 
 \| h \| \leqslant 
 L \theta \|h\|^2 \leqslant
 L \|h\|^2, \\
 \| R(\varepsilon, x, h) \| \leqslant
 L \|h\|^2, 
 \end{gathered}
\end{gather}

Подставим \eqref{A1_series} в выражение для \eqref{y}

\begin{gather*}
 y(\varepsilon) =
 A_0x_0 +
 \frac{1}{2}\varepsilon \Big(
 A_1(x_0,\varepsilon) +
 A_1'(x_0,\varepsilon)(x_1 - x_0)+ 
 R(\varepsilon, x_0, x_1 - x_0) + \\ 
 A_1(x_0,\varepsilon) +
 A_1'(x_0,\varepsilon)(x_2 - x_0)+ 
 R(\varepsilon, x_0, x_2 - x_0)
 \Big) = \\ = 
 A_0x_0 + 
 \varepsilon A_1(x_0,\varepsilon) +
 \xi(\varepsilon,x_1,x_2).
\end{gather*}
где остаточный член может быть записан в форме
\begin{gather*}
 \xi(\varepsilon,x_1,x_2) = \frac{1}{2}\varepsilon\big(R(\varepsilon, x_0, x_1 - x_0) + R(\varepsilon, x_0, x_2 - x_0)\big),
\end{gather*}
и оценен через
\begin{gather*}
 \|\xi(\varepsilon,x_1,x_2)\| \leqslant \frac{1}{2}\varepsilon L \left(\frac{1}{4}\|x_1 - x_2\|^2 + \frac{1}{4}\|x_1 - x_2\|^2 \right) \leqslant \frac{1}{4}L\varepsilon\|x_1 - x_2\|^2. 
\end{gather*}

Таким образом, $y(\varepsilon) = A_0x_0 + \varepsilon A_1(x_0,\varepsilon) + \xi(\varepsilon,x_1,x_2) = F(x_0,\varepsilon) + \xi(\varepsilon,x_1,x_2)$ для всех $x_1, x_2 \in B(0,r)$, $x_1 \neq x_2$, $\varepsilon \in [0, \overline{\varepsilon}]$, а значит выполняется следующее неравенство
\begin{gather*}
 \| F(x_0,\varepsilon) - y(\varepsilon) \| = \|\xi(\varepsilon,x_1,x_2)\| \leqslant \frac{1}{4}L\varepsilon\|x_1-x_2\|^2.
\end{gather*}


Теперь рассмотрим производную $F(x_0, \varepsilon)$ по $x_0$ при фиксированном $\varepsilon$, $F_x'(x_0,\varepsilon) = A_0 + \varepsilon A_1'(x_0,\varepsilon) $.

Из Предположения \ref{as:derivative_of_A1} мы можем оценить $\|A_1'(x_0,\varepsilon\|$ сверху:
\begin{gather*}
 \|A_1'(x_0,\varepsilon) - A_1'(0,\varepsilon)\| \leqslant 
 L\|x_0\| \leqslant
 L r, \\
 \|A_1'(x_0,\varepsilon)\| \leqslant \| A_1'(0,\varepsilon)\| + Lr.
\end{gather*}

Из равенства $\|A_1'(x_0,\varepsilon)\| = \|\big(A_1'(x_0,\varepsilon)\big)^*\| $, следует, что $\|\big(A_1'(x_0,\varepsilon)\big)^*\| \leqslant \| A_1'(0,\varepsilon)\| + Lr$.
Отсюда и из соотношения \eqref{regular}, мы получаем 
\begin{gather*}
 \left\|F'_x(x_0, \varepsilon)^* y\right\| = \left\|\big(A_0 + \varepsilon A_1'(x_0, \varepsilon)\big)^* y\right\| \geqslant \left\|A_0^*y \right\| - \varepsilon \left\|\big(A_1'(x_0,\varepsilon)\big)^*\right\| \left\|y\right\| \geqslant (\nu - k\varepsilon)\|y\|,
\end{gather*} где $k = \max\limits_{\varepsilon \in [0,\overline{\varepsilon}]}\| A_1'(0,\varepsilon)\| + Lr > 0$.
Для малых $\varepsilon$, выполняется следующее неравенство $(\nu - k\varepsilon) \geqslant \dfrac{\nu}{2}$ и мы получаем $\left\|F'_x(x_0, \varepsilon)^* y\right\| \geqslant \dfrac{\nu}{2} \|y\|$. 
Для того, чтобы использовать Лемму \ref{lem:Polyak_lemma}, потребуем

\begin{gather*}
 \| F(x_0,\varepsilon) - y(\varepsilon) \| = \|\xi(\varepsilon,x_1,x_2)\| \leqslant \frac{1}{4}L\varepsilon\|x_1-x_2\|^2 \leqslant \dfrac{\nu}{2} \dfrac{\|x_1-x_2\|^2}{8r}.
\end{gather*}

Чтобы это неравенство выполнялось, значение $\varepsilon$ должно удовлетворять 
\begin{gather}
 \varepsilon \leqslant \varepsilon_0 = \min\left\{\dfrac{\nu}{4Lr}, \dfrac{\nu}{2k}, \overline{\varepsilon}\right\}.
\end{gather} 

Наконец, из Леммы \ref{lem:Polyak_lemma} с параметрами $\mu=\dfrac{\nu}{2}$ и $\rho=\dfrac{\|x_1-x_2\|^2}{8r}$ следует, что существует такой $x^*\in B(x_0, \rho)$, что $F(x^*,\varepsilon) = y(\varepsilon)$.
Так как $B_X(0, r)$ --- гильбертов шар, строго выпуклое множество, то выполняется включение $B(x_0, \rho) \subset B_X(0, r)$, следовательно $x^* \in B_X(0, r)$. 
Таким образом, точка $y(\varepsilon) = \frac{1}{2} \big( F(x_1,\varepsilon) + F(x_2,\varepsilon)\big)$ лежит в образе шара $F\big(B_X(0,r),\varepsilon\big) $ при всех $\varepsilon \leqslant \varepsilon_0$ и $x_1, x_2 \in B_X(0,r)$. 
Из-за замкнутости, для всех $\varepsilon \leqslant \varepsilon_0$, образ шара $F\big(B_X(0,r),\varepsilon\big) $ будет выпуклым.
\hfill$\square$\\[1ex]%--- P r o o f.

\subsection{Некоторые свойства решений квазилинейных систем}


В этом разделе исследуются решения \eqref{sec3:nonlinear} для того, чтобы обосновать применение результатов предыдущего раздела, а именно Теоремы \ref{th:ImageConvexity}. 

Обозначим фундаментальную матрицу системы $\dot{x}(t) = A(t) x(t)$ через $X(t,\tau)$.
Эта матрица является решением уравнения
\begin{gather*}
 \frac{\partial X(t,\tau)}{\partial t} = A(t) X(T,\tau), \qquad X(\tau,\tau) = I.
\end{gather*}

Если $x(\cdot,\varepsilon, u(\cdot))$ --- решение \eqref{sec3:nonlinear}, порожденное управлением $u(\cdot)$ и начальным условием $x_0$, то оно удовлетворяет интегральному уравнению
\begin{gather*}
 x\big(T,\varepsilon, u(\cdot)\big) =
 X(T,t_0)x_0 + 
 \int\limits_{t_0}^T X(T,\tau) \bigg(Bu(\tau) +
 \varepsilon f\Big(x\big(\tau,\varepsilon, u(\cdot)\big),\tau\Big) \bigg)\ d\tau = \\ =
 X(T,t_0)x_0 +
 \int\limits_{t_0}^T X(T,\tau) B(t)u(\tau)\ d\tau 
 + \varepsilon \int\limits_{t_0}^T X(T,\tau) f\Big(x\big(\tau,\varepsilon, u(\cdot)\big),\tau\Big) \ d\tau.
\end{gather*}

Определим отображение $F:B_{\mathbb{L}_2}(0,\overline{\mu})\times [0,\overline{\varepsilon}] \to \mathbb{R}^n$ равенством $F(u(\cdot),\varepsilon) = x(T,\varepsilon,u(\cdot))$, где $x(T,\varepsilon,u(\cdot))$ --- решение \eqref{sec3:nonlinear} в момент $T$ отвечающее управлению $u(\cdot)$ и малому параметру $\varepsilon$.

Для того, чтобы использовать результаты предыдущего раздела, перепишем $F$ в виде
\begin{gather*}
 F(u(\cdot),\varepsilon) = a_0 + A_0 u(\cdot) + \varepsilon A_1(u(\cdot), \varepsilon), 
\end{gather*}
где $a_0 = X(T,0)x_0 $, а отображения $A_0: B_{\mathbb{L}_2}(0,\overline{\mu}) \mapsto \mathbb{R}^n$ и $A_1: B_{\mathbb{L}_2}(0,\overline{\mu}) \times [0,\overline{\varepsilon}] \to \mathbb{R}^n$ определены равенствами
\begin{gather}\label{A1_def}
 A_0 u(\cdot) = \int\limits_{t_0}^T X(T,\tau) B(t)u(\tau)\ d\tau, \qquad
 A_1(u(\cdot),\varepsilon) = \int\limits_{t_0}^T X(T,\tau) f\Big(x\big(\tau,\varepsilon, u(\cdot)\big),\tau\Big) \ d\tau.
\end{gather}

Множество достижимости $G(T,\mu,\varepsilon) $ квазилинейной системы \eqref{sec3:nonlinear} --- это образ шара $B_{\mathbb{L}_2}(0,\mu)$ при его отображении $F$, $G(T,\mu,\varepsilon) = F(B_{\mathbb{L}_2}(0,\mu),\varepsilon)$.

\begin{utv}\label{ReachableSetcloseness}
 Пусть выполнены условия Предположений \ref{s1:as:right_hand_side_conditions_global} и \ref{s1:as:right_hand_side_diff_lip}. 
Тогда, для всех $\varepsilon\in [0,\overline{\varepsilon}]$, множество достижимости $G(T,\mu,\varepsilon) $ --- замкнуто.
\end{utv}
\doc. 
Доказательство следует из равностепенной непрерывности траекторий, равномерной ограниченности траекторий и слабой компактности $B_{\mathbb{L}_2}(0,\mu)$ (см., например \cite{GusZyk}).
\hfill$\square$\\[1ex]%--- P r o o f.

Для того, чтобы применить Теорему \ref{th:ImageConvexity} к отображению $F$, нам необходимо показать, что Предположение \ref{as:derivative_of_A1} выполняется для $A_1$, определенного в \eqref{A1_def}.

Применяя Лемму \ref{s1:lem:lip_of_solutions_global} к системе \eqref{sec3:nonlinear} получим, что для всех $\varepsilon\in [0,\overline{\varepsilon}]$, найдется такая $L_x(\varepsilon) > 0$, что для любых $u_i(\cdot) \in B_{\mathbb{L}_2}(0,\mu), \ i = 1,2$ и $t \in [t_0,T]$, 
\begin{gather*}
	\| x_1(t) - x_2(t) \| \leqslant L_x(\varepsilon) \| u_1(\cdot) - u_2(\cdot) \|_{\mathbb{L}_2},
\end{gather*}
где $x_i(t) = x(t,\varepsilon,u_i(\cdot))$, $i = 1,2$. 
Кроме того, $L_x(\varepsilon) \leqslant L_x(\overline{\varepsilon})$. 
%\begin{lemma}\label{lem:Lipx}
% Пусть выполнены условия Предположения \ref{as:right_hand_side_conditions_quasilinear}. 
%Тогда, для 
%\end{lemma}
%\doc.
%Так как $x_i(t) \in D$ для всех $t\in [t_0,T]$, из Предположения \ref{as:right_hand_side_conditions_quasilinear}, мы имеем
%\begin{gather*}
% \|f\big(x_1(t),t\big) - f\big(x_2(t),t\big)\| \leqslant L_f\|x_1(t) - x_2(t)\|.
%\end{gather*}
%Из интегрального соотношения
%\begin{gather}\label{nonlinear_solution}
% x_i(t) = x_0 + \int\limits_{t_0}^t A(\tau)x_i(\tau)\ d\tau + \int\limits_{t_0}^t B(\tau)u_i(\tau)\ d\tau+
% \varepsilon\int\limits_{t_0}^t f\big(x_i(\tau),\tau\big)\ d\tau,
%\end{gather}
%мы имеем
%\begin{gather*}
% \| x_1(t) - x_2(t) \| \leqslant
% \left\| \int\limits_{t_0}^t A(\tau)\big(x_1(\tau) - x_2(\tau)\big)\ d\tau \right\| + 
% \left\|\int\limits_{t_0}^t B(\tau) \big(u_1(\tau) - u_2(\tau)\big)\ d\tau \right\| + \\ +
% \varepsilon \left\| \int\limits_{t_0}^t \Big( f\big(x_1(\tau),\tau\big) - f\big(x_2(\tau),\tau\big) \Big)\ d\tau \right\| 
% \leqslant \\ \leqslant
% \int\limits_{t_0}^t \big(k_A + L_f \varepsilon) \left\| x_1(\tau) - x_2(\tau)\right\| \ d\tau + k_u \| u_1(\cdot) - u_2(\cdot) \|_{\mathbb{L}_2}.
%\end{gather*}
%Здесь, 
%\begin{gather*}
% k_u = \sqrt{(T-t_0) \max\limits_{\tau \in [{t_0},t]}\|B(\tau)\|}, \quad k_A =\max\limits_{\tau \in [{t_0},t]} \| A(\tau)\|.
%\end{gather*}
%Из неравенства Гронуолла 
%\begin{gather*}
% \| x_1(t) - x_2(t) \| \leqslant L_x(\varepsilon) \| u_1(\cdot) - u_2(\cdot) \|_{\mathbb{L}_2},
%\end{gather*}
%где $L_x(\varepsilon) = k_u \exp\big((k_A + L_f \varepsilon)(T - t_0)\big) $. 
%Заметим, что $L_x(\varepsilon) \leqslant L_x(\overline{\varepsilon})$. 
%\hfill$\square$\\[1ex]%--- P r o o f.



% To analyze the derivative of mapping $A_1$, we require a lemma about the differentiation of an integral mapping. The following result is similar to the one in the book\cite{Luenberger}, but with one distinction: the book provides it for a function from $\mathbb{C}$, while in our analysis, we require this fact for a function from $\mathbb{L}_2$.

% \begin{lemen}\label{LeibnizRule}\cite[p.~174]{Luenberger}
 % Let $f: \mathbb{L}_2 \to \mathbb{R}^n$ be a nonlinear mapping, defined by $f(x(\cdot)) = \\ = \int_{t_0}^T g(x(t),t)\ dt$, where $g: \mathbb{L}_2 \times \mathbb{R} \to \mathbb{R}^n$ is mapping with Frechet derivative $g_x:\mathbb{L}_2 \times \mathbb{R} \to \mathbb{R}^n$, which is uniform continuous with respect to x and t. Then, the Frechet differential of $f$ is
 % \begin{gather*}
 % \delta f(x,\delta x(\cdot)) = \int\limits_{t_0}^T g_x(x(t),t) \delta x(t)\ dt.
 % \end{gather*}
 % \end{lemen}
% \proofen 
% We have 
% \begin{gather*}
 % \left\|f(x+\delta x) - f(x) - \delta f (x,\delta x)\right\| = \left\|\int\limits_{t_0}^T
 % \Big(g\big(x(t)+\delta x(t),t\big) - g\big(x(t),t\big) - g_x\big(x(t),t\big)\delta x(t)\Big) \ dt\right\|
 % \end{gather*}
% For almost all fixed $t$ we have, by the mean value theorem,
% \begin{gather}\label{meanvalue}
 % g\big(x(t)+\delta x(t),t\big) - g\big(x(t),t\big) = g_x\big(\overline{x}(t),t\big)\delta x(t),
 % \end{gather}
% where $\|x(t) - \overline{x}(t)\| \leqslant \|\delta x(t)\| $. 
% The left part of the equality \eqref{meanvalue} is measurable, therefore the right part is measurable as well.
% The uniform continuity of $g_x$ with recpect to $x$ implies
% \begin{gather*}
 % \|g_x(\overline{x}(t),t) \delta x(t) - g_x(x,t) \delta x(t) \| \leqslant L_g \|x(t) - \overline{x}(t)\| \|\delta x(t)\| \leqslant L_g \|\delta x(t)\|^2.
 % \end{gather*}
% Finally, we have
% \begin{gather*}
 % \left\|f(x+\delta x) - f(x) - \delta f (x,\delta x)\right\| = \left\|\int\limits_{t_0}^T
 % \Big( g_x\big(\overline{x}(t),t\big)- g_x\big(x(t),t\big)\Big)\delta x(t)\Big) \ dt\right\| \leqslant \\ \leqslant
 % \int\limits_{t_0}^T \left\|
 % \Big( g_x\big(\overline{x}(t),t\big)- g_x\big(x(t),t\big)\Big)\delta x(t)
 % \right\| \ dt \leqslant 
 % L_g \|\delta x(\cdot)\|_{\mathbb{L}_2}^2.
 % \end{gather*} 
% The result follows.
% \hfill$\square$\\[1ex]%--- P r o o f.

Введем отображение $\overline{F}: [t_0,T] \times [0,\overline{\varepsilon}] \times B_{\mathbb{L}_2}(0,\overline{\mu}) \to \mathbb{R}^n$, равенством $\overline{F}(\tau,\varepsilon, u(\cdot)) = x \big(\tau,\varepsilon, u(\cdot)\big) $, где $x \big(\tau,\varepsilon, u(\cdot)\big)$ --- это решение системы \eqref{sec3:nonlinear} в момент $\tau$ порожденное управлением $u(\cdot)$ и малым параметром $\varepsilon$. 
Производная Фреше этого отображения $\overline{F}$ по $u(\cdot)$, $\overline{F}': B_{\mathbb{L}_2}(0,\overline{\mu}) \to \mathbb{R}^n $ --- это решение линеаризованной системы, см., например \cite{GusZyk}
\begin{gather}\label{dF}
 \overline{F}'(\tau,\varepsilon, u(\cdot)) \delta u(\cdot) = \delta x(\tau), 
\end{gather}
где $\delta x(\tau)$ --- это решение системы \eqref{sec3:nonlinear}, линеаризованной вдоль $(u(\cdot),x(\cdot,\varepsilon, u(\cdot))$ и отвечающее управлению $\delta u(\cdot)$ и нулевым начальным условиям:
\begin{gather}\label{dx}
 \delta\dot{x} = \overline{A}\big(t,\varepsilon,u(\cdot)\big) \delta x + B(t)\delta u(t), \qquad 0\leqslant t \leqslant \tau, \qquad \delta x(0) = 0,
\end{gather}
где
\begin{gather*}
 \overline{A}\big(t,\varepsilon,u(\cdot)\big) = A(t) +\varepsilon \frac{\partial f\big(x(t,\varepsilon,u(\cdot)\big),t\big)}{\partial x}.
\end{gather*}
Применяя Лемму \ref{s1:lem:lip_dx_global} к системе \eqref{sec3:nonlinear} получим, что существует $L_u(\varepsilon) > 0$ такая, что для всех $\varepsilon\in [0,\overline{\varepsilon}]$, $u_i(\cdot) \in B_{\mathbb{L}_2}(0,\mu)$ и $\tau \in [t_0,T]$, 
\begin{gather*}
	\| \overline{F}'(\tau,\varepsilon, u_1(\cdot)) - \overline{F}'(\tau,\varepsilon, u_2(\cdot)) \| \leqslant L_u(\varepsilon) \| u_1(\cdot) - u_2(\cdot) \|_{\mathbb{L}_2},
\end{gather*}
где $i = 1,2$.

Теперь перейдем к определению дифференцируемости Фреше отображения $A_1(u(\cdot),\varepsilon)$ по $u(\cdot)$.
Выберем произвольную $u(\cdot) \in B_{\mathbb{L}_2}(0,\mu)$ и $\delta u(\cdot)$, так чтобы $\|\delta u(\cdot)\|_{\mathbb{L}_2} \leqslant \overline{\mu}-\mu$ и рассмотрим разность
\begin{gather}\label{diff_A}
 A_1(u(\cdot) + \delta u(\cdot),\varepsilon) - A_1(u(\cdot),\varepsilon) = \int\limits_{t_0}^T X(T,\tau) \left[ 
 f\Big(x\big(\tau,\varepsilon, u(\cdot) + \delta u(\cdot)\big),\tau\Big) - 
 f\Big(x\big(\tau,\varepsilon, u(\cdot)\big),\tau\Big) \right]\ d\tau.
\end{gather}

Изучим подробнее разность между решениями \eqref{sec3:nonlinear}, порожденными $u(\cdot)$ и $u(\cdot) + \delta u(\cdot)$. 
Выпишем разность этих решений
\begin{gather}\label{diff_of_x}
 \begin{gathered}
 x\big(t,\varepsilon, u(\cdot) + \delta u(\cdot)\big) -
 x\big(t,\varepsilon, u(\cdot)\big) 
 = \int\limits_{t_0}^t A(\tau) \left[
 x\big(\tau,\varepsilon, u(\cdot) + \delta u(\cdot)\big) -
 x\big(\tau,\varepsilon, u(\cdot)\big) 
 \right]\ d\tau + \\ +
 \int\limits_{t_0}^t B(\tau) \delta u(\tau)\ d\tau +
 \varepsilon\int\limits_{t_0}^t \left[ 
 f\Big(x\big(\tau,\varepsilon, u(\cdot) + \delta u(\cdot)\big),\tau\Big) -
 f\Big(x\big(\tau,\varepsilon, u(\cdot)\big),\tau\Big)
 \right]\ d\tau.
 \end{gathered}
\end{gather}

Пусть $y \in \mathbb{R}^n$ и $h \in \mathbb{R}^n$ выбраны так, что выполнялись включения $y\in D$ и $y+h \in D$. 
Тогда, для всех $\tau \in [t_0,T]$, используя представление приращения функции через интеграл по параметру, мы имеем

\begin{gather*}
 f(y + h, \tau) - f(y,\tau) = 
 % \int\limits_{y}^{y+h} 
 % \frac{\partial f}{\partial x} \big(x, \tau\big) dx = 
 \left( \int\limits_{0}^{1} 
 \frac{\partial f}{\partial x} \big(y + \xi h, \tau\big)\ d\xi \right) h =
 \frac{\partial f}{\partial x} \big(y, \tau\big) h + \omega(y,h,\tau),
\end{gather*}
где 
\begin{gather*}
 \omega(y,h,\tau) = \left( \int\limits_{0}^{1} 
 \left[\frac{\partial f}{\partial x} \big(y + \xi h, \tau\big) -
 \frac{\partial f}{\partial x} \big(y, \tau\big) \right] \ d \xi \right) h.
\end{gather*}

Так как $D$ выпукло, $ y + \xi h \in D$ для всех $0 \leqslant \xi \leqslant 1$.
Значит, используя Предположения \ref{s1:as:right_hand_side_conditions_global} и \ref{s1:as:right_hand_side_diff_lip}, мы можем получить следующую оценку
\begin{gather*}
 \|\omega(y,h,\tau)\| \leqslant l_f \left( \int\limits_0^1 \left\| \xi h \right\| \ d\xi \right)h \leqslant \frac{l_f}{2} \|h\|^2.
\end{gather*}

Положив $y = x\big(\tau,\varepsilon, u(\cdot)\big)$ и $h = \Delta x\big(\tau, \varepsilon, \delta u(\cdot)\big) = x\big(\tau,\varepsilon, u(\cdot) + \delta u(\cdot)\big) - x\big(\tau,\varepsilon, u(\cdot)\big)$, для всех $\tau \in [t_0,T]$ мы получаем
\begin{gather}\label{mean-value}
 \begin{gathered}
 f\Big(x\big(\tau,\varepsilon, u(\cdot) + \delta u(\cdot)\big),\tau\Big) -
 f\Big(x\big(\tau,\varepsilon, u(\cdot)\big),\tau\Big) = \\ = 
 \frac{\partial f}{\partial x} \Big(x\big(\tau,\varepsilon, u(\cdot)\big), \tau\Big) 
 \Delta x\big(\tau, \varepsilon, \delta u(\cdot)\big) + 
 \omega\Big(x\big(\tau,\varepsilon, u(\cdot)\big),\Delta x\big(\tau, \varepsilon, \delta u(\cdot)\big),\tau\Big),
 \end{gathered}
\end{gather}
где (см. Лемму \ref{s1:lem:lip_of_solutions_global})
\begin{gather}\label{omega_est}
 \left\|\omega\Big(x\big(\tau,\varepsilon, u(\cdot)\big),\Delta x\big(\tau, \varepsilon, \delta u(\cdot)\big),\tau\Big)\right\| 
 \leqslant
 \frac{l_f}{2} \left\|\Delta x\big(\tau, \varepsilon, \delta u(\cdot)\big)\right\|^2 
 \leqslant
 \frac{l_f}{2} L_x^2(\overline{\varepsilon}) \|\delta u(\cdot)\|_{\mathbb{L}_2}^2.
\end{gather}
% where $\overline{x}(\tau) \in [x\big(\tau,\varepsilon, u(\cdot), x\big(\tau,\varepsilon, u(\cdot) + \delta u(\cdot)\big)]$ for all $\tau \in [t_0, T]$, so $\|\overline{x}(\tau) - x\big(\tau,\varepsilon, u(\cdot)\big)\| \leqslant \|x\big(\tau,\varepsilon, u(\cdot) + \delta u(\cdot)\big) - x\big(\tau,\varepsilon, u(\cdot)\big)\|$. 

Из \eqref{mean-value} следует, что $\omega\Big(x\big(\tau,\varepsilon, u(\cdot)\big),\Delta x\big(\tau, \varepsilon, \delta u(\cdot)\big),\cdot\big)$ измерима, как сумма измеримых функций.
Подставляя \eqref{mean-value} в \eqref{diff_of_x}, получаем
\begin{gather*}
 \Delta x\big(t, \varepsilon, \delta u(\cdot)\big)
 = \int\limits_{t_0}^t 
 \overline{A}\big(\tau,\varepsilon,u(\cdot)\big) 
 \Delta x\big(\tau, \varepsilon, \delta u(\cdot)\big)\ d\tau + 
 \int\limits_{t_0}^t B(\tau) \delta u(\tau)\ d\tau + \\ +
 \varepsilon\int\limits_{t_0}^t \omega\Big(x\big(\tau,\varepsilon, u(\cdot)\big),\Delta x\big(\tau, \varepsilon, \delta u(\cdot)\big),\tau\Big) d\tau = 
 \delta x(t) + \Omega(t,\varepsilon, \delta u(\cdot)),
\end{gather*}
где $\delta x(t)$ --- это решение системы \eqref{dx} и
\begin{gather*}
 \Omega(t,\varepsilon, \delta u(\cdot)) = \varepsilon\int\limits_{t_0}^t 
 \omega\Big(x\big(\tau,\varepsilon, u(\cdot)\big),\Delta x\big(\tau, \varepsilon, \delta u(\cdot)\big),\tau\Big) d\tau
\end{gather*}


Поскольку \eqref{omega_est} мы можем оценить $\Omega(t,\varepsilon, \delta u(\cdot)) $ сверху для всех $t \in [t_0, T]$
\begin{gather}
 \| \Omega(t,\varepsilon, \delta u(\cdot))\| \leqslant \frac{l_f}{2} \overline{\varepsilon} L_x^2(\overline{\varepsilon})(T-t_0)\|\delta u(\cdot)\|_{\mathbb{L}_2}^2.
\end{gather}

Теперь мы можем переписать \eqref{mean-value} в виде, 
\begin{gather*}
 f\Big(x\big(\tau,\varepsilon, u(\cdot) + \delta u(\cdot)\big),\tau\Big) -
 f\Big(x\big(\tau,\varepsilon, u(\cdot)\big),\tau\Big) = \\ =
 \frac{\partial f}{\partial x} \Big(x\big(\tau,\varepsilon, u(\cdot)\big), \tau\Big) \delta x(t) + 
 \frac{\partial f}{\partial x} \Big(x\big(\tau,\varepsilon, u(\cdot)\big), \tau\Big) \Omega(t,\varepsilon, \delta u(\cdot))
 + \\ + 
 \omega\Big(x\big(\tau,\varepsilon, u(\cdot)\big),\Delta x\big(\tau, \varepsilon, \delta u(\cdot)\big),\tau\Big).
\end{gather*}

А затем оценить норму остаточного члена сверху:
\begin{gather*}
 \left\| 
 \frac{\partial f}{\partial x} \Big(x\big(\tau,\varepsilon, u(\cdot)\big), \tau\Big) \Omega(t,\varepsilon, \delta u(\cdot)) 
 \right\| 
 \leqslant
 \frac{l_f}{2}
 \overline{\varepsilon} 
 L_x^2(\overline{\varepsilon})
 (T-t_0)
 \max_{\substack{x\in D \\ \tau \in [t_0,T]}} 
 \left\|\frac{\partial f}{\partial x} \Big(x, \tau\Big) \right\|
 \|\delta u(\cdot)\|_{\mathbb{L}_2}^2,
\end{gather*}
Это позволяет нам переписать \eqref{diff_A} в виде
\begin{gather}
 A_1(u(\cdot) + \delta u(\cdot),\varepsilon) - A_1(u(\cdot),\varepsilon) = \int\limits_{t_0}^T X(T,\tau) \frac{\partial f}{\partial x} \Big(x\big(\tau,\varepsilon, u(\cdot)\big), \tau\Big) \delta x(t) \ d\tau + o(\|\delta u(\cdot)\|^2).
\end{gather}
Отсюда следует, что производная Фреше $A_1'(u(\cdot),\varepsilon): B_{\mathbb{L}_2}(0,\overline{\mu}) \to \mathbb{R}^n $ существует и может быть определена равенством
\begin{gather}\label{A1_diff}
 A_1'(u(\cdot),\varepsilon)\delta u(\cdot) = \int\limits_{t_0}^T X(T,\tau) \frac{\partial f}{\partial x} \Big(x\big(\tau,\varepsilon, u(\cdot)\big), \tau\Big) \delta x(t) \ d\tau 
\end{gather}
% In order to apply the Lemma \ref{LeibnizRule} to $A_1(u(\cdot),\varepsilon)$, we first have to show that the integrand in formula \eqref{A1_def} has a Lipschitz continuous derivative.
% This derivative is equal to the product of $\frac{\partial f \Big(x\big(\tau,\varepsilon, u(\cdot)\big),\tau\Big)} {\partial x}$ and $\frac{\partial x \big(\tau,\varepsilon, u(\cdot)\big)}{\partial u}$.
Липшицевость $\delta x(\cdot)$ была установлена в Лемме \ref{s1:lem:lip_dx_global}. 
Производная $\frac{\partial f}{\partial x} \Big(x\big(\tau,\varepsilon, u(\cdot)\big),\tau\Big)$ является липшицевой как композиция липшицевых функций.
\begin{gather*}
 \left\| \frac{\partial f \Big(x\big(\tau,\varepsilon, u_1(\cdot)\big),\tau\Big)} {\partial x} - \frac{\partial f \Big(x\big(\tau,\varepsilon, u_2(\cdot)\big),\tau\Big)} {\partial x} \right\| \leqslant l_f \left\|x\big(\tau,\varepsilon, u_1(\cdot) \big) - x\big(\tau,\varepsilon, u_2(\cdot)\big) \right\| \leqslant \\ \leqslant l_f L_x(\varepsilon) \left\| u_1(\cdot) - u_2(\cdot) \right\|_{\mathbb{L}_2}, \quad \tau \in [t_0, T], \qquad u_1(\cdot), u_2(\cdot) \in B_{\mathbb{L}_2}(0,\mu).
\end{gather*}

Тогда подынтегральное выражение в \eqref{A1_diff} также удовлетворяет условию Липшица для всех $\varepsilon \in [0, \overline{\varepsilon}]$ и $\tau \in [t_0, T]$, 
\begin{gather*}
 \left\|
 \frac{\partial f}{\partial x} \Big(x\big(\tau,\varepsilon, u_1(\cdot)\big), \tau\Big)
 \overline{F}'(\tau,\varepsilon, u_1(\cdot))
 \delta u(\cdot) -
 \frac{\partial f}{\partial x} \Big(x\big(\tau,\varepsilon, u_2(\cdot)\big), \tau\Big)
 \overline{F}'(\tau,\varepsilon, u_2(\cdot))
 \delta u(\cdot) 
 \right\| \leqslant \\ \leqslant
 (\overline{\mu} - \mu)
 \left(
 l_f L_x(\varepsilon) \max_{\substack{u(\cdot) \in B_{\mathbb{L}_2}(0,\mu) \\ \tau \in [t_0,T]}} \|\overline{F}'(\tau,\varepsilon, u(\cdot)) \| + 
 L_u(\varepsilon) 
 \max_{\substack{x \in D \\ \tau \in [t_0,T]}} \left\|
 \frac{\partial f}{\partial x} \Big(x, \tau\Big)
 \right\|
 \right)
 \left\|
 u_1(\cdot) - u_2(\cdot)
 \right\|,
\end{gather*}
а значит, и вся производная $A_1'(u(\cdot),\varepsilon)$ также удовлетворяет условию Липшица по $u(\cdot)$. 

Для того, чтобы проверить выполнение условий Предположения \ref{as:derivative_of_A1}, остается показать, что эта производная будет непрерывна по $\varepsilon$. 
Это справедливо в силу того, что правая часть системы \eqref{sec3:nonlinear} линейна по параметру $\varepsilon$, а матрица $\overline{A}\big(t,\varepsilon,u(\cdot)\big)$ линеаризованной системы \eqref{dx} непрерывно зависит от $\varepsilon$.


Таким образом, отображение $A_1(u(\cdot),\varepsilon)$, определенное в \eqref{A1_def}, удовлетворяет условию предположения \ref{as:derivative_of_A1}, и мы можем сформулировать основной результат этой главы в следующей теореме.

\begin{theorem}\label{th:ReachableSetsConvexity}
 Пусть выполнены условия Предположений \ref{s1:as:right_hand_side_conditions_global} и \ref{s1:as:right_hand_side_diff_lip}, тогда существует такое положительное значение $\varepsilon_0$, что множество достижимости $G(T,\mu,\varepsilon) $ квазилинейной системы \eqref{sec3:nonlinear} выпукло для всех $\varepsilon < \varepsilon_0$. 
\end{theorem}
\doc. 
Справедливость этого утверждения можно обосновать, применив Теорему \ref{th:ImageConvexity} к отображению $F$, при условии, что липшицевость $A_1'$ и замкнутость $G(T,\mu,\varepsilon) $ (Утверждение \ref{ReachableSetcloseness}) были доказаны ранее.
\hfill$\square$\\[1ex]%--- P r o o f.

\begin{zam} 
 В статье \cite{Albrecht2} исследованы опорные функции множеств достижимости квазилинейных систем с интегральными ограничениями и доказана их непрерывная зависимость от малого параметра.
 Это влечет непрерывную зависимость множеств достижимости от малого параметра. 
 Автор также отметил, что из непрерывной зависимости множеств достижимости от параметра вытекает его выпуклость при малых значениях малого параметра, однако никаких доказательств этого факта не приведено. 
 Более того, непрерывности множества достижимости недостаточно для доказательства его выпуклости.
\end{zam}

\subsection{Примеры}

В этом разделе представлены результаты численных экспериментов, которые призваны проиллюстрировать применение теорем \ref{th:ImageConvexity} и \ref{th:ReachableSetsConvexity}. 

\textbf{Пример 1.}
 Первая исследуемая система --- осциллятор Дуффинга. 
 Мы работаем с уравнениями
 \begin{gather}\label{sec3:Duffing}
 \dot{x_1} = x_2, \qquad
 \dot{x_2} = -x_1 - 10 \varepsilon x_1^3 + u,\qquad 0\leqslant t \leqslant 2
 \end{gather}
 описывающими движение нелинейной упругой пружины под действием внешней силы $u$. 
 Влияние нелинейного слагаемого силы упругости определяется малым параметром $\varepsilon > 0$. 
Начальное состояние $x_1(0) = x_2(0) = 0 $, а управление ограничено 
 \begin{gather}\label{Duffing_controls}
 \int\limits_0^2u^2dt \leqslant 1.
 \end{gather}
 При $\varepsilon = 0$, уравнения \eqref{sec3:Duffing} описывают линейную систему с матрицами 
 \begin{gather*}
 A = \begin{pmatrix} 0 & 1\\
 -1 & 0
 \end{pmatrix}, \quad B = \begin{pmatrix}
 0\\
 1
 \end{pmatrix}.
 \end{gather*}
 
 Нелинейное слагаемое состоит из малого параметра и функции $f(x) = [0;-10x_1^3]$. 
 Для такой нелинейной функции условие \eqref{sec3:sublinear_growth} не выполняется. 
 Однако, мы можем использовать оценки, полученные в работе \cite{Zykov2019} для того, чтобы показать, что все траектории системы \eqref{sec3:Duffing}, соответствующие допустимым управлениям и нулевым начальным условиям, лежат в компактном множестве D. 
 
 Положим $v_{\varepsilon}(t,x) = \frac{5}{2}\varepsilon x_1^4 + \frac{1}{2}x_1^2 + \frac{1}{2}x_2^2$ и вычислим производную в силу системы 
 \begin{gather}\label{Lyap}
 \frac{d}{dt} v_{\varepsilon}(t,x(t)) = \nabla v_{\varepsilon}(t,x(t)) \big(A x(t) + B u(t) + \varepsilon f(x(t))\big) = x_2(t) u(t). 
 \end{gather} 
 
 Для каждого $\varepsilon \geqslant 0$ и каждого управления $u(\cdot)$ удовлетворяющего \eqref{Duffing_controls}, существует $\tau>0$, такое, что решение \eqref{sec3:Duffing}, порожденное этим управлением $u(\cdot)$ и нулевыми начальными условиями, определено на некотором интервале $[0, \tau]$. 
 Проинтегрируем \eqref{Lyap} от $0$ до $\tau$. 
 Имеем 
 \begin{gather*}
 v_{\varepsilon}(\tau,x(\tau)) =
 \int\limits_0^{\tau} x_2(t) u(t) \ dt 
 \leqslant 
 \left(\int\limits_0^2 u^2(t) \ dt \right)^{\frac{1}{2}} \left(\int\limits_0^{\tau} x_2^2(t) \ dt \right)^{\frac{1}{2}} \leqslant \sqrt{2} \left(\int\limits_0^{\tau} v_{\varepsilon}(t,x(t)) \ dt \right)^{\frac{1}{2}}.
 \end{gather*}
 
 Применяя теорему сравнения к этому неравенству, можно получить, что $v_{\varepsilon}(\tau,x(\tau)) \leqslant \tau$ и, следовательно, $\|x(\tau)\|^2 \leqslant 2 \tau$. 
 Используя известную технику, можно сделать вывод, что любое решение \eqref{sec3:Duffing}, порожденное управлением $u(\cdot) \in B_{\mathbb{L}_2}(0,1)$ и нулевым начальным состоянием, может быть продолжено на временной интервал $[0,2]$ и принадлежит выпуклому множеству $D = B_{\mathbb{R}^n}(0,2)$.
 
 Условия Предположений \ref{s1:as:right_hand_side_conditions_global} и \ref{s1:as:right_hand_side_diff_lip} выполнены: Пара $(A,B)$ является константой; функция $f$ непрерывна и непрерывно дифференцируема; также функция $f$ и ее производная $\partial f/\partial x$ удовлетворяют условию Липшица на множестве D.
 
 \begin{figure}[t]
 \centerline{
 \includegraphics[width=\textwidth]{images/Osipov_QuaziDuffing.eps}}
 \caption{Множества достижимости системы \eqref{sec3:Duffing}.}
 \label{fig:Duffing}
 \end{figure}
 
 Таким образом, для системы \eqref{sec3:Duffing} выполняются требования Теоремы \ref{th:ReachableSetsConvexity}, и соответствующие множества достижимости должны быть выпуклыми для малых значений параметра. 
 Это видно на рисунке \ref{fig:Duffing}, демонстрирующем множества достижимости $G^{\varepsilon}(T,\mu)$, построенные с помощью численного метода Монте-Карло \cite{Patent,Zykov}.
 
 Видно, что множества $G^{0.01}(1,1) $ и $G^{0.1}(1,1) $ близки к множеству $G^{0}(1,1) $, которое построено для линейной системы. 
 Также видно, что при увеличении параметра $\varepsilon$ множества становятся невыпуклыми. 
 
\begin{figure}[ht]
 \centerline{
 \includegraphics[width=\textwidth]{images/Osipov_QuaziDubins.eps}}
 \caption{Множества достижимости системы \eqref{Linear+Dubins}.}
 \label{fig:LinearDubins}
\end{figure}
\textbf{Пример 2.}
 Второй исследуемой системой является
 \begin{gather}\label{Linear+Dubins}
 \begin{pmatrix} 
 \dot{x}_1 \\
 \dot{x}_2 \\ 
 \dot{x}_3 \end{pmatrix} = 
 \begin{pmatrix}
 0 & 1 & 0 \\
 0 & 0 & 1 \\
 0 & 0 & 0
 \end{pmatrix}
 \begin{pmatrix} 
 x_1 \\
 x_2 \\ 
 x_3 \end{pmatrix} + 
 \varepsilon
 \begin{pmatrix}
 \cos x_3 - x_2\\
 \sin x_3 - x_3 \\
 0
 \end{pmatrix} + 
 \begin{pmatrix}
 0 \\ 0 \\ 1
 \end{pmatrix} u.
 \end{gather}
 При $\varepsilon = 0$, уравнения \eqref{Linear+Dubins} описывают линейную систему с матрицами 
 \begin{gather*}
 A = \begin{pmatrix} 0 & 1 & 0\\
 0 & 0 & 1\\ 
 0 & 0 & 1 
 \end{pmatrix}, \quad B = \begin{pmatrix}
 0\\
 0\\
 1
 \end{pmatrix},
 \end{gather*}
 а при $\varepsilon = 1$, они описывают уницикл. 
 Нелинейное слагаемое состоит из малого параметра и функции
 \begin{gather*}
 f(x) = \begin{pmatrix}
 \cos x_3 - x_2\\
 \sin x_3 - x_3 \\
 0
 \end{pmatrix}.
 \end{gather*}
 
 Начальное состояние --- нулевое, $x_1(0) = x_2(0) = x_3(0) = 0$, ограничения на управление такие же, как и в первом примере, но мы будем рассматривать эту систему на временном интервале $0 \leqslant t \leqslant 1$.
 
 Как и в предыдущем примере, условия предположений \ref{s1:as:right_hand_side_conditions_global} и \ref{s1:as:right_hand_side_diff_lip} выполнены, что позволяет применить Теорему \ref{th:ReachableSetsConvexity}. 
 На рисунке \ref{fig:LinearDubins} показаны проекции на плоскость $(x_1,x_2)$ численно построенных множеств достижимости $G^{\varepsilon}(T,\mu)$ для системы \eqref{Linear+Dubins}. 
 
 Видно, что проекции множеств $G^{0.001}(1,1) $ и $G^{0.01}(1,1) $ близки к проекции множества $G^{0}(1,1) $, построенной для линейной системы. 
 Также видно, что проекции множеств становятся невыпуклыми при увеличении параметра $\varepsilon$. 



\end{document}