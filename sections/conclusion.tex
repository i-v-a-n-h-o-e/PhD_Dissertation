\documentclass[../main.tex]{subfiles}
\begin{document}
\clearpage
\section*{Заключение}
 \addcontentsline{toc}{section}{Заключение}
В диссертации получены следующие основные результаты
\begin{enumerate}
	\item Для аффинных по управлению систем с интегрально-квадратичными ограничениями на управление исследована выпуклость множеств достижимости на малом интервале времени.
	Получены достаточные условия выпуклости множеств достижимости в виде ограничений на асимптотику собственных чисел соответствующего грамиана управляемости линеаризованной системы.
	Разработан метод проверки этих условий, основанный на рекуррентной процедуре вычисления коэффициентов разложения грамиана управляемости в ряд по степеням малого параметра.
	Доказано, что нелинейные системы второго порядка, линеаризация которых приводит к линейным стационарным вполне управляемым системам, обладают выпуклыми множествами достижимости на малых интервалах времени.
	
	\item Для выпуклых компактов в $\mathbb{R}^n$, зависящих от малого параметра, введено понятие асимптотической эквивалентности, основанное на расстоянии Банаха-Мазура.
	Получены достаточные условия асимптотической эквивалентности множеств достижимости по выходу нелинейных систем с интегрально-квадратичными ограничениями на управление и соответствующих множеств линеаризованных систем. 
	Эти условия совпадают с достаточным условием выпуклости множеств достижимости на малых интервалах времени.
	
	\item Исследована задача синтеза управления для нелинейной аффинной по управлению системы с интегрально-квадратичным функционалом. 
	Доказано, что линейная обратная связь, построенная для линеаризованной системы, также обеспечивает локальное решение задачи синтеза для нелинейной системы на достаточно малом промежутке времени. 
	Это требует ограничений на асимптотику грамиана управляемости, которые совпадают с достаточными условиями, обеспечивающими асимптотическую эквивалентность множеств достижимости (множеств нуль-управляемости). 
	В этом случае, линейная обратная связь, решающая задачу приведения линеаризованной системы в начало координат за заданное время, будет приводить в начало координат и нелинейную систему из малой окрестности нуля.
	Также, при этих условиях получена оценка для относительных значений погрешности интегрального функционала. 
	
	\item Для квазилинейных систем с интегральными квадратичными ограничениями на управление исследована выпуклость множеств достижимости. 
	Опираясь на достаточные условия выпуклости образа гильбертова шара при его квазилинейном отображении, доказано, что множества достижимости квазилинейной системы остаются выпуклыми при малых значениях малого параметра. 
\end{enumerate}

Теоретическая значимость проведенного исследования состоит в развитии методов анализа множеств достижимости нелинейных управляемых систем с интегральными ограничениями. 
Впервые получены конструктивные условия выпуклости множеств достижимости, выраженные через асимптотические свойства грамиана управляемости линеаризованной системы. 
Это позволяет применять развитые конструкции выпуклого анализа для решения задач управления в существенно нелинейных системах.

Практическая ценность результатов заключается в возможности их использования при разработке численных алгоритмов решения задач оптимального управления и синтеза управляющих воздействий. 
Установленные условия применимости метода линеаризации дают строгое теоретическое обоснование для широко используемых инженерных подходов к синтезу управления.

Перспективы дальнейших исследований включают обобщение полученных результатов на случай $\mathbb{L}_p$-ограничений на управление, а также разработку эффективных численных методов построения множеств достижимости на основе установленных свойств выпуклости.

Полученные в диссертации результаты представляют научный и методологический интерес, строго доказаны и проиллюстрированы содержательными примерами.


\end{document}