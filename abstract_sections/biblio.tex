\documentclass[../main.tex]{subfiles}
\begin{document}
\begin{thebibliography}{99}

\section*{Публикации автора по теме диссертации}
\addcontentsline{toc}{subsection}{Публикации автора по теме диссертации}
\subsubsection*{Публикации в изданиях, рекомендованных ВАК}
% 2019
\bibitem{GusevOsipovTrudy}
Гусев~М.\,И., Осипов~И.\,О. Асимптотическое поведение множеств достижимости на малых временных промежутках~// Тр. Ин-та математики и механики УрО РАН. --- 2019. --- Т.~25, №~3. --- С.~86--99.
\hrefdoi{10.21538/0134-4889-2019-25-3-86-99} \\
ВАК К1; WoS (ESCI), Scopus, MathSciNet, RSCI
\\ Переводная версия: \\
Gusev~M.\,I., Osipov~I.\,O. Asymptotic behavior of reachable sets on small time intervals~// Proc. Steklov Inst. Math. --- 2020. --- Vol.~309, Suppl.~1. --- P.~S52--S64. \hrefdoi{10.1134/S0081543820040070}\\
WoS(SCIE), Scopus, MathSciNet, Springer.

%2021
\bibitem{Osipov}
Осипов~И.\,О. О выпуклости множеств достижимости по части координат нелинейных управляемых систем на малых промежутках времени~// Вестник Удмуртского университета. Математика. Механика. Компьютерные науки. --- 2021. --- Т.~31, вып.~2. --- C.~210--225.
\hrefdoi{10.35634/vm210204} \\
 ВАК; К1: WoS (ESCI), Scopus, MathSciNet, zbMATH, RSCI.

%2021 
\bibitem{Voronezh}
Осипов~И.\,О. Об асимптотике собственных чисел грамиана управляемости линейной системы с малым параметром~// Итоги науки и техники. Сер. Соврем. математика и ее прил. Темат. обзоры. --- 2021. --- Т.~191. --- С.~115--122.
\hrefdoi{10.36535/0233-6723-2021-191-115-122}
\\ ВАК; К1: MathSciNet
\\Переводная версия: \\
Osipov~I.\,O. On the Asymptotics of Eigenvalues of the Controllability Gramian of a Linear System with a Small Parameter~// J. Math. Sci. --- 2025. --- Vol.~288. --- P.~780--787. 
\hrefdoi{10.1007/s10958-025-07767-4} \\
К1: Scopus, zbMATH, Springer

%2022
\bibitem{GusevOsipov} 
Гусев~М.\,И., Осипов~И.\,О. О задаче локального синтеза для нелинейных систем с интегральными ограничениями~// Вестник Удмуртского университета. Математика. Механика. Компьютерные науки. --- 2022. --- Т.~32, №~2. --- С.~171--186. 
\hrefdoi{10.35634/vm220202} \\
 ВАК; К1: WoS (ESCI), Scopus, MathSciNet, zbMATH, RSCI.


%2023
\bibitem{OsipovUMJ}
Osipov~I.\,O. Convexity of Reachable Sets of Quasilinear Systems~// Ural Math. J. --- 2023. --- Vol.~9, no.~2. --- P.~141--156.
\hrefdoi{10.15826/umj.2023.2.012} \\
ВАК; К1: Scopus, MathSciNet, zbMATH, RSCI


\subsubsection*{Другие публикации, включенные в международные базы данных}
%2019
\bibitem{AIP}
Gusev~M.\,I., Osipov~I.\,O. On Convexity of Small-time Reachable Sets of Nonlinear Control Systems~// AIP Conference Proceedings. --- 2019. --- Vol.~2164. --- Art.~no.~060007.
\hrefdoi{10.1063/1.5130809} \\
WoS, Scopus,

%2022
\bibitem{GusevOsipovPyat}
Gusev~M., Osipov~I. On Local Control Synthesis for Nonlinear Systems under Integral Constraints~// 2022 16th International Conference on Stability and Oscillations of Nonlinear Control Systems (Pyatnitskiy's Conference), Moscow, 2022. --- P.~1--4. \hrefdoi{10.1109/STAB54858.2022.9807463}.\\
Scopus

%2023
\bibitem{GusevOsipovMotor}
Gusev~M., Osipov~I. Approximate Solution of Small-Time Control Synthesis Problem Based on Linearization~// Mathematical Optimization Theory and Operations Research. MOTOR 2023. --- Cham~: Springer, 2023. --- (Lecture Notes in Computer Science~; vol.~13930). --- P.~362--377.
\hrefdoi{10.1007/978-3-031-35305-5\_25}
zbMATH, MathSciNet, Springer

\subsubsection*{Свидетельства о государственной регистрации программ для ЭВМ}
\bibitem{Patent} 
Программа для построения методом Монте-Карло множеств достижимости нелинейных систем с интегральными ограничениями на управление~: свидетельство о~гос. регистрации программы для ЭВМ №~2020661557 ~/~И.\,В.~Зыков, И.\,О.~Осипов~; правообладатель ИММ УрО РАН // Федеральная служба по интеллектуальной собственности (РосПатент) --- Зарег. 24.09.2020. --- URL: https://www.elibrary.ru/item.asp?id=44104691 (дата обращения: 07.09.2025).

\subsubsection*{Материалы научных конференций}

\bibitem{OsipovVoronezhAbstract}
Осипов~И.\,О. Об асимптотике собственных чисел грамиана управляемости линейной системы с малым параметром~// Современные методы в теории краевых задач~: материалы Воронеж. весен. мат. шк. «Понтрягинские чтения – XXX», Воронеж, 3--9 мая 2019~г. --- Воронеж~: Изд. дом ВГУ, 2019. --- С.~215--216.

\bibitem{OsipovAIPAbstract}
Gusev~M.\,I., Osipov~I.\,O. On Convexity of Small-time Reachable Sets of Nonlinear Control Systems~// Application of Mathematics in Technical and Natural Sciences (AMiTaNS'11): 11th Intern. Conf., Albena, Bulgaria, June 20-25, 2019: abstracts. --- Sofia, 2019. --- P.~30.

% 2020
\bibitem{SubbotinConf}
Гусев~М.\,И., Осипов~И.\,О. Об асимптотике множеств достижимости на малых временных промежутках~// Теория управления и теория обобщенных решений уравнений Гамильтона-Якоби~: материалы III междунар. семинара, посвящ. 75-летию акад. А.\,И.~Субботина, Екатеринбург, 26--30 окт. 2020~г. --- Екатеринбург~: ИММ УрО РАН, 2020. --- С.~142--145.


\bibitem{Minskconf}
Гусев~М.\,И., Осипов~И.\,О. Асимптотика множеств достижимости нелинейных управляемых систем на малых промежутках времени~// Динамические системы: устойчивость, управление, оптимизация~: материалы Междунар. науч. конф. памяти Р.\,Ф.~Габасова, Минск, 5--10 окт. 2021~г. --- Минск~: Изд. центр БГУ, 2021. --- С.~85--87. 

\bibitem{GusevOsipovOCTA}
Gusev~M.\,I., Osipov~I.\,O. Asymptotics of Small-Time Reachable Sets and a Problem Of Local Control Synthesis~// Теория оптимального управления и приложения (OCTA 2022)~: материалы Междунар. конф., Екатеринбург, 27 июня–1 июля 2022~г. --- С.~300--303.

\bibitem{GusevOsipovPyatAbstract}
Гусев~М.\,И., Осипов~И.\,О. О локальном синтезе управления для нелинейных систем с интегральными ограничениями~// Устойчивость и колебания нелинейных систем управления (конференция Пятницкого)~: материалы XVI Междунар. конф., Москва, 1--3 июня 2022~г. --- М.~: ИПУ РАН, 2022. --- С.~164--167.

\bibitem{OsipovNLA}
Osipov~I.\,O. On the Linearization Method in Small-time Control Synthesis~// Nonlinear Analysis and Extremal Problems (NLA-2022): 7th Intern. conf., Irkutsk, Russia, July 15–22, 2022: proceedings. --- Irkutsk~: ISDCT SB RAS, 2022. --- P.~83--85.

\bibitem{OsipovSopromat2022}
Осипов~И.\,О. О применимости метода линеаризации в задаче локального синтеза на малом интервале времени~// Современные проблемы математики и~её приложений~: материалы 53-й Междунар. молодёжной шк.-конф., Екатеринбург, 31 янв. --- 4 февр. 2022~г. --- Екатеринбург, 2022. --- С.~99--101.

\bibitem{OsipovSopromat2023}
Осипов~И.\,О. Об асимптотике функционала при решении задачи локального синтеза методом линеаризации~// Современные проблемы математики и~её приложений~: материалы 54-й Междунар. молодёжной шк.-конф., Екатеринбург, 6--10, 17 февр. 2023~г. --- Екатеринбург, 2023. --- С.~100--101.

%2024
\bibitem{OsipovVSPU2024}
Осипов~И.\,О. Выпуклость множеств достижимости квазилинейных систем~// XIV Всероссийское совещание по проблемам управления (ВСПУ-2024)~: сб. науч. тр., Москва, 17--20 июня 2024~г. --- М.~: ИПУ РАН, 2024. --- C.~150--155.

\bibitem{OsipovSopromat2024}
Осипов~И.\,О. О свойстве выпуклости множеств достижимости квазилинейных систем~// Динамические системы: устойчивость, управление, дифференциальные игры (SCDG2024)~: тр. Междунар. конф., посвящ. 100-летию со дня рождения акад. Н.\,Н.~Красовского, Екатеринбург, 9--13 сент. 2024~г. --- Екатеринбург, 2024. --- С.~234--238. 


% \bibitem{Chi}
% Chi, R., Hui, Y., Huang, B., Hou, Z.: Adjacent-Agent Dynamic Linearization-Based Iterative Learning Formation Control. IEEE Transactions on Cybernetics, \textbf{50}(10), 4358-4369 (2019). \doi{10.1109/tcyb.2019.2899654} 

% \bibitem{Zhong}
% Zhong, Z., del Rio-Chanona, E.A., Petsagkourakis, P.: Tube-based distributionally robust model predictive control for nonlinear process systems via linearization. Computers \& Chemical Engineering, \textbf{170}, (2023) \doi{10.1016/j.compchemeng.2022.108112}.

% \bibitem{Gusev2024}
% М. И. Гусев, “О некоторых свойствах множеств достижимости нелинейных систем с ограничениями на управление в Lp”, Тр. ИММ УрО РАН, 30, № 3, 2024, 99–112; Proc. Steklov Inst. Math., 327, suppl. 1 (2024), S124–S137

%\bibitem{ref_proc1}
%Author, A.-B.: Contribution title. In: 9th International Proceedings
%on Proceedings, pp. 1--2. Publisher, Location (2010)

%\bibitem{ref_article1}
%Author, F.: Article title. Journal \textbf{2}(5), 99--110 (2016)

%\bibitem{ref_lncs1}
%Author, F., Author, S.: Title of a proceedings paper. In: Editor,
%F., Editor, S. (eds.) CONFERENCE 2016, LNCS, vol. 9999, pp. 1--13.
%Springer, Heidelberg (2016). \doi{10.10007/1234567890}


%\bibitem{Zyk}
%Zykov I.V. On external estimates of reachable sets of control systems with integral constraints, {\it Izvestiya Instituta Matematiki i Informatiki Udmurtskogo Gosudarstvennogo Universiteta}, 2019. vol. 53. pp. 61--72. (in Russian). 
%doi: 10.20537/2226-3594-2019-53-06
%
%\bibitem{GusZyk}
%Gusev, M.I., Zykov, I.V.: On Extremal Properties of the Boundary Points of Reachable Sets for Control Systems with Integral Constraints. In: Sergeev, A.G. (eds.), Proc. Steklov Inst. Math. 300, pp. 114--125. Springer (2018). doi: 10.1134/S0081543818020116

%\bibitem{Atans}
%Atans, M., Falb, P.: Optimal control. 
%Dover Publications, New York, McGraw-Hill (1966)

%\bibitem{Perv}
%Pervozvanskii A.A., {\it Kurs teorii avtomaticheskogo upravleniya} (Course in automatic control theory). Moscow: Nauka, 1986, 615 p.

%\bibitem{Gus1}
%Gusev, M.I.: On Convexity of Reachable Sets of a Nonlinear System under Integral Constraints. IFAC-PapersOnLine \textbf{51}(32), 207--212 (2018). \doi{10.1016/j.ifacol.2018.11.382}


%\bibitem{Krener}
%Krener, A., Sch\"{a}ttler, H.: The structure of small-time reachable sets in low dimensions. SIAM Journal on Control and Optimization %\textbf{27}(1) 120--147 (1989). \doi{10.1137/0327008}

%\bibitem{Schattler}
%Sch\"{a}ttler, H.: Small-time reachable sets and time-optimal feedback control. In: Mordukhovich, B.S.,
%Sussmann, H.J. (eds.) Nonsmooth Analysis and Geometric Methods in Deterministic Optimal Control.
%The IMA Volumes in Mathematics and Its Applications. Springer, New York. 1996. Vol. 78. p.~203--225. \doi{10.1007/978-1-4613-8489-2\_9}

% \bibitem{Watson}
% Watson, J., Abdulsamad, H., Peters, J.: Stochastic Optimal Control as Approximate Input Inference. In: Kaelbling, L. P., Kragic, D., Sugiura, K. (eds.) Proceedings of the Conference on Robot Learning. Proceedings of Machine Learning Research. \textbf{100}, 697--716 PMLR (2020) 
% \doi{10.48550/arXiv.1910.03003}

% \bibitem{Cao}
% Cao, R., Jiang, N., Lu, M.: Sensorless Control of Linear Flux-Switching Permanent Magnet Motor Based on Extended Kalman Filter. In: IEEE Transactions on Industrial Electronics, \textbf{67}(7), 5971--5979, (2020) \doi{10.1109/TIE.2019.2950865}.



%\bibitem{ref_book1}
%Author, F., Author, S., Author, T.: Book title. 2nd edn. Publisher,
%Location (1999)



%\bibitem{ref_url1}
%LNCS Homepage, \url{http://www.springer.com/lncs}. Last accessed 4
%Oct 2017
\end{thebibliography}
\end{document}