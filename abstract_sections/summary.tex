\documentclass[../abstract.tex]{subfiles}
\begin{document}
\newpage
\section*{Cодержание работы}
\textbf{Введение.} 
В этом разделе обосновывается актуальность темы исследования, приводится краткий обзор работ по данной тематике, определяются цель и задачи работы и формулируются основные положения, выносимые на защиту.


\textbf{Глава 1.}
Первая глава посвящена множествам достижимости нелинейных систем с интегральными ограничениями.  
Здесь даются основные понятия, описываются предположения и свойства решений исследуемых систем, которые используются в работе. 
Вводится отображение, действующее из пространства управлений в пространство концов траекторий нелинейной системы. 
Приводятся условия выпуклости образа малого шара в гильбертовом пространстве при его нелинейном отображении, полученные в \cite{Polyak2001}. 
Использование этого условия позволило получить условие выпуклости множеств достижимости при малых интегральных ограничениях на управление \cite{Polyak2004}.
Множество достижимости нелинейной системы рассматривается как образ шара, соответствующего допустимым управлениям, при его нелинейном отображении, заданном решениями системы.
При помощи замены времени удается сформулировать условия выпуклости множеств достижимости рассматриваемых нелинейных систем на малых интервалах времени. 
Проверка этих условий требует изучения асимптотики наименьшего собственного числа грамиана управляемости линеаризованной системы.
Один из возможных способов проверки асимптотики собственных чисел грамиана предлагается в последнем разделе этой главы. 
Там же приводится доказательство выпуклости множеств достижимости на малых интервалах времени для некоторых классов нелинейных систем второго порядка. 

\textbf{Глава 2.}
Вторая глава состоит из двух разделов, связанных общим понятием --- асимптотической эквивалентностью множеств достижимости. 

В первом разделе этой главы исследуются множества достижимости по выходу нелинейных систем на малых интервалах времени.

Рассмотрим выпуклые компактные множества $ X,Y \subset \mathbb R^n $, такие, что $0 \in \operatorname{int}\ X$, $0 \in \operatorname{int}\ Y$. 
Тогда определена величина $r(X, Y) = \inf \{t \geqslant 1: tX \supset Y \}$. 
Расстоянием Банаха-Мазура будем называть величину $ \rho (X, Y) $, определенную равенством 
\begin{gather*}
	\rho (X, Y): = \log (r(X,Y) \cdot r(Y, X)).
\end{gather*}

Пусть $ X = X(\varepsilon) $, $ Y = Y(\varepsilon) $ --- выпуклые компактные множества, такие, что $ 0 \in \operatorname{int}\,X(\varepsilon) $, $ 0 \in \operatorname{int}\,Y(\varepsilon) $ при $0 \leqslant \varepsilon \leqslant \overline{\varepsilon} $.
Тогда, следуя \cite{Ovs}, назовем множества $ X(\varepsilon) $ и $ Y(\varepsilon) $ {\textit асимптотически эквивалентными}, если $ \rho (X(\varepsilon), Y(\varepsilon)) \rightarrow 0 $ при $\varepsilon \rightarrow 0 $.

Основной результат этого раздела сформулирован для нелинейной системы на малом интервале времени $\ t_0 \leqslant t \leqslant t_0 + \overline{\varepsilon} $.
\begin{gather}\label{s2:nonlinear_with_output}
	\begin{gathered}
		\dot{x}(t)=f_1(t,x(t))+f_2(t,x(t))u(t), \qquad x(t_0) = x_0, \\
		y(t) = C x(t).
	\end{gathered}
\end{gather}
Здесь $ x \in \mathbb{R}^n $ --- вектор состояния, $ u \in \mathbb{R}^r $ --- управление, $ y\in\mathbb{R}^m (m \leqslant n) $ --- выход системы,
$ C\in \mathbb{R}^{m \times n} $ --- матрица полного ранга, $ \overline{\varepsilon} $ --- некоторое фиксированное положительное число.

Функции $ f_1: [t_0, t_0 + \overline{\varepsilon}] \times \mathbb{R}^{n} \rightarrow \mathbb{R}^{n} $, $ f_2: [t_0, t_0 + \overline{\varepsilon}] \times \mathbb{R}^{n} \rightarrow \mathbb{R}^{n \times r} $ предполагаются непрерывными по $(x, t)$ и обладающими непрерывными производными по $x$ на $ [t_0, t_0 + \overline{\varepsilon}]$.

\begin{assumption}\label{s1:as:right_hand_side_conditions_global}
	Существует такое $\overline{\mu} > 0 $, что все решения (траектории) $ x(t, u(\cdot)) $ системы \eqref{s2:nonlinear_with_output}, отвечающие управлениям $u(\cdot) \in B_{\mathbb{L}_2[t_0, t_0 + \overline{\varepsilon}]}(0,\overline{\mu})$, определены на интервале $ [t_0,t_0 + \overline{\varepsilon}] $ и лежат в некотором выпуклом компакте $D \subset \Omega \subset \mathbb{R}^n$. 
\end{assumption}

Управление $u(\cdot)$ будем выбирать из пространства интегрируемых с квадратом функций $\mathbb{L}_2[t_0,t_0+\overline{\varepsilon}]$ и ограничим шаром радиуса $ 0 < \mu < \overline{\mu} $ в этом пространстве, который будем обозначать через $B_{\mathbb{L}_2}(0, \mu)$
\begin{gather*}
	\lVert u(\cdot)\rVert^2_{\mathbb{L}_2} = \left(u(\cdot),u(\cdot) \right) \leqslant \mu^2.
\end{gather*}
Пусть $ 0 < \varepsilon \leqslant \overline{\varepsilon} $. 

\begin{assumption}\label{s1:as:right_hand_side_diff_lip}
	Функции $f_1(t,x)$ и $f_2(t,x)$ имеют непрерывные производные по $x$, которые удовлетворяют условию Липшица при всех $t \in [t_0;t_0 + \varepsilon]$, $x_1, x_2 \in D$.
	\begin{gather*}
		\left\| \frac{\partial f_1}{\partial x}(t,x_1) - \frac{\partial f_1}{\partial x}(t,x_2) \right\| \leqslant l_{f_1} \| x_1 - x_2\|, \\
		\left\| \frac{\partial f_2^k}{\partial x}(t,x_1) - \frac{\partial f_2^k}{\partial x}(t,x_2) \right\| \leqslant l_{f_2^k} \| x_1 - x_2\|,
	\end{gather*}
	где $l_{f_1} > 0$ и $l_{f_2^k} > 0$, $k = 1,...,r$.
\end{assumption}

\begin{definition}
	{\textit Множеством достижимости $\overline{G}(\varepsilon)$ системы \eqref{s2:nonlinear_with_output} по выходу} $ y = C x $ будем называть множество всех выходов $ y(t_0+\varepsilon) $,
	соответствующих концам траекторий $ x(t_0+\varepsilon, u(\cdot)) $, порождённых управлениями $ u(\cdot) \in B_{\mathbb{L}_2}(0,\mu)$
	\begin{gather*}
		\overline{G}(\varepsilon)=\{y\in \mathbb{R}^m:\exists u(\cdot)\in B_{\mathbb{L}_2}(0,\mu),\; y=Cx(t_0+\varepsilon, u(\cdot))\}.
	\end{gather*}
\end{definition}

\begin{definition}
	Симметричная матрица, определённая равенством
	\begin{gather*}
		\overline{W}(t_0 + \varepsilon) = \int_{t_0}^{t_0+\varepsilon} C X(t_0+\varepsilon,t)B(t)B^{\top}(t)X^{\top}(t_0+\varepsilon,t) C^\top \, dt=CW(\varepsilon)C^\top,
	\end{gather*}
	называется грамианом управляемости линейной системы 
	\begin{gather}\label{s2:linear_system}
		\dot{s} = A(t) s + B(t) u
	\end{gather}
	на интервале времени $ t_0 \leqslant t \leqslant t_0 + \varepsilon $ по выходу $y = C z$. 
	Здесь $X(\tau_1, \tau_0) = \Phi(\tau_1) \Phi^{−1}(\tau_0)$, где $\Phi(\tau)$ ---
	фундаментальная матрица решений однородной системы, удовлетворяющая уравнению
	$\dot{\Phi}(t) = A(t) \Phi(t)$, $\Phi(t_0) = I$.
\end{definition}

\begin{definition}\label{s1:def:linearized_system}
	Пусть $ x(\cdot,u(\cdot)) $ ~--- движение, отвечающее управлению $ u(\cdot)$.
	Назовем систему
	\begin{gather}\label{s1:linearized_system}
		\delta \dot{x} = A(t) \delta x + B(t) \delta u, \qquad t_0 \leqslant t \leqslant T, \qquad \delta x(t_0) = 0,
	\end{gather}
	{\textit линеаризацией} системы \eqref{s2:nonlinear_with_output} вдоль пары траектории и управления $\left( x(\cdot,u(\cdot)),u(\cdot)\right) $, если 
	\begin{gather*}
		A(t) = \dfrac{\partial f_1}{\partial x} \Big(t,x\big(t,u(\cdot)\big)\Big) 
		+ 
		\sum\limits_{k = 1}^{r}
		\dfrac{\partial f_2^k}{\partial x}\Big(t,x\big(t,u(\cdot)\big)\Big) u^k(t), \\
		B(t) = f_2 \Big(t,x\big(t,u(\cdot)\big)\Big).
	\end{gather*}
	Здесь $ A(\cdot) $ представляет собой матрицу Якоби функции $ f_1(\cdot, x) + f_2(\cdot, x) u(\cdot) $, вычисленную вдоль траектории $ x(\cdot,u(\cdot)) $.
\end{definition}

Обозначим через $\overline{\lambda}(\varepsilon)$ наименьшее собственное число грамиана управляемости линейной системы, соответствующей паре $(\varepsilon A(t), B(t))$  на интервале $ [0; 1]$.

\begin{theorem}\label{s2:th:assimptotic_equality}
	При достаточно малых $ \varepsilon $ множество достижимости $ \overline{G}(\varepsilon) $ системы \eqref{s2:nonlinear_with_output} по выходу $ y = C x $ выпукло и асимптотически эквивалентно множеству $\overline{W}^{1/2}(\varepsilon)B_{\mathbb{R}^m}(0,\mu) + Cx(t_0+\varepsilon,0)$, если найдутся такие $ K>0 $, $ \alpha > 0 $, $ 0< \varepsilon_0<\overline{\varepsilon} $, что для всех $ \varepsilon \leqslant \varepsilon_0 $
	\begin{gather}\label{s2:cond1}
		\overline{\lambda}(\varepsilon) \geqslant \left\{ {\begin{array}{*{20}{l}}
				{K\varepsilon ^{3 - \alpha}, \mbox{\ если \ } f_2(t,x) \mbox{\ не зависит от \ } x}, \\
				{K\varepsilon ^{1 - \alpha}}, \mbox{\ в противном случае}.
		\end{array}} \right.
	\end{gather}
\end{theorem}

В следующем разделе предлагается метод решения задачи локального синтеза для аффинных по управлению систем на малом интервале времени. 
В этом разделе изучается задача синтеза управления с интегральным квадратичным функционалом. 
\begin{gather}\label{s22:nonlinear}
	\dot{z}(t)=f(z(t))+B u(t),\qquad 0 \leqslant t \leqslant T, \qquad z(0) = z_0.
\end{gather}
где $ z \in \mathbb{R}^n $ --- вектор состояния, $ u \in \mathbb{R}^r $ --- вектор управления, $B \in \mathbb{R}^{n \times r}$, а $ 
T$ --- некоторое положительное число. 
Предполагается, что управление $u(\cdot)$ принадлежит пространству $\mathbb{L}_2 $.
\begin{assumption}\label{s22:as:solution_bounded}
	Существует такое $\mu > 0$, что все решения $x(s, \upsilon(\cdot))$ системы $\dot{x} = -f(x)-B\upsilon(t)$, выходящие из некоторой окрестности нуля и отвечающие управлениям $\upsilon(\cdot) \in B_{\mathbb{L}_2}(0, \mu)$, определены на интервале $[0, T]$ и лежат в шаре $ B_{\mathbb{R}^n}(0,\overline{r})$, $\overline{r} > 0$.
\end{assumption}
Здесь $ B_{\mathbb{R}^n}(0,\overline{r}) $ --- это шар радиуса $\overline{r}$ с центром в точке $0 \in \mathbb{R}^n$. 
Будем считать, что функция $f$ обладает следующим свойством.

\begin{assumption}\label{s22:as:Residial_term_bounds}
	Найдутся такие $\overline{r} > r >0$, $k>0$, что при всех $ z \in B_{\mathbb{R}^n}(0,r) $, функция $f(z)$ может быть представлена в форме $ f(z) = Az + R(z) $, причем $A \in \mathbb{R}^{n \times n}$ и $ \|R(z) \| \leqslant k \| z\|^2 $. 
\end{assumption}
Это свойство выполняется, если $f(0) = 0 $, $\frac{\partial f}{\partial z}(0) 
= A $ и $f(z)$ дважды непрерывно дифференцируема. 

В качестве функционала мы рассматриваем 
\begin{gather}\label{s22:cost}
	I(T,u):=\int_0^Tu^\top (t)u(t)dt= \lVert u(\cdot)\rVert^2_{\mathbb{L}_2.} 
\end{gather}
Задача состоит в синтезе закона управления $u(t)=u(t,z(t))$, который бы приводил траектории замкнутой системы 
\begin{gather*}
	\dot{z}(t)=f(z(t))+B u(t,z(t)),\qquad 0 \leqslant t \leqslant T, \qquad z(0) = z_0.
\end{gather*}
в начало координат за время $T$ и обеспечивал при этом минимальное значение $I(T,u)$. 
Для линейной системы 
\begin{gather}\label{s22:linear}
	\dot{z} = A z + B u, \qquad 0 \leqslant t \leqslant T.
\end{gather}
решение описанной выше задачи в случае, если пара $(A,B)$ управляема --- это линейный по состоянию закон управления
\begin{gather}\label{s22:linear_feedback}
	u(t,z) = -B^{\top} Q_T(t) z
\end{gather}
(см., например, \cite{Abgar,Kur1,GusevOsipov}).
Здесь $Q_T(t)=W^{-1}(T-t)$, а $W(t)$ --- грамиан управляемости системы $\dot{x} = -A x - B u$:
\begin{gather*}
	W(t) = \int_0^t e^{-A\tau}BB^\top e^{-A^{\top}\tau}d\tau. 
\end{gather*}
Через $\nu(t), \eta(t)$ обозначим наименьшее и наибольшее собственное число $W(t)$ соответственно. 
\begin{assumption} \label{s22:asm1} 
	Пусть существует $\overline{T}>0$ и непрерывная положительная функция $\varphi(\tau): (0, \overline{T}] \to \mathbb{R}$ такие, что
	\begin{gather*}
		0 < \frac{\eta(\tau)}{\sqrt{\nu(\tau)}} \leqslant \varphi(\tau), \qquad 0 < \tau \leqslant \overline{T},\quad \int\limits_0^ {\overline{T}}\varphi(\tau) d\tau<\infty.
	\end{gather*} 
\end{assumption}
\begin{assumption}\label{s22:asm2}
	Существует такое $ 0 < \beta \leqslant 1$ что $\frac{\sqrt{\eta(T)}}{\Phi^\beta(T)} \to 0$ при $T \to 0$.
\end{assumption}

Далее исследуется асимптотическое поведение траекторий системы \eqref{s22:nonlinear}, замкнутой линейной обратной связью $ u(t,z) = -B^{\top} Q_T(t) z$ при условии, что $T$ достаточно мало:
\begin{gather}\label{s22:nonlinear_closed}
	\dot{z} = f(z) - B B^{\top} Q_T(t) z, \qquad 0 \leqslant t \leqslant T, \qquad z(0) = z_0.
\end{gather}

\begin{theorem}\label{s22:th:tends_to_zero}
	Пусть выполнены предположения \ref{s22:asm1}, \ref{s22:asm2}. 
	Тогда существует такое $ T_1 \leqslant T^*$ что для всех $ T \leqslant T_1$, найдется такой $ r_1(T)$, что траектории системы \eqref{s22:nonlinear_closed} выходящие из $z(0) = z_0 \in B(0,r_1(T))$ стремятся к $0$ при $t \to T$.
\end{theorem}
\begin{theorem}\label{s22:th:functional_error_estimate}
	Пусть выполнено предположение \ref{s22:asm1}. 
	Пусть $x(t)$ --- такая траектория системы \eqref{s22:nonlinear_closed}, что $x(t)\in B(0,r)$, при $0\leqslant t \leqslant \tilde{T} \leqslant \overline{T} $ и $V_{\tilde{T}}(t,x(t))\to 0$ при $t\to \tilde{T}$. 
	Тогда существует такое $T_2 \leqslant \tilde{T}$, что для всех $0 < T < T_2 $ выполняется следующая оценка
	\begin{gather} \label{s22:est}
		\left| \frac{ J(T) - J_0(T)}{J_0(T)}\right| \leqslant 16k\Phi({T})J^{1/2}_0(T).
	\end{gather}
	Здесь $J(T)=J(T,x(\tilde{T}-T))$, $J_0(T)=J_0(T,x(\tilde{T}-T))$ --- значения функционала $I(T,u(\cdot))$ на траекториях нелинейной и линеаризованной систем соответственно.
\end{theorem}

Из результатов \cite{Polyak2001,GusevOsipovTrudy,Osipov,GusevMotor} следует, что множества достижимости 
систем \eqref{s22:nonlinear} и \eqref{s22:linear} асимптотически эквивалентны при $\tau \to 0$ если пара $(A,B)$ --- управляема 
и существуют такие $ l > 0$, $\tau_0 > 0$ и $\alpha > 0$, что для всех $0 < \tau \leqslant \tau_0 $
\begin{gather}\label{s22:gramas}
	\nu(\tau)\geqslant l\tau^{4-\alpha}.
\end{gather}

\begin{theorem}
	Пусть выполнено неравенство \eqref{s22:gramas}. 
	Пусть $T \leqslant \overline{T}$ и траектория $z(t)$ системы \eqref{s22:nonlinear_closed} стремится к нулю при $t\to T$. 
	Тогда $V_{T}(t,z(t)) =z^{\top}(t)Q(t)z(t) \to 0$ при $t \to T$.
\end{theorem}

\textbf{Глава 3.}
Предметом исследования в этой главе являются множества достижимости квазилинейных систем с интегральными квадратичными ограничениями
\begin{gather}\label{sec3:nonlinear}
	\begin{gathered}
	\dot{x}(t) = A(t)x(t)+B(t)u(t)+\varepsilon f\big(x(t),t\big), \\ t_0 \leqslant t \leqslant T, \qquad x(t_0) = x_0,
	\end{gathered}
\end{gather}
где $ x \in \mathbb{R}^n $ --- вектор состояния, $ u \in \mathbb{R}^r $ --- вектор управления, $t_0$ --- неотрицательное число, $T$ --- положительное число, а $\varepsilon$ --- малый параметр, такой, что $\varepsilon \in [0,\overline{\varepsilon}]$, $ \overline{\varepsilon} > 0$. 
Матричные отображения $A:[t_0,T] \to \mathbb{R}^{n\times n} $, $B: [t_0,T] \to \mathbb{R}^{n\times r} $ предполагаются непрерывными. 
Вектор-функция $f: \mathbb{R}^n \times [t_0,T] \to \mathbb{R}^n$ непрерывна по паре $(x, t)$ и непрерывно-дифференцируема по $x$.

Далее, мы будем считать, что Предположения \ref{s1:as:right_hand_side_conditions_global} и \ref{s1:as:right_hand_side_diff_lip} выполнены для правой части системы \eqref{sec3:nonlinear} при управлениях $ u(\cdot) \in B_{\mathbb{L}_2}(0,\mu) $ на интервале $ [t_0, T]$.

Как и в предыдущих главах, исследование сводится к анализу нелинейного отображения из пространства управлений в пространство концов траекторий, порожденных этими управлениями.
В таком случае, множество достижимости --- это образ гильбертова шара при применении к его точкам этого отображения. 
Основной особенностью отображения, определенного решением квазилинейной системы является тот факт, что при нулевом значении параметра это отображение становится линейным.

В параграфе 3.2 рассмотрен общий вид таких отображений
\begin{gather*} 
	F(x, \varepsilon) = a_0 + A_0x + \varepsilon A_1(x,\varepsilon): B_X(0, r) \times \mathbb{R}_+ \rightarrow Y,
\end{gather*} 
где $X$ и $Y$ --- гильбертовы пространства.
Здесь $a_0 \in Y$ --- константа, которая не зависит ни от $x$, ни от $\varepsilon$; $A_0$ --- линейный оператор, который предполагается сюръективным отображением $X$ в $Y$. 

\begin{assumption}\label{as:derivative_of_A1}
	Существует такое $\overline{\varepsilon} > 0$, что для всех $x \in B_X(0,r)$, $\varepsilon \in [0, \overline{\varepsilon}]$ отображение $A_1(x, \varepsilon)$ имеет производную Фреше $\frac{\partial A_1(x, \varepsilon)}{\partial x} = A_1'(x, \varepsilon)$ которая непрерывна по $\varepsilon$ и липшицева по $x$: существует $L>0$, такая что
	\begin{gather*}
		\|A_1'(x_1,\varepsilon) - A_1'(x_2,\varepsilon) \| \leqslant L\|x_1-x_2\|, \qquad x_1, x_2 \in B_X(0,r), \qquad \varepsilon \in [0, \overline{\varepsilon}].
	\end{gather*}
\end{assumption}

\begin{theorem}\label{th:ImageConvexity}
	Через $F\big(B_X(0,r),\varepsilon\big)$ обозначим образ шара $B_X(0, r)$ при его отображении $F$, т.~е. $F\big(B_X(0,r),\varepsilon\big) = \big\{F(x,\varepsilon): x\in B_X(0, r)\big\}$.
	Пусть выполнено Предположение \ref{as:derivative_of_A1} и $F\big(B_X(0,r),\varepsilon\big)$ --- замкнутое множество для каждого $\varepsilon \in [0, \overline{\varepsilon}]$. 
	Тогда найдется такое $ \varepsilon_0 \in (0, \overline{\varepsilon}]$, что при всех положительных $\varepsilon < \varepsilon_0$ образ $F\big(B_X(0,r),\varepsilon\big)$ --- выпуклое множество в $Y$. 
\end{theorem}

Параграф 3.3 посвящен применению Теоремы \ref{th:ImageConvexity} к отображению, определяемому системой \eqref{sec3:nonlinear}.

Определим отображение $F:B_{\mathbb{L}_2}(0,\overline{\mu})\times [0,\overline{\varepsilon}] \to \mathbb{R}^n$ равенством $F(u(\cdot),\varepsilon) = x(T,\varepsilon,u(\cdot))$, где $x(T,\varepsilon,u(\cdot))$ --- решение \eqref{sec3:nonlinear} в момент $T$, отвечающее управлению $u(\cdot)$ и малому параметру $\varepsilon$.

Для того, чтобы использовать Теорему \ref{th:ImageConvexity}, перепишем $F$ в виде
\begin{gather*}
	F(u(\cdot),\varepsilon) = a_0 + A_0 u(\cdot) + \varepsilon A_1(u(\cdot), \varepsilon), 
\end{gather*}
где $a_0 = X(T,0)x_0 $, а отображения $A_0: B_{\mathbb{L}_2}(0,\overline{\mu}) \mapsto \mathbb{R}^n$ и $A_1: B_{\mathbb{L}_2}(0,\overline{\mu}) \times [0,\overline{\varepsilon}] \to \mathbb{R}^n$ определены равенствами
\begin{gather}\label{A1_def}
	\begin{gathered}
	A_0 u(\cdot) = \int\limits_{t_0}^T X(T,\tau) B(t)u(\tau)\ d\tau, \\
	A_1(u(\cdot),\varepsilon) = \int\limits_{t_0}^T X(T,\tau) f\Big(x\big(\tau,\varepsilon, u(\cdot)\big),\tau\Big) \ d\tau.
	\end{gathered}
\end{gather}

Здесь через  $X(t,\tau)$ обозначена фундаментальная матрица системы $\dot{x}(t) = A(t) x(t)$.
Эта матрица является решением уравнения
\begin{gather*}
	\frac{\partial X(t,\tau)}{\partial t} = A(t) X(T,\tau), \qquad X(\tau,\tau) = I.
\end{gather*}




\begin{theorem}\label{th:ReachableSetsConvexity}
	Пусть выполнены условия Предположений \ref{s1:as:right_hand_side_conditions_global} и \ref{s1:as:right_hand_side_diff_lip}, тогда существует такое положительное значение $\varepsilon_0$, что множество достижимости $G(T,\mu,\varepsilon) $ квазилинейной системы \eqref{sec3:nonlinear} выпукло для всех $\varepsilon < \varepsilon_0$. 
\end{theorem}

\end{document}