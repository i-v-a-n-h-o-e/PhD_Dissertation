\documentclass[../main.tex]{subfiles}
\begin{document}
\pagebreak
\section*{Cодержание работы}
\textbf{Введение.} 
В этом разделе обосновывается актуальность темы исследования, приводится краткий обзор работ по данной тематике, определяются цель и задачи работы и формулируются основные положения, выносимые на защиту.


\textbf{Глава 1.}
Первая глава посвящена множествам достижимости нелинейных систем с интегральными ограничениями. 
Основной результат этого раздела сформулирован для нелинейной системы на малом интервале времени $\ t_0 \leqslant t \leqslant t_0 + \overline{\varepsilon} $.
\begin{gather}\label{s1:common_nonlinear_small_time}
	\dot{x}(t)=f_1(t,x(t))+f_2(t,x(t))u(t), \qquad x(t_0) = x_0.
\end{gather}
Здесь $ \overline{\varepsilon} $ ~--- некоторое фиксированное положительное число.

\begin{assumption}\label{s1:as:right_hand_side_conditions_global}
	Существует такое $\overline{\mu} > 0 $, что все решения (траектории) $ x(t, u(\cdot)) $ системы \eqref{s1:common_nonlinear_small_time}, отвечающие управлениям $u(\cdot) \in B_{\mathbb{L}_2[t_0, t_0 + \overline{\varepsilon}]}(0,\overline{\mu})$, определены на интервале $ [t_0,t_0 + \overline{\varepsilon}] $ и лежат в некотором выпуклом компакте $D \subset \Omega \subset \mathbb{R}^n$. 
\end{assumption}

Управление $u(\cdot)$ будем выбирать из пространства интегрируемых с квадратом функций $\mathbb{L}_2[t_0,t_0+\bar{\varepsilon}]$ и ограничим шаром радиуса $ 0 < \mu < \overline{\mu} $ в этом пространстве, который будем обозначать через $B_{\mathbb{L}_2}(0, \mu)$
\begin{gather*}
	\lVert u(\cdot)\rVert^2_{\mathbb{L}_2} = \left(u(\cdot),u(\cdot) \right) \leqslant \mu^2.
\end{gather*}

\begin{assumption}\label{s1:as:right_hand_side_diff_lip}
	Функции $f_1(t,x)$ и $f_2(t,x)$ имеют непрерывные производные по $x$, которые удовлетворяют условию Липшица при всех $t \in [t_0;t_0 + \varepsilon]$, $x_1, x_2 \in D$.
	\begin{gather*}
		\left\| \frac{\partial f_1}{\partial x}(t,x_1) - \frac{\partial f_1}{\partial x}(t,x_2) \right\| \leqslant l_{f_1} \| x_1 - x_2\|, \\
		 \left\| \frac{\partial f_2^k}{\partial x}(t,x_1) - \frac{\partial f_2^k}{\partial x}(t,x_2) \right\| \leqslant l_{f_2^k} \| x_1 - x_2\|,
	\end{gather*}
	где $l_{f_1} > 0$ и $l_{f_2^k} > 0$, $k = 1,...,r$.
\end{assumption}

\begin{definition}\label{s1:def:linearized_system}
	Пусть $ x(\cdot,u(\cdot)) $ ~--- --$ u(\cdot)$.
	Назовем систему
	\begin{gather}\label{s1:linearized_system}
		\delta \dot{x} = A(t) \delta x + B(t) \delta u, \qquad t_0 \leqslant t \leqslant T, \qquad \delta x(t_0) = 0,
	\end{gather}
	{\textit линеаризацией} системы \eqref{s1:common_nonlinear} вдоль пары траектории и управления $\left( x(\cdot,u(\cdot)),u(\cdot)\right) $, если 
	\begin{gather*}
		A(t) = \dfrac{\partial f_1}{\partial x} \Big(t,x\big(t,u(\cdot)\big)\Big) 
		+ 
		\sum\limits_{k = 1}^{r}
		\dfrac{\partial f_2^k}{\partial x}\Big(t,x\big(t,u(\cdot)\big)\Big) u^k(t), \ 
		B(t) = f_2 \Big(t,x\big(t,u(\cdot)\big)\Big).
	\end{gather*}
	Здесь $ A(\cdot) $ представляет собой матрицу Якоби функции $ f_1(\cdot, x) + f_2(\cdot, x) u(\cdot) $, вычисленную вдоль траектории $ x(\cdot,u(\cdot)) $.
\end{definition}

\begin{definition}\label{s1:def:grammian}
	Симметричная матрица, определённая равенством
	\begin{gather*}
		W(T) = \int\limits_{t_0}^{T}X(T,t)B(t)B^{\top}(t)X^{\top}(T,t) \, dt,
	\end{gather*}
	называется грамианом управляемости линейной системы $\dot{y} = A(t) y + B(t) u $ на интервале времени $ t_0 \leqslant t \leqslant T $.
	Как и в Определении \ref{s1:def:linearized_system}, здесь $ X(\tau_1,\tau_0)= \Phi(\tau_1) \Phi^{-1}(\tau_0) $, где $\Phi(t) $ ~--- фундаментальная матрица решений однородной системы, удовлетворяющая уравнению $ \dot{\Phi}(t) = A(t) \Phi(t)$, $ \Phi(t_0) = I $.
\end{definition}
\todo[inline]{Линейная система c $\varepsilon$, ее грамиан и его наименьшее собственное число для формулировки теоремы}
\begin{theorem}\label{s1:th:small_time_convexity}
	Если найдутся такие $K > 0$, $ \alpha > 0$, $ 0 < \varepsilon_0 < \overline{\varepsilon}$, что для всех $\varepsilon \leqslant \varepsilon_0$ выполняется условие
	\begin{gather}\label{s1:small_time_convexity_condition}
		\lambda(\varepsilon) \geqslant \left\{ {\begin{array}{*{20}{l}}
				{K\varepsilon ^{3 - \alpha}, \mbox{\ если \ } f_2(t,x) \mbox{\ не зависит от \ } x}, \\
				{K\varepsilon ^{1 - \alpha}}, \mbox{\ в противном случае},
		\end{array}} \right.
	\end{gather}
	то существует такое $ \varepsilon_1 > 0 $, что множество достижимости $G(\varepsilon)$ системы \eqref{s1:common_nonlinear_small_time} выпукло при всех $\varepsilon < \varepsilon_1 $.
\end{theorem}
\todo[inline]{ - Теорема 3.  \\ - Рекурентная процедура.  \\ - Нелинейные системы второго порядка }
		

\textbf{Глава 2.}
Вторая глава состоит из двух разделов, связанных общим понятием --- асимптотической эквивалентностью множеств достижимости. 

В первом разделе этой главы исследуются множества достижимости по выходу нелинейных систем на малых интервалах времени.
\todo[inline]{ - Определение асимптотической эквивалентности. \\ - Система 2.2.\\ - Опредение 6 и 7. \\ - Теорема 5.}
В следующем разделе предлагается метод решения задачи локального синтеза для аффинных по управлению систем на малом интервале времени. 
В этом разделе изучается задача синтеза управления с интегральным квадратичным функционалом. 
\todo[inline]{- Постановка задачи 2.2.1,\\ -  Предположение 3-4, 5,6.\\ - Теорема 6. \\ - Теорема 7.}

\textbf{Глава 3.}
Предметом исследования в этой главе являются множества достижимости квазилинейных систем с интегрально-квадратичными ограничениями
\begin{gather}\label{sec3:nonlinear}
	\begin{gathered}
	\dot{x}(t) = A(t)x(t)+B(t)u(t)+\varepsilon f\big(x(t),t\big), \\ t_0 \leqslant t \leqslant T, \qquad x(t_0) = x_0,
	\end{gathered}
\end{gather}
где $ x \in \mathbb{R}^n $ --- вектор состояния, $ u \in \mathbb{R}^r $ --- вектор управления, $t_0$ --- неотрицательное число, $T$ --- положительное число, а $\varepsilon$ --- малый параметр, такой, что $\varepsilon \in [0,\overline{\varepsilon}]$, $ \overline{\varepsilon} > 0$. 
Матричные отображения $A:[t_0,T] \to \mathbb{R}^{n\times n} $, $B: [t_0,T] \to \mathbb{R}^{n\times r} $ предполагаются непрерывными. 
Вектор-функция $f: \mathbb{R}^n \times [t_0,T] \to \mathbb{R}^n$ непрерывна по паре $(x, t)$ и непрерывно-дифференцируема по $x$.

Система \eqref{sec3:nonlinear} является частным случаем системы \eqref{s1:common_nonlinear_small_time}, которая исследовалась в Главе \ref{s1}, при $f_1(t,x) = A(t) x(t) + \varepsilon f(x(t),t)$ и $f_2(t, x(t)) = B(t)$.

Далее, мы будем считать, что Предположения \ref{s1:as:right_hand_side_conditions_global} и \ref{s1:as:right_hand_side_diff_lip} выполнены для правой части системы \eqref{sec3:nonlinear} при управлениях $ u(\cdot) \in B_{\mathbb{L}_2}(0,\mu) $ на интервале $ [t_0, T]$.

Как и в предыдущих главах, исследование сводится к анализу нелинейного отображения из пространства управлений в пространство концов траекторий, порожденных этими управлениями.
В таком случае, множество достижимости --- это образ гильбертова шара при применении к его точкам этого отображения. 
Основной особенностью отображения, определенного решением квазилинейной системы является тот факт, что при нулевом значении параметра это отображение становится линейным.

В параграфе 3.2 рассмотрен общий вид таких отображений
\begin{gather*} 
	F(x, \varepsilon) = a_0 + A_0x + \varepsilon A_1(x,\varepsilon): B_X(0, r) \times \mathbb{R}_+ \rightarrow Y,
\end{gather*} 
где $X$ и $Y$ --- гильбертовы пространства.
Здесь $a_0 \in Y$ --- константа, которая не зависит ни от $x$, ни от $\varepsilon$; $A_0$ --- линейный оператор, который предполагается сюръективным отображением $X$ в $Y$. 

\begin{assumption}\label{as:derivative_of_A1}
	Существует такое $\overline{\varepsilon} > 0$, что для всех $x \in B_X(0,r)$, $\varepsilon \in [0, \overline{\varepsilon}]$ отображение $A_1(x, \varepsilon)$ имеет производную Фреше $\frac{\partial A_1(x, \varepsilon)}{\partial x} = A_1'(x, \varepsilon)$ которая непрерывна по $\varepsilon$ и липшицева по $x$: существует $L>0$, такая что
	\begin{gather*}
		\|A_1'(x_1,\varepsilon) - A_1'(x_2,\varepsilon) \| \leqslant L\|x_1-x_2\|, \qquad x_1, x_2 \in B_X(0,r), \qquad \varepsilon \in [0, \overline{\varepsilon}].
	\end{gather*}
\end{assumption}

\begin{theorem}\label{th:ImageConvexity}
	Через $F\big(B_X(0,r),\varepsilon\big)$ обозначим образ шара $B_X(0, r)$ при его отображении $F$, т.~е. $F\big(B_X(0,r),\varepsilon\big) = \big\{F(x,\varepsilon): x\in B_X(0, r)\big\}$.
	Пусть выполнено Предположение \ref{as:derivative_of_A1} и $F\big(B_X(0,r),\varepsilon\big)$ --- замкнутое множество для каждого $\varepsilon \in [0, \overline{\varepsilon}]$. 
	Тогда найдется такое $ \varepsilon_0 \in (0, \overline{\varepsilon}]$, что при всех положительных $\varepsilon < \varepsilon_0$ образ $F\big(B_X(0,r),\varepsilon\big)$ --- выпуклое множество в $Y$. 
\end{theorem}

Параграф 3.3 посвящен применению Теоремы \ref{th:ImageConvexity} к отображению, определяемому системой \eqref{sec3:nonlinear}.

Обозначим фундаментальную матрицу системы $\dot{x}(t) = A(t) x(t)$ через $X(t,\tau)$.
Эта матрица является решением уравнения
\begin{gather*}
	\frac{\partial X(t,\tau)}{\partial t} = A(t) X(T,\tau), \qquad X(\tau,\tau) = I.
\end{gather*}

Если $x(\cdot,\varepsilon, u(\cdot))$ --- решение \eqref{sec3:nonlinear}, порожденное управлением $u(\cdot)$ и начальным условием $x_0$, то оно удовлетворяет интегральному уравнению
\begin{gather*}
	x\big(T,\varepsilon, u(\cdot)\big) =
	X(T,t_0)x_0 + 
	\int\limits_{t_0}^T X(T,\tau) \bigg(Bu(\tau) +
	\varepsilon f\Big(x\big(\tau,\varepsilon, u(\cdot)\big),\tau\Big) \bigg)\ d\tau = \\ =
	X(T,t_0)x_0 +
	\int\limits_{t_0}^T X(T,\tau) B(t)u(\tau)\ d\tau 
	+ \varepsilon \int\limits_{t_0}^T X(T,\tau) f\Big(x\big(\tau,\varepsilon, u(\cdot)\big),\tau\Big) \ d\tau.
\end{gather*}

Определим отображение $F:B_{\mathbb{L}_2}(0,\overline{\mu})\times [0,\overline{\varepsilon}] \to \mathbb{R}^n$ равенством $F(u(\cdot),\varepsilon) = x(T,\varepsilon,u(\cdot))$, где $x(T,\varepsilon,u(\cdot))$ --- решение \eqref{sec3:nonlinear} в момент $T$, отвечающее управлению $u(\cdot)$ и малому параметру $\varepsilon$.

Для того, чтобы использовать Теорему \ref{th:ImageConvexity}, перепишем $F$ в виде
\begin{gather*}
	F(u(\cdot),\varepsilon) = a_0 + A_0 u(\cdot) + \varepsilon A_1(u(\cdot), \varepsilon), 
\end{gather*}
где $a_0 = X(T,0)x_0 $, а отображения $A_0: B_{\mathbb{L}_2}(0,\overline{\mu}) \mapsto \mathbb{R}^n$ и $A_1: B_{\mathbb{L}_2}(0,\overline{\mu}) \times [0,\overline{\varepsilon}] \to \mathbb{R}^n$ определены равенствами
\begin{gather}\label{A1_def}
	\begin{gathered}
	A_0 u(\cdot) = \int\limits_{t_0}^T X(T,\tau) B(t)u(\tau)\ d\tau, \\
	A_1(u(\cdot),\varepsilon) = \int\limits_{t_0}^T X(T,\tau) f\Big(x\big(\tau,\varepsilon, u(\cdot)\big),\tau\Big) \ d\tau.
	\end{gathered}
\end{gather}




\begin{theorem}\label{th:ReachableSetsConvexity}
	Пусть выполнены условия Предположений \ref{s1:as:right_hand_side_conditions_global} и \ref{s1:as:right_hand_side_diff_lip}, тогда существует такое положительное значение $\varepsilon_0$, что множество достижимости $G(T,\mu,\varepsilon) $ квазилинейной системы \eqref{sec3:nonlinear} выпукло для всех $\varepsilon < \varepsilon_0$. 
\end{theorem}

\end{document}