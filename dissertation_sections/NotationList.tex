\documentclass[../main.tex]{subfiles}
\begin{document}
\clearpage
\section*{Список обозначений}
\addcontentsline{toc}{section}{Список обозначений}
\begin{tabularx}{\textwidth}{lY}
 $\mathbb{R}^n$ & пространство действительных векторов размерности $n$; \\ 
 $A^{\top}$ & транспонированная вещественная матрица $A$; \\ 
 $I$ & единичная матрица соответствующей размерности; \\ 
 $0$ & нулевой вектор или нулевая матрица соответствующей размерности; \\
 $(\cdot,\cdot)$ & скалярное произведение в евклидовом пространстве, $x \in \mathbb{R}^n$, $y \in \mathbb{R}^n$, $ \Big(x, y\Big) = x^{\top} y$; \\
 $\| \cdot\| $ & норма, в случае вектора $x \in \mathbb{R}^n$ имеется в виду евклидова норма, т.~е. $\|x\| = (x,x)^{\frac{1}{2}}$, а в случае вещественной матрицы $A$ размера $n \times n$ имеется ввиду ее спектральная норма, индуцированная евклидовой нормой вектора; \\
 $\mathbb{L}_1$, $\mathbb{L}_2$ & пространства интегрируемых и интегрируемых с квадратом функций соответственно; \\
 $\|\cdot\|_{\mathbb{L}_1}$, $\|\cdot\|_{\mathbb{L}_2}$ & нормы в пространствах интегрируемых и интегрируемых с квадратом функций соответственно; \\
 $ \mathbb{C} $ & пространство непрерывных вектор-функций; \\
 $\|\cdot\|_\mathbb{C}$ & норма в пространстве непрерывных функций, $\|f(\cdot) \|_\mathbb{C} = \max\limits_{t} \|f(t)\|$; \\
 $\mathcal{L}(\mathbb{L}_2, \mathbb{R}^n)$ & пространство линейных операторов, действующих из $\mathbb{L}_2$ в $\mathbb{R}^n$; \\ 
 $\| \cdot \|_{\mathcal{L}(\mathbb{L}_2, \mathbb{R}^n)}$ & операторная норма, $T: \mathbb{L}_2 \to \mathbb{R}^n $, $\| T \|_{\mathcal{L}(\mathbb{L}_2, \mathbb{R}^n)} = \sup\limits_{\|x\|_{\mathbb{L}_2} = 1} \| T x \| $;\\
 $(\cdot,\cdot)_{X}$ & скалярное произведение в гильбертовом пространстве $X$; \\
 $B_X(a,r)$ & замкнутый шар радиуса $r>0$ с центром в точке $a$, $B_X(a, r) = \{x\in X: \|x-a\|_X \leqslant r \}$. Здесь $X$ --- линейное пространство с нормой $\|\cdot\|_X$; \\
 $O_{X}(a, r)$ & открытый шар радиуса $r>0$ с центром в точке $a$, $O_{X}(a, r) = \{x\in X: \|x-a\|_X < r \}$. Как и в обозначении выше, здесь $X$ --- некоторое линейное пространство с нормой $\|\cdot\|_X$; \\
 $ A^* $ & сопряженный к $A$ линейный оператор, $(Ax, y) = (x, A^*y)$; \\
 $ \operatorname{int}A $ & внутренность множества $A$. \\
\end{tabularx}
\end{document}