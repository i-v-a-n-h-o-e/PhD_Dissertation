\documentclass[../main.tex]{subfiles}
\begin{document}
\clearpage
\section*{Введение}
\addcontentsline{toc}{section}{Введение}
Диссертация посвящена исследованию множеств достижимости нелинейных управляемых систем, содержащих малый параметр, при интегральных квадратичных ограничениях на управление.
Основное внимание уделяется изучению связи множеств достижимости нелинейных систем с множествами достижимости линеаризованных систем. 
Большая часть представленных результатов получена для систем на малом интервале времени, однако часть работы относится к системам с малым ограничением на управление или малой нелинейностью в правой части. 

\textbf{Актуальность темы исследования и степень ее разработанности.} 

Исследование множеств достижимости (МД) является одной из фундаментальных задач математической теории управления. 
МД, описывающие все возможные состояния, которые может достичь управляемая динамическая система при допустимых воздействиях из заданного начального состояния, служат ключевым инструментом для решения широкого круга задач в математической теории управления. 

В первую очередь, МД лежат в основе анализа и синтеза управляющих воздействий. 
В 1960–1970-х годах, в исследованиях Л.\,С.~Понтрягина, В.\,Г.~Болтянского, Р.\,В. ~Гамкрелидзе, Н.\,Н.~Красовского, А.\,И.~Субботина, А.\,Б.~Куржанского и других авторов \cite{Boltyansky, Pontryagin1961, Pontryagin1967, Gamkrelidze, Kras_book, KrasSub, Kurzhanski1977} были заложены основы современного подхода к анализу и синтезу управляющих воздействий на основе свойств множеств достижимости.
Существует тесная связь между МД и основными методами решения задач оптимального управления --- принципом максимума Понтрягина (ПМП) и методом динамического программирования (ДП).
Принцип максимума Понтрягина дает необходимые условия, которым обязаны удовлетворять траектории, лежащие на границе множества достижимости. 
Сопряженная система ПМП и условие максимума гамильтониана определяют направление движения вдоль этой границы, а условие трансверсальности связывает конечное состояние с нормалью к множеству достижимости \cite{Pontryagin1961, Lee}.
Метод динамического программирования опирается на исследование свойств функции цены, определяющей наилучшее значение функционала, достижимое из каждой точки пространства состояния. 
Множества уровня этой функции описывают множество достижимости, и их эволюция подчиняется уравнению Гамильтона–Якоби–Беллмана \cite{Bellman, Kurzhanski1977, GurmanDuhta, Mitchell2002, Osher, Sethian}. 

Кроме того, МД играют важную роль в анализе управляемости: вопрос о том, достижимо ли заданное состояние при существующих ограничениях на управление и начальное состояние, сводится к проверке принадлежности этого состояния множеству достижимости \cite{Kur1, Kurzhanski1977}.

В задачах верификации, безопасного управления и при анализе внешних возмущений, МД используются для оценки состояний, в которые система может попасть под воздействием внешних возмущений и для проверки недостижимости нежелательных областей \cite{Mitchell, Filippova2015}. 

В задачах гарантированного оценивания и наблюдения используются аналоги МД --- информационные множества. 
Эти множества содержат все возможные состояния, в которых может находиться система, подверженная влиянию неизвестных, но ограниченных возмущений, а состояние недоступно для полных и точных измерений \cite{Kurzhanski1977, Kur1, Schweppe, bi2000, Patsko2019}.

В зависимости от характера ограничений на управление принято различать два класса задач: с геометрическими (мгновенными) ограничениями, когда допустимое управление в каждый момент времени должно принадлежать фиксированному компакту в пространстве управлений, и с интегральными ограничениями (см., например, работы А.\,Б.~Куржанского, Б.\,Т.~Поляка, Х.\,Г.~Гусейнова, В.\,Н.~Ушакова, М.\,И.~Гусева, Е.\,К.~Костоусовой, В.\,С.~Пацко \cite{Kur1, Polyak2004, Guseinov2007, Guseinov2009, Guseinov2010, Guseinov2024, GusZyk, Kostousova,Patsko2023}), при которых ограничиваются не значения управления в отдельные моменты времени, а интеграл от некоторой функции управления. 
Эти два типа задач приводят к существенно различным по структуре множествам достижимости и требуют различных аналитических методов.

В диссертации рассматривается частный, но важный случай интегральных ограничений --- интегральные квадратичные ограничения на управление. 
В этом случае ограничена $\mathbb{L}_2$-норма управления. 
Такие ограничения можно трактовать как ограничения на энергию или ресурс, которые можно потратить на управление. 

Важной областью теории дифференциальных уравнений являются задачи с малым параметром.
Такие задачи возникают при исследовании систем с быстрыми и медленными подпроцессами, в задачах с высокой степенью усиления, при учете малых возмущений или нелинейностей, и представляют значительный теоретический и прикладной интерес. 
Принято выделять два класса таких задач: с регулярными и сингулярными возмущениями \cite{Tihonov1948,Tihonov1952}.

Регулярные возмущения характеризуются тем, что решение невозмущенной задачи (при значении параметра $\varepsilon = 0$) является хорошим приближением к решению возмущенной задачи (при малом $\varepsilon > 0$) во всей области определения. 
В этом случае решение можно представить в виде ряда по степеням малого параметра, и этот ряд равномерно сходится при $\varepsilon \to 0$.

В противоположность регулярным, сингулярные возмущения характеризуются тем, что решение невозмущенной задачи (вырожденной) не удовлетворяет всем условиям возмущенной задачи. 
При сингулярных возмущениях в решениях наблюдается явление пограничного слоя --- области, где решение быстро изменяется вблизи точек, где заданы дополнительные условия, исчезающие при вырождении.

В теории управления также изучаются задачи с динамическими системами, содержащими малый параметр. 
Вслед за теорией обыкновенных дифференциальных уравнений с малым параметром принято различать два основных класса таких задач: регулярные и сингулярные. 

Под сингулярно возмущенными системами управления обычно понимают системы, в которых малый параметр при старших производных (или другие параметры) приводит к вырождению порядка уравнений при $\varepsilon\to0$. 
В таких задачах редуцированная система обладает алгебраическим ограничением, требующим дополнительных условий («быстрые» и «медленные» переменные). 
В задачах оптимального управления сингулярность проявляется, например, когда весовой коэффициент при энергозатратах содержит $\varepsilon$ или при разнице масштабов управляющих сигналов. 
Обзору основных результатов, полученных при исследовании сингулярно возмущенных задач управления, посвящена работа \cite{Dmitriev}.
Перечислим здесь некоторые из них. 
В работах Ф.\,Л.~Черноусько \cite{Chernousko1968, Chernousko1977} впервые была получена аппроксимация для задач с ограничениями на управление.
Исследование \cite{Kokotovic} посвящено вопросам управляемости сингулярных линейных задач оптимального управления.
В публикациях \cite{Ilyin1989, Ilyin1998} было показано, что в задачах с гладкими геометрическими ограничениями на управление асимптотическое разложение решений может иметь очень сложную структуру. 
Исследование сингулярных возмущений в области дифференциальных включений и множеств достижимости представляет собой отдельную задачу, которой посвящены работы \cite{FilippovaKurzhansky, Veliov, GONCHAROVAOVSEEVICH}.

Под регулярными задачами (или задачами регулярной структуры) понимаются случаи, когда при $\varepsilon\to0$ система не вырождается и редуцированная модель не накладывает алгебраических ограничений. 
То есть, малый параметр появляется только в качестве гладкого возмущения. 
В этом случае возможно применение методов обычного анализа, таких как разложения в ряды, критерии оптимальности без специальных условий \cite{Haratishvili}. 
М.С. Никольский \cite{Nikolski} ввел понятие «очень регулярной задачи оптимального управления», характеризуя оптимизацию со «скользящим» малым параметром, при которой решения зависят гладко от $\varepsilon$. 
В таких задачах обычно выполняется принцип максимума Понтрягина без «сингулярных» участков, и анализ может вестись стандартными методами вариационного исчисления. 
К регулярным можно отнести, в частности, линейно-квадратичные задачи (LQR) при наличии малых возмущений или неопределенностей, депараметризованные задачи сингулярных регуляторов, оптимальные задачи с «плохой» и «хорошей» матрицами (без переключений) и др. 
Изучению некоторых асимптотических свойств множеств достижимости посвящена работа \cite{Chentsov}.
Такие задачи проще для численного анализа: для них сходимость стандартных схем сохраняется при малых $\varepsilon$.

К регулярным задачам можно отнести работы Б.\,Т.~Поляка, посвященные исследованию выпуклости множеств достижимости нелинейных систем при малом ресурсе управления.
Отметим, что эти исследования основаны на работе \cite{Polyak2001}, где были получены условия выпуклости гильбертова шара при его малом отображении. 
Этот результат был расширен на случай банаховых пространств в работе \cite{Ledyaev}.
Выпуклость множеств достижимости для обыкновенных дифференциальных уравнений изучалась в работе \cite{Reißig}, где представлен критерий и достаточное условие выпуклости МД из шара начальных условий.


Свойство выпуклости МД имеет исключительное значение в теории управления. 
Выпуклость позволяет применять развитый аппарат выпуклого анализа, включая теоремы разделения, двойственности и субдифференциального исчисления.
Выпуклое МД существенно упрощает решение целого ряда задач:
\begin{enumerate}
	\item Задачи минимизации выпуклых функционалов (время, энергия и т.д.) на выпуклом МД гарантированно имеют единственное решение, а методы выпуклой оптимизации (градиентные спуски, методы внутренней точки) могут быть эффективно использованы.
	\item Задачи управляемости и достижимости. 
	Проверка принадлежности целевого состояния выпуклому МД сводится к задаче выпуклого программирования, которая алгоритмически проще, чем аналогичная задача для невыпуклых множеств.
	\item Задачи синтеза управления. 
	Выпуклость МД часто позволяет строить управление в виде линейной обратной связи по состоянию или использовать принцип декомпозиции.
	\item Оценка и аппроксимация. 
	Выпуклое МД легче аппроксимировать (эллипсоидами, полиэдрами) с гарантированными оценками точности. 
	Это критически важно для приложений реального времени (навигация, робототехника, беспилотные системы), где требуется быстрый онлайн-расчет допустимых траекторий или зон безопасности. 
	Верификация безопасности становится значительно проще, когда множество достижимости выпукло — если внешняя аппроксимация выпуклого множества достижимости не пересекается с небезопасным множеством, можно доказать, что оно не может быть достигнуто ни одной траекторией. 
	Полиэдральная аппроксимация выпуклых множеств достижимости может быть выполнена с произвольной точностью, что значительно упрощает численный анализ.
\end{enumerate}

Важным направлением исследований является изучение структуры множеств достижимости на малых временных промежутках. 
Детальное исследование геометрической структуры таких множеств для систем с одним управлением в малых размерностях было проведено в работе \cite{Krener1989}. 
Затем, в работе \cite{Shattler1996}, была исследована связь между множествами достижимости на малых временах и задачами оптимального по времени управления с обратной связью.


Непосредственно изучению выпуклости МД на малом интервале времени посвящена первая глава диссертации, а исследования во второй главе основаны на полученном в первой главе условии выпуклости.
На основе условий выпуклости МД, полученных в первой главе диссертации, во второй главе подробнее изучается связь МД нелинейных и линеаризованных систем, а также использование этой связи для решения задачи синтеза. 

В теории управления хорошо развиты методы, разработанные для линейных систем. 
В ряде задач оптимального управления для линейных систем удается получать аналитические решения. 
В случае же нелинейных систем, аналитическое решение является скорее исключением, чем правилом. 
Поэтому в задачах управления нелинейными системами часто применяются решения, найденные в линеаризованной постановке. 
Иногда такой подход может быть строго обоснован. 
Например, согласно теореме об устойчивости по первому приближению (см., например, \cite{Barbashin_book}), из асимптотической устойчивости линеаризованной в окрестности положения равновесия системы следует устойчивость (локальная) исходной нелинейной системы. 
При решении задачи стабилизации это позволяет приближенно заменять нелинейную систему ее линеаризацией в окрестности положения равновесия. 
И если линеаризованная система окажется вполне управляемой (стабилизируемой), то линейная обратная связь, стабилизирующая эту систему, будет локально (в некоторой окрестности положения равновесия) стабилизировать и нелинейную систему\cite{Kras_add, Stab_lectures, Khalil, Polyak_book}. 
Однако зачастую метод линеаризации применяется без должного обоснования, так как соответствующие условия либо сложны для проверки, либо вообще отсутствуют.

Проведенный выше анализ литературы обосновывает актуальность тематики диссертации, призванной продолжить исследования выпуклости множеств достижимости нелинейных систем с малым параметром и интегральными ограничениями на управление. 
В первых двух главах диссертации рассматриваются нелинейные системы, аффинные по управлению, в которых малый параметр задает ограничения на управление или интервал времени, а в третьей главе малый параметр входит в правую часть. 
Для таких систем, с малой нелинейностью в правой части, отсутствовали доказанные условия выпуклости множеств достижимости при интегральных ограничениях на управление.

Системы с малой нелинейностью в правой части, которым посвящена третья глава, обычно называют квазилинейными. 
Изучение таких систем в теории управления началось еще в 1960-х годах \cite{Kiselev, Kras_book, Subbotin1967}.
Э.\,Г.~Альбрехт исследовал несколько задач для квазилинейных систем, в том числе задачу оптимального управления движением \cite{Albrecht1} и игровую задачу о встрече движений \cite{Albrecht2} при интегральных квадратичных ограничениях на управление.
Задачи управления квазилинейными системами также рассматриваются в следующих работах \cite{Dauer, Kremlev, KalininLavrinovich2018, Gabasov}.
Исследование А.\,Г.~Кремлева \cite{Kremlev} посвящено управлению системами с неопределенными начальными условиями, где делается акцент на построении оптимальных стратегий управления, причём управление также подчинено интегральным квадратичным ограничениям, как и в настоящей диссертации.
В работе \cite{KalininLavrinovich2018} предложен новый метод минимизации интегральных квадратичных функционалов на траекториях квазилинейных систем, а в статье Р.\,Ф.~Габасова и соавторов \cite{Gabasov} подробно рассматриваются вопросы оптимизации и устойчивости таких систем в прикладных задачах.
В современных приложениях теории управления, квазилинейные системы возникают при использовании линеаризации обратной связью и стохастической линеаризации \cite{Ching, Gui}.

\textbf{Цель и задачи исследования.} Изучение поведения множеств достижимости нелинейных систем, содержащих малый параметр, с интегральными  квадратичными ограничениями на управление и исследование их взаимосвязи с множествами достижимости линеаризованных систем.

\textbf{Методология и методы исследования.} Предлагаемые исследования основаны на результатах теории дифференциальных уравнений и математической теории управления, нелинейном и выпуклом анализе.
Полученные результаты иллюстрируются численными примерами.

\textbf{Основные положения, выносимые на защиту:} 
\begin{enumerate}
	\item Для аффинных по управлению систем с интегрально-квадратичными ограничениями на управление исследована выпуклость множеств достижимости на малом интервале времени.
	Получены достаточные условия выпуклости множеств достижимости в виде ограничений на асимптотику собственных чисел соответствующего грамиана управляемости линеаризованной системы.
	Разработан метод проверки этих условий, основанный на рекуррентной процедуре вычисления коэффициентов разложения грамиана управляемости в ряд по степеням малого параметра.
	Доказано, что нелинейные системы второго порядка, линеаризация которых приводит к линейным стационарным вполне управляемым системам, обладают выпуклыми множествами достижимости на малых интервалах времени.
	
	\item Для выпуклых компактов в $\mathbb{R}^n$, зависящих от малого параметра, введено понятие асимптотической эквивалентности, основанное на расстоянии Банаха-Мазура.
	Получены достаточные условия асимптотической эквивалентности множеств достижимости по выходу нелинейных систем с интегрально-квадратичными ограничениями на управление и соответствующих множеств линеаризованных систем. 
	Эти условия совпадают с достаточным условием выпуклости множеств достижимости на малых интервалах времени.
	
	\item Исследована задача синтеза управления для нелинейной аффинной по управлению системы с интегрально-квадратичным функционалом. 
	Доказано, что линейная обратная связь, построенная для линеаризованной системы, также обеспечивает локальное решение задачи синтеза для нелинейной системы на достаточно малом промежутке времени. 
	Это требует ограничений на асимптотику грамиана управляемости, которые совпадают с достаточными условиями, обеспечивающими асимптотическую эквивалентность множеств достижимости (множеств нуль-управляемости). 
	В этом случае, линейная обратная связь, решающая задачу приведения линеаризованной системы в начало координат за заданное время, будет приводить в начало координат и нелинейную систему из малой окрестности нуля.
	Также при этих условиях получена оценка для относительных значений погрешности интегрального функционала. 
	
	\item Для квазилинейных систем с интегральными квадратичными ограничениями на управление исследована выпуклость множеств достижимости. 
	Опираясь на достаточные условия выпуклости образа гильбертова шара при его квазилинейном отображении, доказано, что множества достижимости квазилинейной системы остаются выпуклыми при малых значениях малого параметра. 
\end{enumerate}


\textbf{Научная новизна.} Все полученные в работе результаты являются новыми.

\textbf{Теоретическая и практическая ценность работы.} Диссертация носит в основном теоретический характер.
Полученные результаты могут использоваться в дальнейших исследованиях задач управления с малым параметром и интегральными ограничениями, а также при разработке численных методов их решения.

\textbf{Степень достоверности и апробация результатов.} Степень достоверности результатов проведенных исследований подтверждается строгостью математических доказательств, а также проведенными вычислительными экспериментами.
Основные результаты, полученные в процессе исследования, докладывались и обсуждались на следующих конференциях:
\begin{enumerate}
	\item Воронежская весенняя математическая школа <<Современные методы теории краевых задач. Понтрягинские чтения–XXX>>, Воронеж, 3 -- 9 мая 2019 г.;
	\item Application of Mathematics in Technical and Natural Sciences \\ (AMiTaNS'11): 11th Intern. Conf., June 20 -- 25, 2019, Albena, Bulgaria;
	\item 51-я, 52-я, 53-я, 54-я и 55-я Международные молодежные школы-конференции "Современные проблемы математики и ее приложений" (2020, 2021, 2022, 2023, 2024), Екатеринбург;
	\item III международный семинар, посвященный 75-летию академика А.\,И.~Субботина, Екатеринбург, 26 -- 30 октября 2020 г.;
	\item Динамические системы: устойчивость, управление, оптимизация: материалы Междунар. науч. конф. памяти Р.\,Ф.~Габасова, Минск, 5 -- 10 окт. 2021 г.;
	\item XVI Международная конференция «Устойчивость и колебания нелинейных систем управления» (конференция Пятницкого), Москва, 1 -- 3 июня 2022 г.;
	\item Теория оптимального управления и приложения (OCTA 2022), Екатеринбург, 27 июня -- 1 июля 2022 г.;
	\item 7-я Международная школа-семинар «Нелинейный анализ и экстремальные задачи» (NLA-2022), 15 -- 22 июля 2022 г., Иркутск;
	\item 22nd International Conference Mathematical Optimization Theory and Operations Research (MOTOR 2023) Dedicated to 90th Anniversary of Academician I.\,I.~Eremin July 2 -- 8, 2023, Ekaterinburg;
	\item ВСПУ XIV Всероссийское совещание по проблемам управления, Россия, Москва, ИПУ РАН, 17 -- 20 июня 2024 г.;
	\item Международная конференция «Динамические системы: устойчивость, управление, дифференциальные игры» (SCDG2024), посвященная 100-летию со дня рождения академика Н.\,Н.~Красовского, 9 сентября 2024 г. -- 13 сентября 2024 г., г. Екатеринбург;
	\item Современные проблемы математики и ее приложений. Международная (56-я Всероссийская) молодежная школа-конференция памяти ученого и учителя Александра Георгиевича Гейна (29.01.1950 – 23.01.2025) 2 февраля 2025 г. -- 18 февраля 2025 г., г. Екатеринбург;
\end{enumerate}

\textbf{Публикации.} 
 Основные результаты по теме диссертации изложены в 19 научных работах \cite{AIP,GusevOsipovTrudy,OsipovVoronezhAbstract,OsipovAIPAbstract, SubbotinConf, Osipov, Voronezh, Minskconf, GusevOsipov, GusevOsipovPyat, GusevOsipovOCTA, GusevOsipovPyatAbstract, OsipovNLA, OsipovSopromat2022, GusevOsipovMotor, OsipovUMJ, OsipovSopromat2023, OsipovVSPU2024, OsipovSopromat2024}, из которых 5 изданы в научных журналах категории К1 \cite{GusevOsipovTrudy, Osipov, Voronezh, GusevOsipov, OsipovUMJ} перечня рецензируемых научных изданий ВАК или приравненных к ним. 
 Работы \cite{AIP, GusevOsipovMotor, GusevOsipovPyat} опубликованы в сборниках трудов международных
 научных конференций и проиндексированы в международных реферативных базах данных и системах цитирования.
Получено 1 свидетельство о государственной регистрации программ для ЭВМ \cite{Patent}.
Работа выполнена в рамках исследований, проводимых в Уральском математическом центре при финансовой поддержке Министерства науки и высшего образования Российской Федерации (номера соглашений 075-02-2024-1377 и 075-02-2025-1549).


\textbf{Личный вклад.} 
Все основные результаты диссертации получены автором самостоятельно. 
В совместных работах \cite{GusevOsipovTrudy, AIP, GusevOsipov, GusevOsipovMotor, GusevOsipovPyat, OsipovAIPAbstract, GusevOsipovPyatAbstract, SubbotinConf, Minskconf, GusevOsipovOCTA} научному руководителю М.\,И.~Гусеву принадлежат постановки задач и общая схема их исследования, формулировки и доказательства результатов принадлежат автору диссертации. 
В \cite{Patent} И.\,В.~Зыкову принадлежит идея и общая схема алгоритма, а автором диссертации предложен и программно реализован способ отбора управлений. 

\textbf{Структура и объем работы.}
Диссертация состоит из списка обозначений, введения, трех глав, заключения, списка литературы и двух приложений. 
Главы разбиты на разделы и подразделы.
Общий объем диссертации составляет \pageref{LastPage} страниц, библиографический список включает \total{bibcount} наименования. 
Нумерация формул двойная: в первой позиции указывается номер главы, в которой приведена формула, во второй — порядковый номер формулы в этой главе. 
Нумерация теорем, лемм, утверждений, следствий, предположений, замечаний, определений, --- сквозная. 
Все используемые обозначения объяснены в списке обозначений или в тексте диссертации там, где впервые встречаются.

\textbf{Краткое содержание работы.} \textbf{Глава 1. } 
Первая глава посвящена множествам достижимости нелинейных систем с интегральными ограничениями. 
Здесь даются основные понятия, описываются предположения и свойства решений исследуемых систем, которые используются в работе. 
Вводится отображение, действующее из пространства управлений в пространство концов траекторий нелинейной системы. 
Приводятся условия выпуклости образа малого шара в гильбертовом пространстве при его нелинейном отображении, полученные в \cite{Polyak2001}. 
Использование этого условия позволило получить условие выпуклости множеств достижимости при малых интегральных ограничениях на управление \cite{Polyak2004}.
Множество достижимости нелинейной системы рассматривается как образ шара, соответствующего допустимым управлениям, при его нелинейном отображении, заданном решениями системы.
При помощи замены времени удается сформулировать условия выпуклости множеств достижимости рассматриваемых нелинейных систем на малых интервалах времени. 
Проверка этих условий требует изучения асимптотики наименьшего собственного числа грамиана управляемости линеаризованной системы.
Один из возможных способов проверки асимптотики собственных чисел грамиана предлагается в последнем разделе этой главы. 
Там же приводится доказательство выпуклости множеств достижимости на малых интервалах времени для некоторых классов нелинейных систем второго порядка. 

\textbf{Глава 2. }
Вторая глава состоит из двух разделов, связанных общим понятием --- асимптотической эквивалентностью множеств достижимости. 

В первом разделе этой главы исследуются множества достижимости по выходу нелинейных систем на малых интервалах времени.
Используя понятие асимптотической эквивалентности множеств \cite{Ovs}, основанное на расстоянии Банаха-Мазура \cite{Thompson}, доказывается близость множеств достижимости нелинейных систем по выходу соответствующим множествам достижимости линеаризованных систем на малых интервалах времени. 

В следующем разделе предлагается метод решения задачи локального синтеза для аффинных по управлению систем на малом интервале времени. 
Этот метод основан на линеаризации исходной нелинейной системы в окрестности точки равновесия. 
Линеаризация часто применяется для решения различных задач управления, таких как задача стабилизации \cite{Kras_add,Khalil}, стохастические и численные методы управления \cite{Roxin,EKF,denBerg,Pang}, в подходах, основанных на предсказывающей модели (MPC) \cite{Murillo,LTV_MPC} и т.д.

В этом разделе изучается задача синтеза управления с интегральным квадратичным функционалом. 
Отдельно отметим, что задача рассматривается на конечном и, более того, малом интервале времени. 
Цель управления --- перевести систему в начало координат за заданное время, обеспечив при этом минимальное значение функционала. 
Линейное управление с обратной связью, найденное для линеаризованной системы, используется в качестве входа исходной нелинейной системы. 
Для линейной системы управления оптимальная обратная связь является линейной по состоянию, и ее коэффициент усиления неограниченно возрастает при приближении к конечному моменту времени. 
Последнее затрудняет обоснование применимости метода линеаризации. 
В этом случае необходимо выполнение условий, имеющих вид ограничений на асимптотику грамиана управляемости линеаризованной системы в отличие, например, от задачи стабилизации, в которой управляемость (стабилизируемость) линеаризованной системы приводит к стабилизируемости нелинейной системы. 
Эти условия совпадают с условиями асимптотической эквивалентности для множеств достижимости (множеств нуль-управляемости) нелинейных и линеаризованных систем, которые приведены в первой половине главы. 
В работе \cite{GusevOsipov} было показано, что при этих условиях управление в виде линейной обратной связи по состоянию приводит все траектории, выходящие из некоторой окрестности нуля, к нулю, если интервал времени управления достаточно мал. 

В диссертации приведено обобщение результата \cite{GusevOsipov}, которое было опубликовано в \cite{GusevOsipovMotor}. 
Предлагаемое достаточное условие имеет форму неравенства с некоторым несобственным интегралом.
Подынтегральная функция выбирается так, чтобы ограничивать сверху отношение наибольшего к корню из наименьшего собственного числа грамиана управляемости линеаризованной системы, которая содержит малый параметр. 
Выбор этой функции позволяет охватить более широкий класс систем управления, а условия из \cite{GusevOsipov} получены здесь как частный случай для определенного значения параметра.

Для линеаризованной системы рассматриваемый линейный регулятор обеспечивает минимальное значение интегрального функционала для любого начального состояния. 
Для нелинейной системы это не так, поэтому важно получить оценку значения функционала. 
Это и было сделано в заключительном разделе второй главы, где была исследована связь между значениями интегрального функционала для траекторий нелинейной и линеаризованной систем и дана оценка относительной погрешности. 

\textbf{Глава 3. }
Предметом исследования в этой главе являются множества достижимости квазилинейных систем с интегрально-квадратичными ограничениями.

Как и другие главы, третья глава опирается на работу Б.\,Т.~Поляка \cite{Polyak2001}, в которой были получены достаточные условия выпуклости нелинейного отображения малого гильбертова шара.
%These conditions were further applied to problems in control theory, demonstrating that the reachable set of a nonlinear system exhibits convexity given a sufficient small control resource, provided that the linearized system is controllable. A series of papers \cite{GusevOsipovTrudy, GusevUMJ, Osipov, GusevOsipov} used the convexity conditions of the small ball mapping to investigate the reachable sets of nonlinear systems under integral constraints over small time intervals. 
% In this case, it is important to note that the controllability of the linearized system alone does not guarantee convexity of the reachable sets for the nonlinear system. 
% Additional conditions related to the asymptotic behavior of the eigenvalues of the controllability Gramian of the linearized system need to be imposed. 
% Once these conditions are fulfilled, the reachable sets of the nonlinear system not only exhibit convexity but are also asymptotically equivalent to the reachable sets of the linearized system.

В предыдущих главах исследовались множества достижимости нелинейных систем с малым ресурсом управления или на малом интервале времени.
В этой главе обсуждается еще один, третий вариант задачи о выпуклости множеств достижимости нелинейных систем с малым параметром, а именно, с малой нелинейностью в правой части. 

В этой главе изучается выпуклость множеств достижимости квазилинейных систем с интегральными ограничениями в пространстве $\mathbb{L}_2$. 
Как и в предыдущих главах, исследование сводится к анализу нелинейного отображения из пространства управлений в пространство концов траекторий, порожденных этими управлениями.
В таком случае, множество достижимости --- это образ гильбертова шара при применении к его точкам этого отображения. 
Основной особенностью отображения, определенного решением квазилинейной системы является тот факт, что при нулевом значении параметра это отображение становится линейным.
Доказано, что для того, чтобы образ шара сохранял свою выпуклость при малых значениях малого параметра, достаточно, чтобы производная нелинейной части отображения была липшицевой. 
Схема доказательства этого утверждения похожа на схему доказательства основной теоремы в \cite{Polyak2001}.

\textbf{Заключение. }
В Заключении кратко описаны основные результаты, полученные в диссертации, и намечены перспективы дальнейших исследований в этом направлении.

\textbf{Приложения. }
В Приложении А описан метод построения множеств достижимости при интегрально-квадратичных ограничениях на управление, который используется для численных расчетов примеров в основном тексте диссертации. 
На программу, реализующую этот метод, было получено свидетельство о регистрации компьютерных программ для ЭВМ \cite{Patent}. 
Скан этого свидетельства приведен в Приложении B.
\pagebreak
\end{document}