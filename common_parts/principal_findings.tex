\begin{enumerate}	
	\item Для выпуклых компактов в $\mathbb{R}^n$, зависящих от малого параметра, введено понятие асимптотической эквивалентности, основанное на расстоянии Банаха-Мазура.
	Получены достаточные условия асимптотической эквивалентности множеств достижимости по выходу нелинейных систем с интегрально-квадратичными ограничениями на управление и соответствующих множеств линеаризованных систем. 
	Эти условия совпадают с достаточным условием выпуклости множеств достижимости на малых интервалах времени.
	
	\item Исследована задача синтеза управления для нелинейной аффинной по управлению системы с интегрально-квадратичным функционалом. 
	Доказано, что линейная обратная связь, построенная для линеаризованной системы, также обеспечивает локальное решение задачи синтеза для нелинейной системы на достаточно малом промежутке времени. 
	Это требует ограничений на асимптотику грамиана управляемости, которые совпадают с достаточными условиями, обеспечивающими асимптотическую эквивалентность множеств достижимости (множеств нуль-управляемости). 
	В этом случае, линейная обратная связь, решающая задачу приведения линеаризованной системы в начало координат за заданное время, будет приводить в начало координат и нелинейную систему из малой окрестности нуля.
	Также при этих условиях получена оценка для относительных значений погрешности интегрального функционала. 
	
	\item Для квазилинейных систем с интегральными квадратичными ограничениями на управление исследована выпуклость множеств достижимости. 
	Опираясь на достаточные условия выпуклости образа гильбертова шара при его квазилинейном отображении, доказано, что множества достижимости квазилинейной системы остаются выпуклыми при малых значениях малого параметра. 
\end{enumerate}