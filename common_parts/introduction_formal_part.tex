\textbf{Цель и задачи исследования.} Изучение поведения множеств достижимости нелинейных систем, содержащих малый параметр, с интегральными  квадратичными ограничениями на управление и исследование их взаимосвязи с множествами достижимости линеаризованных систем.

\textbf{Методология и методы исследования.} Предлагаемые исследования основаны на результатах теории дифференциальных уравнений и математической теории управления, нелинейном и выпуклом анализе.
Полученные результаты иллюстрируются численными примерами.

\textbf{Основные положения, выносимые на защиту:} 
\begin{enumerate}
	\item Для аффинных по управлению систем с интегрально-квадратичными ограничениями на управление исследована выпуклость множеств достижимости на малом интервале времени.
	Получены достаточные условия выпуклости множеств достижимости в виде ограничений на асимптотику собственных чисел соответствующего грамиана управляемости линеаризованной системы.
	Разработан метод проверки этих условий, основанный на рекуррентной процедуре вычисления коэффициентов разложения грамиана управляемости в ряд по степеням малого параметра.
	Доказано, что нелинейные системы второго порядка, линеаризация которых приводит к линейным стационарным вполне управляемым системам, обладают выпуклыми множествами достижимости на малых интервалах времени.
	
	\item Для выпуклых компактов в $\mathbb{R}^n$, зависящих от малого параметра, введено понятие асимптотической эквивалентности, основанное на расстоянии Банаха-Мазура.
	Получены достаточные условия асимптотической эквивалентности множеств достижимости по выходу нелинейных систем с интегрально-квадратичными ограничениями на управление и соответствующих множеств линеаризованных систем. 
	Эти условия совпадают с достаточным условием выпуклости множеств достижимости на малых интервалах времени.
	
	\item Исследована задача синтеза управления для нелинейной аффинной по управлению системы с интегрально-квадратичным функционалом. 
	Доказано, что линейная обратная связь, построенная для линеаризованной системы, также обеспечивает локальное решение задачи синтеза для нелинейной системы на достаточно малом промежутке времени. 
	Это требует ограничений на асимптотику грамиана управляемости, которые совпадают с достаточными условиями, обеспечивающими асимптотическую эквивалентность множеств достижимости (множеств нуль-управляемости). 
	В этом случае, линейная обратная связь, решающая задачу приведения линеаризованной системы в начало координат за заданное время, будет приводить в начало координат и нелинейную систему из малой окрестности нуля.
	Также при этих условиях получена оценка для относительных значений погрешности интегрального функционала. 
	
	\item Для квазилинейных систем с интегральными квадратичными ограничениями на управление исследована выпуклость множеств достижимости. 
	Опираясь на достаточные условия выпуклости образа гильбертова шара при его квазилинейном отображении, доказано, что множества достижимости квазилинейной системы остаются выпуклыми при малых значениях малого параметра. 
\end{enumerate}


\textbf{Научная новизна.} Все полученные в работе результаты являются новыми.

\textbf{Теоретическая и практическая ценность работы.} Диссертация носит в основном теоретический характер.
Полученные результаты могут использоваться в дальнейших исследованиях задач управления с малым параметром и интегральными ограничениями, а также при разработке численных методов их решения.

\textbf{Степень достоверности и апробация результатов.} Степень достоверности результатов проведенных исследований подтверждается строгостью математических доказательств, а также проведенными вычислительными экспериментами.
Основные результаты, полученные в процессе исследования, докладывались и обсуждались на следующих конференциях:
\begin{enumerate}
	\item Воронежская весенняя математическая школа <<Современные методы теории краевых задач. Понтрягинские чтения–XXX>>, Воронеж, 3 -- 9 мая 2019 г.;
	\item Application of Mathematics in Technical and Natural Sciences \\ (AMiTaNS'11): 11th Intern. Conf., June 20 -- 25, 2019, Albena, Bulgaria;
	\item 51-я, 52-я, 53-я, 54-я и 55-я Международные молодежные школы-конференции "Современные проблемы математики и ее приложений" (2020, 2021, 2022, 2023, 2024), Екатеринбург;
	\item III международный семинар, посвященный 75-летию академика А.\,И.~Субботина, Екатеринбург, 26 -- 30 октября 2020 г.;
	\item Динамические системы: устойчивость, управление, оптимизация: материалы Междунар. науч. конф. памяти Р.\,Ф.~Габасова, Минск, 5 -- 10 окт. 2021 г.;
	\item XVI Международная конференция «Устойчивость и колебания нелинейных систем управления» (конференция Пятницкого), Москва, 1 -- 3 июня 2022 г.;
	\item Теория оптимального управления и приложения (OCTA 2022), Екатеринбург, 27 июня -- 1 июля 2022 г.;
	\item 7-я Международная школа-семинар «Нелинейный анализ и экстремальные задачи» (NLA-2022), 15 -- 22 июля 2022 г., Иркутск;
	\item 22nd International Conference Mathematical Optimization Theory and Operations Research (MOTOR 2023) Dedicated to 90th Anniversary of Academician I.\,I.~Eremin July 2 -- 8, 2023, Ekaterinburg;
	\item ВСПУ XIV Всероссийское совещание по проблемам управления, Россия, Москва, ИПУ РАН, 17 -- 20 июня 2024 г.;
	\item Международная конференция «Динамические системы: устойчивость, управление, дифференциальные игры» (SCDG2024), посвященная 100-летию со дня рождения академика Н.\,Н.~Красовского, 9 сентября 2024 г. -- 13 сентября 2024 г., г. Екатеринбург;
	\item Современные проблемы математики и ее приложений. Международная (56-я Всероссийская) молодежная школа-конференция памяти ученого и учителя Александра Георгиевича Гейна (29.01.1950 – 23.01.2025) 2 февраля 2025 г. -- 18 февраля 2025 г., г. Екатеринбург;
\end{enumerate}

\textbf{Публикации.} 
 Основные результаты по теме диссертации изложены в 19 научных работах \cite{AIP,GusevOsipovTrudy,OsipovVoronezhAbstract,OsipovAIPAbstract, SubbotinConf, Osipov, Voronezh, Minskconf, GusevOsipov, GusevOsipovPyat, GusevOsipovOCTA, GusevOsipovPyatAbstract, OsipovNLA, OsipovSopromat2022, GusevOsipovMotor, OsipovUMJ, OsipovSopromat2023, OsipovVSPU2024, OsipovSopromat2024}, из которых 5 изданы в научных журналах категории К1 \cite{GusevOsipovTrudy, Osipov, Voronezh, GusevOsipov, OsipovUMJ} перечня рецензируемых научных изданий ВАК или приравненных к ним. 
 Работы \cite{AIP, GusevOsipovMotor, GusevOsipovPyat} опубликованы в сборниках трудов международных
 научных конференций и проиндексированы в международных реферативных базах данных и системах цитирования.
Получено 1 свидетельство о государственной регистрации программ для ЭВМ \cite{Patent}.
Работа выполнена в рамках исследований, проводимых в Уральском математическом центре при финансовой поддержке Министерства науки и высшего образования Российской Федерации (номера соглашений 075-02-2024-1377 и 075-02-2025-1549).


\textbf{Личный вклад.} 
Все основные результаты диссертации получены автором самостоятельно. 
В совместных работах \cite{GusevOsipovTrudy, AIP, GusevOsipov, GusevOsipovMotor, GusevOsipovPyat, OsipovAIPAbstract, GusevOsipovPyatAbstract, SubbotinConf, Minskconf, GusevOsipovOCTA} научному руководителю М.\,И.~Гусеву принадлежат постановки задач и общая схема их исследования, формулировки и доказательства результатов принадлежат автору диссертации. 
В \cite{Patent} И.\,В.~Зыкову принадлежит идея и общая схема алгоритма, а автором диссертации предложен и программно реализован способ отбора управлений. 

\textbf{Структура и объем работы.}
Диссертация состоит из списка обозначений, введения, трех глав, заключения, списка литературы и двух приложений. 
Главы разбиты на разделы и подразделы.
Общий объем диссертации составляет \pageref{LastPage} страниц, библиографический список включает \total{bibcount} наименования. 
Нумерация формул двойная: в первой позиции указывается номер главы, в которой приведена формула, во второй — порядковый номер формулы в этой главе. 
Нумерация теорем, лемм, утверждений, следствий, предположений, замечаний, определений, --- сквозная. 
Все используемые обозначения объяснены в списке обозначений или в тексте диссертации там, где впервые встречаются.